\documentclass[10pt]{beamer}
\usepackage[spanish]{babel}

\usetheme{Berlin}
\usecolortheme{seahorse}

%%%%%%%%%%%%%%%%%%%%%%%%%%%%%%%%%%
%%%%%%%%%%%%%%%%%%%%%%%%%%%%%   %%
%%        Datos Trabajo     %%  %%
%%%%%%%%%%%%%%%%%%%%%%%%%%%%%%%%%%
\newcommand{\titulo}[0]{Actividad 1. “Diseños de la investigación”}
\newcommand{\materia}[0]{Fundamentos de Investigación}
\newcommand{\grupo}[0]{BI-BFIN-2002-B2-013}
\newcommand{\unidad}[0]{Unidad 4}


%%%%%%%%%%%%%%%%%%%%%%%%%%%%%%%%%%
%%%%%%%%%%%%%%%%%%%%%%%%%%%%%%%%%%
\usepackage{amssymb}
\usepackage{enumerate}
\usepackage{geometry}
\usepackage{mathtools}
\usepackage{multicol}
\usepackage{soul}

\usepackage{graphicx}
	\graphicspath{ {assets/} }

\usepackage{hyperref}
	\hypersetup{
			pdftex,
		        pdfauthor={bench},
		        pdftitle={\titulo},
		        pdfsubject={\materia},
		        pdfkeywords={\grupo, \unidad, UnADM},
		        pdfproducer={Latex with hyperref, Ubuntu},
		        pdfcreator={pdflatex, or other tool},
			colorlinks=true,
				linkcolor=[rgb]{0,0,0.45},
				urlcolor=cyan,
				filecolor=green,
				citecolor=blue}

%%%%%%%%%%%%%%%%%%%%%%%%%%%%%%%%%%
%%%%%%%%%%%%%%%%%%%%%%%%%%%%%%%%%%

\title[Actividad 1]{
	%\includegraphics[width=3cm]{../../../assets/logo-unadm} \\
	\bf{\titulo}}

\author{ Benjamín Rivera }
\institute{Universidad Abierta y a Distancia de México \\
	\tiny
	TSU en Biotecnolog\'ia }
\date{\textit{Fecha de entrega:} \today}


%%%%%%%%%%%%%%%%%%%%%%%%%%%%%
%%        Documento         %%
%%%%%%%%%%%%%%%%%%%%%%%%%%%%%%%
\begin{document}
	\begin{frame}
		\maketitle\tiny
		\textit{Materia:} \materia \\
		\textit{Grupo:} \grupo \\
		\textit{Unidad:} \unidad \\
		\textit{Matricula:} ES202105994
		
	\end{frame}

	\begin{frame}{Tabla de contenidos}
		\tableofcontents
	\end{frame}

\section{\cite{inv 1}}
	\begin{frame}{\cite{inv 1}}
		\begin{description}
			\item[Problema o tema.] Producción de biocombustibles
			\item[Pregunta y supuestos.] Pruducción de bioenergéticos y su afectación en el suministro alimentos en México. Para esto se parte de que el SRM\footnote{Senado de la Repúplica Mexicana} ya promueve la producción de estos y que México es una de los grandes productres de productos agrícolas.
		\end{description}
	\end{frame}
	\begin{frame}{\cite{inv 1}}
		\begin{description}
			\item[Alcance y metodología.] Mediante el análisis del mercado americano se tratara de encontrar similaridades con el mercado méxicano para tratar de entender el futuru y proponer acciones para encaminar a la producción de biocombustibles en el país. 
			\par Esta fue una investigación cualitativa de tipo exploratorio; porque, acorde con \cite{basica}, busca examinar un tema que no esta bien delimitado del cual no se conoce completamente el estado del arte.
			
			
			\item[Instrumentos y fuentes.] Este artículo baso su trabajo en estudios de mercados externos y del estado del arte en nuestro país. Esto se puede notar en su bibliografía; podemos notar que este trabajo uso principalmente datos de fuentes secundarias.
		\end{description}
	\end{frame}










\section{\cite{inv 2}}
	\begin{frame}{\cite{inv 2}}
		\begin{description}
			\item[Problema o tema.] Producción de fármacos asisitdo por computadora.
			\item[Pregunta y supuestos.] En este trabajos e supone que la asistencia por parte de procedimientos computacionales durante el desarrollo de farmacos ayuda a que estos sean más efectivos y que reduce tanto los costos como los tiempos involucrados.
		\end{description}
	\end{frame}
	
	\begin{frame}{\cite{inv 2}}
		\begin{description}
			\item[Alcance y metodología.] Este artículo trata de entender el actual estado dle arte del desarrollo de fámacos asistido por computadora. Para esto se busca en los resultados actuales en donde se estudia el exito de otros y se lleva paso a paso del desarrollo de un medicamenteo contra el cancer para mostrar el resultado.
			\par De maera que esta es una investigación de cualitativa de tipo explicativo, esto porque la pregunta no esta delimitada y busca mostrar los avances existentes en el estado del arte.
			\item[Instrumentos y fuentes.] Las principales fuentes que usa este artículo son experimentos y productos desarrollados por otros investigadores. Por lo que se usaron fuentes secundarias.
		\end{description}
		
	\end{frame}










\section{\cite{inv 3}}
	\begin{frame}{\cite{inv 3}}
		\begin{description}
			\item[Problema o tema.] Uso de hongos como fertilizantes.
			\item[Pregunta y supuestos.] Este parte de que el uso de hongos como fertilizantes naturales es, al menos, tan efectivo como si usaran los fetilizantes \textit{clásicos}. Estos claramente con la ventaja de que son naturales.
		\end{description}
	\end{frame}
	\begin{frame}{\cite{inv 3}}
		\begin{description}
			\item[Alcance y metodología.] Este artículo se centro en estudiar la relación entre el crecimiento de las semillas de papaya \textit{(Carica papaya L.)} en convivnencia con la cepa de Hongos Micorrízicos Arbusculares \textit{(HMA)}.
			\par Esta es una investigación cuantitativa de tipo descriptivo; porque busca entender las condiciones óptimas para el uso del hongo  Los datos que usaron fueron de una fuente primaria, porque fueron obtenidos directamente de una experimentación que ellos desarrollaron.
			
			\item[Instrumentos y fuentes.] Este trabajo uso principalmente, como fuente de su argumentación, el análisis de un experimento realizado por ellos mismos.
		\end{description}
		
	\end{frame}




%%%%%%%%%%%%%%%%%%%%%%%%%%%%%%%%
%%         Bibliografia        %%
%%%%%%%%%%%%%%%%%%%%%%%%%%%%%%%%%%
\newpage
\scriptsize
\section{Referencias}
\begin{frame}{Referencias}
	\begin{thebibliography}{X}
		\bibitem[González, 2008]{inv 1} González Merino, A. (2008). Biocombustibles, biotecnología y alimentos: Impactos sociales para México. Argumentos (México, D.F.), 21(57), 55–83.
		\bibitem[Prada, 2016]{inv 2} Prada Gracia, D. (2016). Application of computational methods for anticancer drug discovery, design, and optimization. Boletín Médico Del Hospital Infantil de México, 73(6), 411–423. \url{https://doi.org/10.1016/j.bmhimx.2016.10.006}
		
		\bibitem[Quiñones et al, 2012]{inv 3} Quiñones Aguilar, E. E., Hernández-Acosta, E., Rincón-Enríquez, G., \& Ferrera-Cerrato, R. (2012). Interacción de hongos micorrízicos arbusculares y fertilización fosfatada en papaya. Terra Latinoamericana, 30(2), 165–176.
		
		\bibitem[UnADM, 2020]{basica} UnADM. (2020). Fundamentos de investigación Unidad 4. El diseño de investigación. 7 de noviembre de 2020, de UnADM Sitio web: \url{https://campus.unadmexico.mx/contenidos/DCSBA/TC/FIN/unidad_04/descargables/FIN_U4_Contenido.pdf}
	\end{thebibliography}
\end{frame}
\end{document}