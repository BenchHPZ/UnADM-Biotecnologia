\documentclass[14pt]{article}
\usepackage[spanish]{babel}

%%%%%%%%%%%%%%%%%%%%%%%%%%%%%%%%%%
%%%%%%%%%%%%%%%%%%%%%%%%%%%%%   %%
%%        Datos Trabajo     %%  %%
%%%%%%%%%%%%%%%%%%%%%%%%%%%%%%%%%%
\newcommand{\titulo}[0]{Autorreflexión. Unidad 2}
\newcommand{\materia}[0]{Estadística Básica}
\newcommand{\grupo}[0]{BI-BEBA-2002-B2-013}
\newcommand{\unidad}[0]{Unidad 2}


%%%%%%%%%%%%%%%%%%%%%%%%%%%%%%%%%%
%%%%%%%%%%%%%%%%%%%%%%%%%%%%%%%%%%
\usepackage{amssymb}
\usepackage{enumerate}
\usepackage{geometry}
\usepackage{mathtools}
\usepackage{multicol}
\usepackage{soul}

\usepackage{graphicx}
	\graphicspath{ {assets/} }

\usepackage{hyperref}
	\hypersetup{
			pdftex,
		        pdfauthor={bench},
		        pdftitle={\titulo},
		        pdfsubject={\materia},
		        pdfkeywords={\grupo, \unidad, UnADM},
		        pdfproducer={Latex with hyperref, Ubuntu},
		        pdfcreator={pdflatex, or other tool},
			colorlinks=true,
				linkcolor=[rgb]{0,0,0.45},
				urlcolor=cyan,
				filecolor=green,
				citecolor=blue}

%%%%%%%%%%%%%%%%%%%%%%%%%%%%%%%%%%
%%%%%%%%%%%%%%%%%%%%%%%%%%%%%%%%%%

\title{
	\includegraphics{../../../assets/logo-unadm} \\
	\ \\ Benjam\'in Rivera \\
	\bf{\titulo}\\\ \\}

\author{
	Universidad Abierta y a Distancia de México \\
	TSU en Biotecnolog\'ia \\
	\textit{Materia:} \materia \\
	\textit{Grupo:} \grupo \\
	\textit{Unidad:} \unidad \\
	\\
	\textit{Matricula:} ES202105994 }

\date{\textit{Fecha de entrega:} \today}


%%%%%%%%%%%%%%%%%%%%%%%%%%%%%
%%        Documento         %%
%%%%%%%%%%%%%%%%%%%%%%%%%%%%%%%
\begin{document}
\maketitle\newpage

	\par A lo largo de esta unidad estuvimos explorando distintas formas para poder comprender y analizar distintos datos. Pasando primero por formas completamente abstractas, como las fórmulas de las distribuciones de frecuencia; luego llegando a las representaciones gráficas, donde repasamos distintas gráficas; y por ultimo nos dieron distintas técnicas para hacer análisis de la información.
	
	\par Específicamente del primer punto, la distribución de frecuencias, podemos ver que nos dieron cuatro fórmulas relevantes. Las de 
	
%%%%%%%%%%%%%%%%%%%%%%%%%%%%%%%%
%%         Bibliografia        %%
%%%%%%%%%%%%%%%%%%%%%%%%%%%%%%%%%%

\begin{thebibliography}{X}
	\bibitem{EA anterior} Rivera C., B. (2020). \textit{Evidencia de Aprendizaje U1}. No Publicado.
	\bibitem{basico} Universidad Abierta y a Distancia de México. (s/f). \textit{Unidad 2. Representación numérica y gráfica de datos}. UnADM.

 
\end{thebibliography}

\end{document}