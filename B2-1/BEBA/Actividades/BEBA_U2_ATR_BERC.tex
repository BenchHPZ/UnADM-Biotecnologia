
\documentclass[12pt]{article}
\usepackage[spanish]{babel}

%%%%%%%%%%%%%%%%%%%%%%%%%%%%%%%%%%
%%%%%%%%%%%%%%%%%%%%%%%%%%%%%   %%
%%        Datos Trabajo     %%  %%
%%%%%%%%%%%%%%%%%%%%%%%%%%%%%%%%%%
\newcommand{\titulo}[0]{Autorreflexión. Unidad 2}
\newcommand{\materia}[0]{Estadística Básica}
\newcommand{\grupo}[0]{BI-BEBA-2002-B2-013}
\newcommand{\unidad}[0]{Unidad 2}

for
%%%%%%%%%%%%%%%%%%%%%%%%%%%%%%%%%%
%%%%%%%%%%%%%%%%%%%%%%%%%%%%%%%%%%
\usepackage{amssymb}
\usepackage{enumerate}
\usepackage{geometry}
\usepackage{mathtools}
\usepackage{multicol}
\usepackage{soul}

\usepackage{graphicx}
	\graphicspath{ {assets/} }

\usepackage{hyperref}
	\hypersetup{
			pdftex,
		        pdfauthor={bench},
		        pdftitle={\titulo},
		        pdfsubject={\materia},
		        pdfkeywords={\grupo, \unidad, UnADM},
		        pdfproducer={Latex with hyperref, Ubuntu},
		        pdfcreator={pdflatex, or other tool},
			colorlinks=true,
				linkcolor=[rgb]{0,0,0.45},
				urlcolor=cyan,
				filecolor=green,
				citecolor=blue}

%%%%%%%%%%%%%%%%%%%%%%%%%%%%%%%%%%
%%%%%%%%%%%%%%%%%%%%%%%%%%%%%%%%%%

\title{
	\includegraphics{../../../assets/logo-unadm} \\
	\ \\ Benjam\'in Rivera \\
	\bf{\titulo}\\\ \\}

\author{
	Universidad Abierta y a Distancia de México \\
	TSU en Biotecnolog\'ia \\
	\textit{Materia:} \materia \\
	\textit{Grupo:} \grupo \\
	\textit{Unidad:} \unidad \\
	\\
	\textit{Matricula:} ES202105994 }

\date{\textit{Fecha de entrega:} \today}


%%%%%%%%%%%%%%%%%%%%%%%%%%%%%
%%        Documento         %%
%%%%%%%%%%%%%%%%%%%%%%%%%%%%%%%
\begin{document}
\maketitle\newpage

	\par A lo largo de esta unidad estuvimos explorando distintas formas para poder comprender y analizar distintos datos. Pasando primero por formas completamente abstractas, como las fórmulas de las distribuciones de frecuencia; luego llegando a las representaciones gráficas, donde repasamos distintas gráficas; y por ultimo nos dieron distintas técnicas para hacer análisis de la información.
	
	\par Específicamente del primer punto, la distribución de frecuencias, podemos ver que nos dieron cuatro fórmulas relevantes. Las de frecuencia absoluta y porcentual, y para cada una de estas sus partes individuales y acumuladas. La elección correcta de cada una de ellas dependerá de lo que buscamos de los datos; en los casos en que busquemos conocer datos fríos nos convendrá las frecuencias absolutas, mientras que en las que nos interese más conocer la relación de cierta clase con los datos totales. Por otro lado, el uso de las versiones acumuladas de estas dependerá de si estamos buscando el comportamiento individual o de la variación relacionada con el comportamiento global.
	
	\par Una vez que ya calculamos los datos anteriores, ahora toca mostrarlos a todas las personas. En esto es útil usar gráficas, ya que nos permite poder entender el comportamiento de los datos sin tener que leer los datos individualmente; para que esta representación sea lo más útil posible, es necesario escoger la gráfica correcta. En la unidad revisamos distintos tipos de gráficas, entre estos encontramos los histogramas, la gráfica de barras, el polígono de frecuencias, la gráfica de pastel y la ojiva. Personalmente prefiero la ojiva y la gráfica de pastel para los datos acumulados, aunque las gráficas circulares están empezando a ser bastante desmeritadas. Por otro lado, para las versiones no acumuladas me gusta la gráfica de pastel, con las frecuencias relativas, y los gráficos de barra junto con los histogramas, para las frecuencias absolutas.
	
	\par En lo personal, para los dos puntos anteriores, yo use conocimientos previos que tenía para manejar grandes bases de datos y poder hacer ese análisis de una manera más eficiente, también hubiera tratado de hacer gráficos un poco más interesantes como los que se pueden apreciar en \cite{graficas}.
	
	\par Para el último punto, que es el análisis de datos, se empieza a poner más interesante. Aunque, según entendí, esto depende bastante de la intuición, y esto solo se puede trabajar con la experiencia. De manera que, como mencione al inicio, el uso de cada uno dependerá de lo que la intuición nos diga, y de como salgan las conclusiones.
	
	\par Como conclusión, el análisis de datos es algo que debe practicarse, y cuanto más lo hagamos, nos permitirá ser más eficientes al escoger las variables correctas, junto con las mejores representaciones y poder hacer un análisis adecuado.
	
%%%%%%%%%%%%%%%%%%%%%%%%%%%%%%%%
%%         Bibliografia        %%
%%%%%%%%%%%%%%%%%%%%%%%%%%%%%%%%%%

\begin{thebibliography}{X}
	\bibitem{EA anterior} Rivera C., B. (2020). \textit{Evidencia de Aprendizaje U1}. No Publicado.
	\bibitem{basico} Universidad Abierta y a Distancia de México. (s/f). \textit{Unidad 2. Representación numérica y gráfica de datos}. UnADM.

	\bibitem{graficas} Korpela, K. (s. f.). \textit{The Art of Data Visualization: A Gift or a Skill?, Part 1. } ISACA. \url{https://www.isaca.org/resources/isaca-journal/issues/2016/volume-1/the-art-of-data-visualization-a-gift-or-a-skill-part-1}
 
\end{thebibliography}

\end{document}