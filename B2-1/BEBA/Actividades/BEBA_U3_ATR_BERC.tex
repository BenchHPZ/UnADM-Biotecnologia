
\documentclass[12pt]{article}
\usepackage[spanish]{babel}

%%%%%%%%%%%%%%%%%%%%%%%%%%%%%%%%%%
%%%%%%%%%%%%%%%%%%%%%%%%%%%%%   %%
%%        Datos Trabajo     %%  %%
%%%%%%%%%%%%%%%%%%%%%%%%%%%%%%%%%%
\newcommand{\titulo}[0]{Autorreflexión. Unidad 3}
\newcommand{\materia}[0]{Estadística Básica}
\newcommand{\grupo}[0]{BI-BEBA-2002-B2-013}
\newcommand{\unidad}[0]{Unidad 3}


%%%%%%%%%%%%%%%%%%%%%%%%%%%%%%%%%%
%%%%%%%%%%%%%%%%%%%%%%%%%%%%%%%%%%
\usepackage{amssymb}
\usepackage{enumerate}
\usepackage{geometry}
\usepackage{mathtools}
\usepackage{multicol}
\usepackage{soul}

\usepackage{graphicx}
	\graphicspath{ {assets/} }

\usepackage{hyperref}
	\hypersetup{
			pdftex,
		        pdfauthor={bench},
		        pdftitle={\titulo},
		        pdfsubject={\materia},
		        pdfkeywords={\grupo, \unidad, UnADM},
		        pdfproducer={Latex with hyperref, Ubuntu},
		        pdfcreator={pdflatex, or other tool},
			colorlinks=true,
				linkcolor=[rgb]{0,0,0.45},
				urlcolor=cyan,
				filecolor=green,
				citecolor=blue}

%%%%%%%%%%%%%%%%%%%%%%%%%%%%%%%%%%
%%%%%%%%%%%%%%%%%%%%%%%%%%%%%%%%%%

\title{
	\includegraphics{../../../assets/logo-unadm} \\
	\ \\ Benjam\'in Rivera \\
	\bf{\titulo}\\\ \\}

\author{
	Universidad Abierta y a Distancia de México \\
	TSU en Biotecnolog\'ia \\
	\textit{Materia:} \materia \\
	\textit{Grupo:} \grupo \\
	\textit{Unidad:} \unidad \\
	\\
	\textit{Matricula:} ES202105994 }

\date{\textit{Fecha de entrega:} \today}


%%%%%%%%%%%%%%%%%%%%%%%%%%%%%
%%        Documento         %%
%%%%%%%%%%%%%%%%%%%%%%%%%%%%%%%
\begin{document}
\maketitle\newpage

\noindent
\textbf{Respecto a las medidas de tendencia central y dispersión, ¿Consideras que son importantes para el estudio estadístico? ¿Por qué?}

\par \
\par Me parece que son una de la parte más importates que esta disciplina ofrece, además de que con estas\footnote{O al menos sus contrapartes más formales} son parte de los fundamentos para técnicas más avanzadas. El uso de estas permite poder entender la naturalidad y distribución de los datos sin la necesidad de tener la capacidad de poder dimensionar todos los datos, lo cual no parece fácil ni práctico; más teniendo en consideración que la cantidad de datos, que se deben tomar en cuenta, aumenta considerablemente con proyectos reales.

\par \ \\\ \\ \noindent
\textbf{Respecto a los temas de todo el curso, ¿Te consideras preparado para realizar un estudio estadístico por tu cuenta?}

\par \
\par A pesar de que considero que este curso es un muy buen acercamiento a las partes esenciales de lo que se necesita para tener la capacidad de realzar un estudio estadístico, únicamente conel contenido de este curso, yo aún no me sentiría preparado para dirigir uno. Creo que el incluir más actividades prácticas, con diferentes situaciones reales, nos permitirían comprender de manera más profunda los témas que se revisaron en el curso e identificar los temas que no hayamos comprendido del todo.

\par \ \\\ \\ \noindent
\textbf{Durante el desarrollo de este curso, ¿Cambiaste tu perspectiva de la estadística básica?}

\par \
\par No. Yo ya había tenido la oportunidad de revisar estos temas a profundidad en el bachillerato y luego, con aún más profundidad y rigurosidad, en mi primer licenciatura; por lo que estos temas no representan novedad alguna para mi.
\par A pesar de lo anterior, me parece que es un curso que esta bien preparado desde el punto de las temáticas que aborda; además de que la profundidad de los temas me parece adecuada para alguien que no se vaya a encargar de dirigir un equipo estadístico, o que requiera de usar métodos más complejos. 
\par Aunque creo que sería muy beneficiado si fuera separado en dos bloques para que, en su unión de un semestre, pudieran ser revisados los témas con más profundidad y que permita incluir más actividades prácticas que nos ayuden a reforzar los conocimientos del curso.

\par \ \\ \noindent
\textbf{¿Cuál es tu opinión sobre este curso y que sugerencias le harías a tu docente para una versión posterior del mismo?}

\par \
\par Creo que sería bueno considerar incluir en este curso más herramientas computacionales para apoyar los temas; y no incluir únicamente herramientas con \textit{interfaz gráfica}. Los distintos lenguajes de programación con enfoque directo a esta area ofrecen herramientas muy útiles que claramente serían útiles en cualquier proyecto de la vida diaria.




\section*{Conclusión}

\par En conclusión, este curso me parece que esta bien preparado y cumple con lo necesario para un primer curso del área. Existen ciertos puntos de mejora, como
	\begin{itemize}
		\item la inclusión en el programa de un lenguaje de programación con enfoque en aplicaciones estadísticas,
		\item la ampliación del tiempo asignado a este curso y
		\item la inclusión de más actividades prácticas con casos y datos reales;
	\end{itemize}
	
	\noindent pero estos puntos son únicamente de mejora, el curso ya es bueno.

%%%%%%%%%%%%%%%%%%%%%%%%%%%%%%%%
%%         Bibliografia        %%
%%%%%%%%%%%%%%%%%%%%%%%%%%%%%%%%%%

\begin{thebibliography}{X}
	\bibitem{EA anterior} Rivera C., B. (2020). \textit{Evidencia de Aprendizaje U2}. No Publicado.
	\bibitem{basico} Universidad Abierta y a Distancia de México. (s/f). \textit{Unidad 3. Representación numérica y gráfica de datos}. UnADM.

	\bibitem{graficas} Korpela, K. (s. f.). \textit{The Art of Data Visualization: A Gift or a Skill?, Part 1. } ISACA. \url{https://www.isaca.org/resources/isaca-journal/issues/2016/volume-1/the-art-of-data-visualization-a-gift-or-a-skill-part-1}
 
\end{thebibliography}

\end{document}