\documentclass[12pt]{article}
\usepackage[spanish]{babel}

%%%%%%%%%%%%%%%%%%%%%%%%%%%%%%%%%%
%%%%%%%%%%%%%%%%%%%%%%%%%%%%%   %%
%%        Datos Trabajo     %%  %%
%%%%%%%%%%%%%%%%%%%%%%%%%%%%%%%%%%
\newcommand{\titulo}[0]{Actividad 1. Dimensiones del hombre.}
\newcommand{\materia}[0]{Desarrollo Humano}
\newcommand{\grupo}[0]{BI-BDHU-2002-B2-012}
\newcommand{\unidad}[0]{Unidad 2}


%%%%%%%%%%%%%%%%%%%%%%%%%%%%%%%%%%
%%%%%%%%%%%%%%%%%%%%%%%%%%%%%%%%%%
\usepackage{amssymb}
\usepackage{enumerate}
\usepackage{geometry}
\usepackage{mathtools}
\usepackage{multicol}
\usepackage{soul}

\usepackage{graphicx}
	\graphicspath{ {assets/} }

\usepackage{hyperref}
	\hypersetup{
			pdftex,
		        pdfauthor={bench},
		        pdftitle={\titulo},
		        pdfsubject={\materia},
		        pdfkeywords={\grupo, \unidad, UnADM},
		        pdfproducer={Latex with hyperref, Ubuntu},
		        pdfcreator={pdflatex, or other tool},
			colorlinks=true,
				linkcolor=red,
				urlcolor=cyan,
				filecolor=green,
				citecolor=blue}

%%%%%%%%%%%%%%%%%%%%%%%%%%%%%%%%%%
%%%%%%%%%%%%%%%%%%%%%%%%%%%%%%%%%%

\title{
	%\includegraphics{../../../assets/logo-unadm}
	\ \\\ \\\
	\ \\ Benjam\'in Rivera \\
	\bf{\titulo}\\\ \\}

\author{
	{\Huge Universidad Abierta y a Distancia de México} \\
	TSU en Biotecnolog\'ia \\
	\textit{Materia:} \materia \\
	\textit{Grupo:} \grupo \\
	\textit{Unidad:} \unidad \\
	\\
	\textit{Matricula:} ES202105994 }

\date{\textit{Fecha de entrega:} \today}


%%%%%%%%%%%%%%%%%%%%%%%%%%%%%
%%        Documento         %%
%%%%%%%%%%%%%%%%%%%%%%%%%%%%%%%
\begin{document}
\maketitle\newpage

\subsection*{Carmen}
	
	\begin{quote}\begin{description} 
		\item [Biol\'ogica] La se\~nora Carmen internacional  es una se\~nora de edad avanzada. Tiene apariencia  fr\'agil y no aparenta gozar  de una buena condici\'on física; a pesar de todo lo anterior parece ser muy activa y constantemente se esfuerza bastante f\'isicamente. Ella sufre de dolor de la espalda y de las distintas dolencias relacionadas con la edad. Es una se\~nora morena, bajita y parece un poco pasada de peso, adem\'as que parece cojear. Tiene pelo negro entre ondulado y lacio. Adem[as tiene ojos obscuros y chicos.
		\item [Psicol\'ogica] La se\~nora Carmen parece ser una person \textit{honrada}, con mucha \'etica en el trabajo y con respeto por las demas personas. Tambi\'en es una persona con mucha tolerancia, en ocasiones extrema. La se\~nora parece tener como proncipal preocipaci\'on el apoyar a su familia, lo que parece indicar que es una persona solidariay responsable con aquellos que dependen de ella. Aunque al final del capitulo mostro que su autocontrol no era lo suficiente como para tratar de tomar decisiones importantes con la cabeza fria.
		\item [Social] La se\~nora Carmen se comoprta de manera cordialmente con todos, se interesa por el bienestar de todas las personas que la rodean; aunqe no todas ellas sean también cordiales con ella como la se\~nora Susana. Adem\'as de esto, parece ser, que la principal motivaci\'on para hacer cosas en su vida es procurar el bienestar de su famila; eso se demuestra al tratar de apoyar a su hijo para dejar el alcohl y que la principal raz\'on por la cual \textit{soporta} a la se\`nora Susana es porque quiere que su familia tenga dinero para comer.
	\end{description}\end{quote}


\subsection*{Susana}
	
	\begin{quote}\begin{description}
		\item [Biol\'ogica] La se\~nora Susana parece una mujer de edad media. Es de tez clara; pelo, que parece pintado con rayos, en tonalidades claras y es claramente lacio; con ojos claros y bastante resaltados por las facciones de su cara; ella posee una voz chirreante y contundente, con la cual suele hablar bastante r\'apido; usa bastante maquillaje para resaltar sus cejas. La se\~nora es considerablemente m\'as grande que la se\~nora Carmen. Parece que no padece de alg\'una dolencia f\'isica, o al menos no lo muestra en el episodio.
		\item [Psicol\'ogica] Al inicio del capitulo la se\~nora Susana parec\'ia \textit{relativamente} normal. En cuanto se empieza a desarrollar el capitulo, podemos ver que la se\~nora tiene una sencaci\'on de ser superior a la se\~nora Carmen y nos terminamos dando cuenta de que la discrimina por su apariencia f\'isica. Se nos dice que ella ha sufrido un trauma psicol\'ogico que pdr\'ia ser la raz\'on de su comportamiento err\'atico, el abandono por parte de su esposo. De esto podemos ver que ella tiene miedo de las personas, de que la traicionen como sintio que hiz\'o su esposo con ella.
		\item [Social] La se\~nora Susana, como mencionamos anteriormente, parece tener una sensaci\'on de superioridad por factores irreelvantes, como la cantidad de dineo de la que dispone y su apariencia f\'isica, lo que incluye como su color de piel, de cabello y forma f\'isica. Esta sensaci\'on de falsa superioridad la lleva comportarse de manera err\'atica, donde su forma de ser se vuelve irrespetusosa, impsetiva y prepotente con quienes la rodean, en especial con la se\~nora Carmen. Ella no parece sentirse responsable al respecto de su forma de actuar, a pesar de que su hijo s elo hace notar en repetidas situaciones, parec que no alcanza a dimensionar el alcance de sus acciones. Adem\'as, la situaci\'on psicol\'ogica de la se\~nora Susana la lleva a obsecionarse con las apariencias que otras personas en general puedan percibir de ella y su familia.
	\end{description}\end{quote}

\section*{General}

\subsubsection*{¿Por qué crees que Carmen cometió el homicidio?}

	\par Creo que Carmen se sintio abrumada por toda la situaci\'on y con su principal preocupaci\'on en la cabeza, proteger a su hijo y a su noeta de cualquier da\~no posible, tomo una decisi\'on apresurada para eliminar aquello que estaba amenazando directemente a su familia. 
	\par Claramente esta no era la mejor manera de actuar, dado que termino matando a otra persona cuando la vida de las personas que trataba proteger a\'un no estaba amenazada; a\'un hab\'ia muchas otras formas de actuar sin tener que llegar a ese extremo. Pero supongo que los constantes ataques de la se\~nora Susana se acumularon hasta desbordar la paciencia y el autocontrol de la se\~nora Carmen.

\subsubsection*{¿qué sucedió con la dignidad humana y la alteridad en este caso?}

	\par Podemos ver la se\~nora Susana tenia una limitada capacidad para actuar con alteridad, y esto sumado a su intespetuosa forma de ser, su inestabilidad emocional, su miedo al abandono y su sencaci\'on de superioridad; la llevo a ser una persona poco soportable y que no sentia remordimiento alguno por su actuar.
	\par Las caracteristicas anteriores de la se\~nora Susana fueron soportadas por un tiempo por la se\~nora Carmen. Pero al final la se\~nora Carmen termino por desbordarse, me imagino que uno de los principales factores para esto es que la se\~nora Susana no respetaba la dignidad humana de la se\~nora Carmen.



%%%%%%%%%%%%%%%%%%%%%%%%%%%%%%%%
%%         Bibliografia        %%
%%%%%%%%%%%%%%%%%%%%%%%%%%%%%%%%%%

\begin{thebibliography}{X}
	\bibitem{youtube} Video Vision. (2017). \textit{Mujeres Asesinas 2 CARMEN HONRADA videovision} [Vídeo]. YouTube. \url{https://www.youtube.com/watch?v=J9GwXPEm-ek} 
\end{thebibliography}

\end{document}