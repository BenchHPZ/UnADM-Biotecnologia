\documentclass[12pt]{article}
\usepackage[spanish]{babel}

%%%%%%%%%%%%%%%%%%%%%%%%%%%%%%%%%%
%%%%%%%%%%%%%%%%%%%%%%%%%%%%%   %%
%%        Datos Trabajo     %%  %%
%%%%%%%%%%%%%%%%%%%%%%%%%%%%%%%%%%
\newcommand{\titulo}[0]{Autorreflexión}
\newcommand{\materia}[0]{Desarrollo Humano}
\newcommand{\grupo}[0]{BI-BDHU-2002-B2-012}
\newcommand{\unidad}[0]{Unidad 2}


%%%%%%%%%%%%%%%%%%%%%%%%%%%%%%%%%%
%%%%%%%%%%%%%%%%%%%%%%%%%%%%%%%%%%
\usepackage{amssymb}
\usepackage{enumerate}
\usepackage{geometry}
\usepackage{mathtools}
\usepackage{multicol}
\usepackage{soul}

\usepackage{graphicx}
	\graphicspath{ {assets/} }

\usepackage{hyperref}
	\hypersetup{
			pdftex,
		        pdfauthor={bench},
		        pdftitle={\titulo},
		        pdfsubject={\materia},
		        pdfkeywords={\grupo, \unidad, UnADM},
		        pdfproducer={Latex with hyperref, Ubuntu},
		        pdfcreator={pdflatex, or other tool},
			colorlinks=true,
				linkcolor=red,
				urlcolor=cyan,
				filecolor=green,
				citecolor=blue}

%%%%%%%%%%%%%%%%%%%%%%%%%%%%%%%%%%
%%%%%%%%%%%%%%%%%%%%%%%%%%%%%%%%%%

\title{
	%\includegraphics{../../../assets/logo-unadm} \\
	\ \\ Benjam\'in Rivera \\
	\bf{\titulo}\\\ \\}

\author{
	{\Huge Universidad Abierta y a Distancia de México} \\
	TSU en Biotecnolog\'ia \\
	\textit{Materia:} \materia \\
	\textit{Grupo:} \grupo \\
	\textit{Unidad:} \unidad \\
	\\
	\textit{Matricula:} ES202105994 }

\date{\textit{Fecha de entrega:} \today}


%%%%%%%%%%%%%%%%%%%%%%%%%%%%%
%%        Documento         %%
%%%%%%%%%%%%%%%%%%%%%%%%%%%%%%%
\begin{document}
\maketitle\newpage

\subsection*{¿Cómo se relacionan los conceptos de dignidad humana y autoconocimiento con la biotecnología?}

\par Tanto la dignidad humana como el autoconocimiento deben ser reglas que gobiernen nuestra forma de actuar como biotecnologos (y creo que todos los profesionistas en general). La primera porque sin dignidad humana todos dejamos de ser animales, de manera que todos los proyectos en los que trabajemos deben de buscar mejorar esta de alguna forma, jamas deben de tratar de afectarla negativamente. Respecto al autoconocimiento debe ser parte de los principios de cualquier profesional, lo que incluye a los biotecnologos; ya que sin esta el trabajo en equipo es muy complicado.

\subsection*{¿Qué conocimientos nuevos adquiriste sobre la ética profesional y cómo podrías aplicarlos en tu vida profesional?}

 La verdad es que no aprendí nada nuevo. A lo mas le puse nombres a varios conceptos que no tenia del todo claros, pero no agregue reglas alguna a mi ética personal. Considero que, idealmente, una buena persona debe ser un buen profesionista; esto lleva a que todos los valore que debes aplicar como persona también deben ser aplicados como profesionista.




%%%%%%%%%%%%%%%%%%%%%%%%%%%%%%%%
%%         Bibliografia        %%
%%%%%%%%%%%%%%%%%%%%%%%%%%%%%%%%%%

\end{document}