\documentclass[12pt]{article}
\usepackage[spanish]{babel}

%%%%%%%%%%%%%%%%%%%%%%%%%%%%%%%%%%
%%%%%%%%%%%%%%%%%%%%%%%%%%%%%   %%
%%        Datos Trabajo     %%  %%
%%%%%%%%%%%%%%%%%%%%%%%%%%%%%%%%%%
\newcommand{\titulo}[0]{Actividad 1. Leyes positivas y negativas.}
\newcommand{\materia}[0]{Desarrollo Humano}
\newcommand{\grupo}[0]{BI-BDHU-2002-B2-012}
\newcommand{\unidad}[0]{Unidad 3}


%%%%%%%%%%%%%%%%%%%%%%%%%%%%%%%%%%
%%%%%%%%%%%%%%%%%%%%%%%%%%%%%%%%%%
\usepackage{amssymb}
\usepackage{enumerate}
\usepackage{geometry}
\usepackage{mathtools}
\usepackage{multicol}
\usepackage{soul}

\usepackage{graphicx}
	\graphicspath{ {assets/} }

\usepackage{hyperref}
	\hypersetup{
			pdftex,
		        pdfauthor={bench},
		        pdftitle={\titulo},
		        pdfsubject={\materia},
		        pdfkeywords={\grupo, \unidad, UnADM},
		        pdfproducer={Latex with hyperref, Ubuntu},
		        pdfcreator={pdflatex, or other tool},
			colorlinks=true,
				linkcolor=red,
				urlcolor=cyan,
				filecolor=green,
				citecolor=blue}

%%%%%%%%%%%%%%%%%%%%%%%%%%%%%%%%%%
%%%%%%%%%%%%%%%%%%%%%%%%%%%%%%%%%%

\title{
	%\includegraphics{../../../assets/logo-unadm} \\
	\ \\ Benjam\'in Rivera \\
	\bf{\titulo}\\\ \\}

\author{
	{\Huge Universidad Abierta y a Distancia de México}\\
	TSU en Biotecnolog\'ia \\
	\textit{Materia:} \materia \\
	\textit{Grupo:} \grupo \\
	\textit{Unidad:} \unidad \\
	\\
	\textit{Matricula:} ES202105994 }

\date{\textit{Fecha de entrega:} \today}


%%%%%%%%%%%%%%%%%%%%%%%%%%%%%
%%        Documento         %%
%%%%%%%%%%%%%%%%%%%%%%%%%%%%%%%
\begin{document}
\maketitle\newpage

 \par Esta situación en general es algo que yo recuerdo seguir bastante de cerca cuando estaba sucediendo. El circo era una actividad que yo disfrutaba mucho pero que ignoraba constantemente la situación en que los animales eran tratados.
 
 \par \
 \par Respecto a las implicaciones de esa situación me parece que debemos observar tres perspectivas principales. La de la comunidad \textit{cirquera}, la de el gobierno, la de la comunidad en general y la de el cuidado de los animales. 
 
 \par Respecto al punto en donde nos centramos en el bienestar de los animales, definitivamente esta ley promueve un avance en la aceptación y consideración de que los animales son seres sensibles que sufren y sienten como nosotros. Por lo que entendí de la ley \cite{ley}, no se prohibia directamente el uso de animales en la existencia cirquera, pero se solicitaban altos estandares para el cuidado de los animales; esto abría la posibilidad de que algunos circos pudieran seguir usando animales en sus espectáculos, pero siempre procurando el bienestar de estos. En general desde este punto de vista no logro encontrar alguna desventaja, ya que lo que se solicitaba era a favor de los animales.
 \par Por otro lado, la comunidad cirquera me parece que sufrió bastante con esta situación. Primero que nada, no todos los grupos cirqueros tenían la capacidad para poder solventar el cuidado que exigían las nuevas regulaciones, esto llevo a que muchos de estos decidieran donar sus bestias al estado. Este cambio radical ocasiono que no todos los grupos pudieran adaptarse al nuevo estilo de circos y, que por lo tanto, muchos de estos desaparecieran. Aquellos que lograron sobrevivir, que el día de hoy sabemos que son muy pocos, hemos visto que cambiaron radicalmente su manera de atraer público, y aunque son pocos en comparación, podemos ver que aquellos que pudieron encontrar nuevas formas de entrener a su publico aún prosperan. 
 \par Sin embargo no todo fue tan bueno, porque el ritmo apresurado que implicaba la aplicación de esta ley creaba distintos problemas de logística tanto para los grupos cirqueros como para el gobierno que supuestamente iba a resguardar a parte de estos animales. Parte de estos problemas, y otros más, además de una propuesta por parte de ciertos grupos ambientalistas están documentados y expresados en \cite{santuario}; en esta artículo se trata la situación real a la que se enfrentaron los distintos actores durante el proceso de aplicación de la ley. Por un lado esta idea si se pudo llevar a cabo  \cite{santuai}, y aunque liberaba ligeramente el congestionamiento de los zoológicos públicos, no es suficiente para todos los animales que fueron desalojados de los circos.
 \par Específicamente desde el punto de vista del gobierno, me parece que fue una decisión bastante apresurada. Esto porque la nueva ley que entraba en vigor contemplaba que los animales deberían de ser donados, de manera gratuita, al estado; pero en ningún momento se destino un mayor presupuesto a los zoológicos, que se volvieron albergues públicos para estas grandes bestias. Como ya mencione en el párrafo anterior, algunos tuvieron la suerte de acabar en lugares como \textit{santuaai}; pero los que no, llegaron a un lugar casi tan inadecuado como del que acababan de salir, en donde se sufrió de sobrepoblación por falta de recursos. Creo que una forma para mejorar esto implicaba una aplicación paulatina de la ley, donde se fueran reasignando recursos y acogiendo animales conforme la situación lo permitiera; de esta manera tanto los circos como los zoológicos habrían tenido la oportunidad de adaptarse paulatinamente a las nuevas situaciones que fueran surgiendo.
 \par Por ultimo esta la perspectiva del público en general. Claramente para la mayoría de las personas esto paso principalmente desapercibido; una pequeña molestia inicial paso a ser sustituida rápidamente por apatía general. Y la mayoría de ellos no fue afectado de alguna manera directa.
 
 \par \
 \par En conclusión, después de hacer el análisis correspondiente a esta actividad, yo pude notar que las leyes deben ser reflexionadas con mucha dedicación para evitar problemas como este. Esto me lleva a pensar que la mayoría de las personas que nos representan no comprenden la complejidad de las decisiones que toman.

 

%%%%%%%%%%%%%%%%%%%%%%%%%%%%%%%%
%%         Bibliografia        %%
%%%%%%%%%%%%%%%%%%%%%%%%%%%%%%%%%%

\newpage
\begin{thebibliography}{X}
	\bibitem[Geoinnova, 2015]{geoinnova} Asociación Geoinnova. (2015, julio 10). México: Entra en vigor la ley que prohíbe a los circos utilizar ejemplares silvestres en sus espectáculos. Territorio Geoinnova - SIG y Medio Ambiente. \url{https://geoinnova.org/blog-territorio/mexico-entra-en-vigor-la-ley-que-prohibe-a-los-circos-utilizar-ejemplares-silvestres-en-sus-espectaculos/}
	\bibitem[mzepeda, 2015]{animalpolitico} mzepeda. (2015, julio 8). Hoy entra en vigor la ley que prohíbe animales salvajes en circos. Animal Político. \url{https://www.animalpolitico.com/2015/07/a-partir-de-manana-entra-en-vigor-la-ley-que-prohibe-animales-salvajes-en-circos/}
	\bibitem[LGVS, 2015]{ley} Ley General de Vida Silvestre. (2015). 72.
	\bibitem[Ortuño, 2015]{santuario} Ortuño, G. (2015, agosto 8). Santuaai, un refugio que pretende rescatar a animales de circo. Animal Político. \url{https://www.animalpolitico.com/2015/08/santuaai-un-refugio-que-pretende-rescatar-mas-de-mil-animales-de-circo/}
	\bibitem[QC, 2019]{santuai} Santuaai, un proyecto chingón que rescatará animales salvajes en Querétaro. (2019, junio 11). \url{https://www.youtube.com/watch?v=-wd92DlTxZk}





\end{thebibliography}

\end{document}