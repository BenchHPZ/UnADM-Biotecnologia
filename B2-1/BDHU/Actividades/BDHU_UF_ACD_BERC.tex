\documentclass[12pt]{article}
\usepackage[spanish]{babel}

%%%%%%%%%%%%%%%%%%%%%%%%%%%%%%%%%%
%%%%%%%%%%%%%%%%%%%%%%%%%%%%%   %%
%%        Datos Trabajo     %%  %%
%%%%%%%%%%%%%%%%%%%%%%%%%%%%%%%%%%
\newcommand{\titulo}[0]{Evidencia de aprendizaje:\\ Proyecto de vida.}
\newcommand{\materia}[0]{Desarrollo humano}
\newcommand{\grupo}[0]{BI-BDHU-2002-B2-012}
\newcommand{\unidad}[0]{Unidad 3}


%%%%%%%%%%%%%%%%%%%%%%%%%%%%%%%%%%
%%%%%%%%%%%%%%%%%%%%%%%%%%%%%%%%%%
\usepackage{amssymb}
\usepackage{enumerate}
\usepackage{geometry}
\usepackage{mathtools}
\usepackage{multicol}
\usepackage{soul}

\usepackage{graphicx}
	\graphicspath{ {assets/} }

\usepackage{hyperref}
	\hypersetup{
			pdftex,
		        pdfauthor={bench},
		        pdftitle={\titulo},
		        pdfsubject={\materia},
		        pdfkeywords={\grupo, \unidad, UnADM},
		        pdfproducer={Latex with hyperref, Ubuntu},
		        pdfcreator={pdflatex, or other tool},
			colorlinks=true,
				linkcolor=red,
				urlcolor=cyan,
				filecolor=yellow}

%%%%%%%%%%%%%%%%%%%%%%%%%%%%%%%%%%
%%%%%%%%%%%%%%%%%%%%%%%%%%%%%%%%%%

\title{
	%\includegraphics{../../../assets/logo-unadm} \\
	{\Huge Universidad Abierta y a Distancia de M\'exico}
	\ \\\ \\\ \\ {\Large Benjam\'in Rivera} \\
	\bf{\titulo}\\\ \\}

\author{
	TSU en Biotecnolog\'ia \\
	\textit{Materia:} \materia \\
	\textit{Grupo:} \grupo \\
	\textit{Unidad:} \unidad \\
	\\
	\textit{Matricula:} ES202105994 }

\date{\textit{Fecha de entrega:} \today}


%%%%%%%%%%%%%%%%%%%%%%%%%%%%%
%%        Documento         %%
%%%%%%%%%%%%%%%%%%%%%%%%%%%%%%%
\begin{document}
\maketitle\newpage

%%%%%%%%%%%%%%%%%%%%%%%%%%%%%%%%%%%
%%           Contenido            %%
%%%%%%%%%%%%%%%%%%%%%%%%%%%%%%%%%%%%%

\section*{Cortometraje}
	\begin{center}
		\url{https://www.youtube.com/watch?v=4INwx_tmTKw}
	\end{center}

	\begin{quote}\begin{description}
		\item [¿Qué valores observaste y cómo fomentan la dignidad humana?] Pudimos observar empatía, solidaridad, tolerancia, alteridad y respeto por parte de \textit{Maria}; todos estos fueron valores que no fueron mostrados desde el inicio por los demás niños de la escuela. Los profesores, y lo que me imagino que son los padres, también mostraron estos valores, aunque claramente sin la inocencia de Maria; al final también mostraron empatía y agradecimiento por la relación que había formado Maria.
		\ \\
		\item [¿En qué acciones de la niña se pudo observar que ella fue empática?] Prácticamente en todas las acciones, ya que lo trato de incluir en todas las actividades y apoyándolo en todas las actividades que, debido a las varias discapacidades motrices y mentales que tenía, el no podría realizar solo; pasando desde ayudándolo a jugar futbol, hasta tratar de entablar conversación a pesar de que no contestaba.
		\ \\
		\item [¿Crees que la niña asumió su libertad favorablemente?] Definitivamente creo que su elección fue favorable tanto para ella, como para el y para sus compañeros. Para ella porque le permitió entender la vida de otra forma al ayudarlo constantemente a el, desafortunadamente no tuvo un \textit{final inmediato} feliz, pero eventualmente la llevo a tratar de proporcionar esa ayuda de manera profesional a todos. Al niño le ayudo el tener alguien que estuviera apoyando sus actividades y que estuviera supliendo las deficiencias motoras que padecía. Y para el resto de sus compañeros, porque les dio un ejemplo de como mostrar valores a cualquier persona sin importar las diferencias.
		
	\end{description}\end{quote}


\section*{Personas} 
	\begin{quote}\begin{description}
		\item [Yalitza Aparicio] Yalitza Aparicio Martínez es una actriz de cine y docente mexicana, principalmente reconocida por su participación en la película Roma. Su constante actuación y abanderamiento en su cultura, tradiciones y costumbres; han mostrado al mundo la manera en que las razas y apariencias son cosas de las que nos hemos preocupado demasiado e innecesariamente. Sus acciones han apoyado a la concientización e importancia de la inclusión. 
		\ \\
		\item [Miguel Álvarez Gándara] Es el presidente y cofundador de Servicios y Asesoría para la Paz, que es un grupo que trata de generar un espacio dedicado al análisis de los movimientos, las luchas y los procesos sociales que acontecen en México. Ha participado en diálogos entre el Gobierno y el Ejército Zapatista de Liberación Nacional \textit{(EZLN)}, así como las pláticas entre el Gobierno y familiares de los normalistas desaparecidos de Ayotzinapa. Sus acciones han tratado de promover la empatía y la alteridad, además del trabajo colaborativo de la sociedad.
		\ \\
		\item [Consuelo Morales Elizondo] Ella es una activista mexicana, fundadora y directora de la organización Ciudadanos en Apoyo a los Derechos Humanos, A.C.. Esta es una organización civil y sin fines de lucro que tiene como causa principal la defensa de los Derechos Humanos de las personas; buscan despertar la conciencia de la sociedad civil para que asuma como causa propia la defensa y promoción de los Derechos Humanos, contribuyendo así a que la dignidad de la persona se respete en nuestra sociedad. De manera directa, ella esta afectando la percepción y concientización del respeto, tolerancia, alteridad y trabajo en equipo.

	\end{description}\end{quote}

 
\section*{Proyecto de vida}
\noindent \textbf{Describe qué valores tendrás que promover para alcanzar tu objetivo y cómo harás para que tu comportamiento se apegue a la ética profesional.} 
	\ \\ 
	\par Es relevante tener todos los valores en mente al actuar, aunque creo que aquellos que decidamos priorizar sobre otros son parte importante de la manera en que nos definimos como personas y, que por lo tanto, también ayuda a definir nuestra personalidad. Yo creo que aquellos que más se deben priorizar son respeto, prudencia, lealtad, colaboración y empatía. 
	\par El \textbf{respeto} porque no podemos ir por el mundo esperando recibir algo que no estamos dispuestos a dar; queda un detalle aquí en donde, sin importar como se comporten las demás personas con nosotros, siempre se debera demostrar respeto por ellos, es algo que no se puede perder. De la \textbf{prudencia} es para siempre tener en cuenta que las decisiones con la cabeza fría son mejores y que, por lo tanto, ninguna decisión debe ser tomada de manera apresurada o con demasiadas emociones en la cabeza. La \textbf{lealtad} porque los humanos nos somos seres que podamos sobrevivir solos, y es importante apoyar y mostrar gratitud con a aquellos que han demostrado ser de confianza y nos han ayudado e algún punto. Por una idea similar debemos tener siempre en cuenta que la \textbf{colaboración} es indispensable. Y la \textbf{empatía} porque debemos tener en cuenta que no todos tuvieron las mismas oportunidades que nosotros y que hay personas más afortunadas que nosotros; y tanto los que son más, como los que son menos, afortunados que nosotros, todos somos prácticamente los mismos seres y debemos apoyarnos.


%%%%%%%%%%%%%%%%%%%%%%%%%%%%%%%%
%%         Bibliografia        %%
%%%%%%%%%%%%%%%%%%%%%%%%%%%%%%%%%%

\begin{thebibliography}{X}
	
	\bibitem{basico1} Universidad Abierta y a Distancia de México. (s/f). Unidad 1. Desarrollo humano y libertad. UnADM.
	\bibitem{basico2} Universidad Abierta y a Distancia de México. (s/f). Unidad 2. El compromiso individual y social del ser humano. UnADM.
	\bibitem{basico3} Universidad Abierta y a Distancia de México. (s/f). Unidad 3. Desarrollo humano y libertad. UnADM.


\end{thebibliography}

\end{document}