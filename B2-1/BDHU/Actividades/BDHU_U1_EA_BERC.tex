\documentclass[12pt]{article}
\usepackage[spanish]{babel}

%%%%%%%%%%%%%%%%%%%%%%%%%%%%%%%%%%
%%%%%%%%%%%%%%%%%%%%%%%%%%%%%   %%
%%        Datos Trabajo     %%  %%
%%%%%%%%%%%%%%%%%%%%%%%%%%%%%%%%%%
\newcommand{\titulo}[0]{Evidencia de aprendizaje:\\Juicio moral}
\newcommand{\materia}[0]{Desarrollo humano}
\newcommand{\grupo}[0]{BI-BDHU-2002-B2-012}
\newcommand{\unidad}[0]{Unidad 1}


%%%%%%%%%%%%%%%%%%%%%%%%%%%%%%%%%%
%%%%%%%%%%%%%%%%%%%%%%%%%%%%%%%%%%
\usepackage{amssymb}
\usepackage{enumerate}
\usepackage{geometry}
\usepackage{mathtools}
\usepackage{multicol}
\usepackage{soul}

\usepackage{graphicx}
	\graphicspath{ {assets/} }

\usepackage{hyperref}
	\hypersetup{
			pdftex,
		        pdfauthor={bench},
		        pdftitle={\titulo},
		        pdfsubject={\materia},
		        pdfkeywords={\grupo, \unidad, UnADM},
		        pdfproducer={Latex with hyperref, Ubuntu},
		        pdfcreator={pdflatex, or other tool},
			colorlinks=true,
				linkcolor=red,
				urlcolor=cyan,
				filecolor=yellow}

%%%%%%%%%%%%%%%%%%%%%%%%%%%%%%%%%%
%%%%%%%%%%%%%%%%%%%%%%%%%%%%%%%%%%

\title{
	%\includegraphics{../../../assets/logo-unadm} \\
	{\Huge Universidad Abierta y a Distancia de M\'exico}
	\ \\\ \\\ \\ {\Large Benjam\'in Rivera} \\
	\bf{\titulo}\\\ \\}

\author{
	TSU en Biotecnolog\'ia \\
	\textit{Materia:} \materia \\
	\textit{Grupo:} \grupo \\
	\textit{Unidad:} \unidad \\
	\\
	\textit{Matricula:} ES202105994 }

\date{\textit{Fecha de entrega:} \today}


%%%%%%%%%%%%%%%%%%%%%%%%%%%%%
%%        Documento         %%
%%%%%%%%%%%%%%%%%%%%%%%%%%%%%%%
\begin{document}
\maketitle\newpage

%%%%%%%%%%%%%%%%%%%%%%%%%%%%%%%%%%%
%%           Contenido            %%
%%%%%%%%%%%%%%%%%%%%%%%%%%%%%%%%%%%%%
\section*{Juicio Moral}
	\paragraph*{Objeto}
		
		\par Esfuerzos de investigación pública e intereses de grandes compañías fueron puestos sobre la investigación, desarrollo y producción de maíz transgénico. El uso de estos productos, en el suelo mexicano, amenaza la existencia de las especies nativas de maíz; además de que existe una desinformación colectiva respecto al uso de este y sus afectaciones en la salud de las personas.

	\paragraph*{Fin}
		
		\par El fin, el ideal que se buscaba, para este proyecto en espec\'ifico; implicaba solucionar un problema actual. Se buscaba una manera para producir alimento que fuera f\'acil y econ\'omico para producir, tanto desde el aspecto econ\'omico como del ecol\'ogico; y que adem\'as, siguiera teniendo las mismas propiedades nutrimentales, o inlcuso mejores.


	\paragraph*{Circunstancias}

		\par El desarrollo y cambio de la sociedad es inminente. El Crecimiento demogr\'afica del mundo ha credo nuevos retos para la sociedad, principalmente cubrir sus necesidades b\'asicas. Una de ellas es cubrir la alimentaci\'on de la pobalci\'on, esta tambi\'en se puede entender como, encontrar maneras m\'as eficientes para alimetnar saludablemente (dentro de lo posible) a toda la poblaci\'on.
		
\subsection*{Conclusiones}
	\par En una primera instancia, yo considero que las intenciones y la implementaci\'on de esta clase de proyectos ten\'ian un buen objetivo. Esto dado que no es un secreto para nadie que la poblaci\'on esta creciendo a un paso muy acelerado, y las necesidades de la poblaci\'on estan creciendo con ella.
	\par Por otro lado, el que los intereses se inmiscuyeran en este proyecto, lo perjudico terriblemente, lo corrompieron. Para que este proyecto hubiera sido un \'exito, se debio haber extendido el periodo de pruebas para asegurarse de que los productos fueran seguros para el consumo humano; tambi\'en se debi\'o haber destinado tiempo y recursos para hacer una campa\~na publcitaria que permitiera explicar a las personas el \textit{porque} de estos, para que todos tuvieran la informaci\'on necesaria para escoger lo que consederaran mejor para ellos mismos.
	\par Al final podemos concluir que la manera en que se está introduciendo en el país no es la mejor que existe, por lo que es malo.

%%%%%%%%%%%%%%%%%%%%%%%%%%%%%%%%
%%         Bibliografia        %%
%%%%%%%%%%%%%%%%%%%%%%%%%%%%%%%%%%

\begin{thebibliography}{X}
	\bibitem{UAC} Mena, L. B. M. (2015, junio). \textit{Maíz transgénico: ¿Beneficio para quién?} scielo. \url{http://www.scielo.org.mx/scielo.php?script=sci_arttext&pid=S0188-45572015000100006}

	\bibitem{UNAM} Oliv\'e, L., Chauvet, M., \& G\'alvez, A. (s. f.). \textit{IMPLICACIONES \'ETICAS Y SOCIALES DE LA BIOTECNOLOG\'IA}. Sociedad Mexicana de Biotecnolog\'ea y Bioingenier\'ia. Recuperado 26 de septiembre de 2020, de \url{https://smbb.mx/congresos%20smbb/puertovallarta03/TRABAJOS/AREA_IV/SIMPOSIO/SIV-3.pdf}
	
	\bibitem{sites} (s. a.) \textit{1.7. Objeto, fin y circunstancias - DH}. (s. f.). Google Sites. Recuperado 26 de septiembre de 2020, de \url{https://sites.google.com/a/unadmexico.mx/desarrollo-humano/1-7-objeto-fin-y-circunstancias}
\end{thebibliography}

\end{document}