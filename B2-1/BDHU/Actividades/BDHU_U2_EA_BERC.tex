\documentclass[12pt]{article}
\usepackage[spanish]{babel}

%%%%%%%%%%%%%%%%%%%%%%%%%%%%%%%%%%
%%%%%%%%%%%%%%%%%%%%%%%%%%%%%   %%
%%        Datos Trabajo     %%  %%
%%%%%%%%%%%%%%%%%%%%%%%%%%%%%%%%%%
\newcommand{\titulo}[0]{Evidencia de aprendizaje: Una mirada alternativa.}
\newcommand{\materia}[0]{Desarrollo Humano}
\newcommand{\grupo}[0]{BI-BDHU-2002-B2-012}
\newcommand{\unidad}[0]{Unidad 2}


%%%%%%%%%%%%%%%%%%%%%%%%%%%%%%%%%%
%%%%%%%%%%%%%%%%%%%%%%%%%%%%%%%%%%
\usepackage{amssymb}
\usepackage{enumerate}
\usepackage{geometry}
\usepackage{mathtools}
\usepackage{multicol}
\usepackage{soul}

\usepackage{graphicx}
	\graphicspath{ {assets/} }

\usepackage{hyperref}
	\hypersetup{
			pdftex,
		        pdfauthor={bench},
		        pdftitle={\titulo},
		        pdfsubject={\materia},
		        pdfkeywords={\grupo, \unidad, UnADM},
		        pdfproducer={Latex with hyperref, Ubuntu},
		        pdfcreator={pdflatex, or other tool},
			colorlinks=true,
				linkcolor=red,
				urlcolor=cyan,
				filecolor=green,
				citecolor=blue}

%%%%%%%%%%%%%%%%%%%%%%%%%%%%%%%%%%
%%%%%%%%%%%%%%%%%%%%%%%%%%%%%%%%%%

\title{
	%\includegraphics{../../../assets/logo-unadm} \\
	\ \\ Benjam\'in Rivera \\
	\bf{\titulo}\\\ \\}

\author{
	{\Huge Universidad Abierta y a Distancia de México}\\
	TSU en Biotecnolog\'ia \\
	\textit{Materia:} \materia \\
	\textit{Grupo:} \grupo \\
	\textit{Unidad:} \unidad \\
	\\
	\textit{Matricula:} ES202105994 }

\date{\textit{Fecha de entrega:} \today}


%%%%%%%%%%%%%%%%%%%%%%%%%%%%%
%%        Documento         %%
%%%%%%%%%%%%%%%%%%%%%%%%%%%%%%%
\begin{document}
\maketitle\newpage

\section{Videos}

\begin{enumerate}
	\item \textbf{ ¿Cuál de los dos personajes mostró una madurez en su inteligencia emocional y cuál tuvo mayor empatía? ¿Por qué la inteligencia emocional genera que haya una mayor alteridad hacia los demás? }
	
	Me parece que el personaje calvo mostró una mayor inteligencia emocional, lo que, en este caso se traduce en tener una mayor empatía. Y creo que la inteligencia emocional esta directamente ligada a que tanto somos capaces de pensar en lo que nos rodea, mientras mayor sea esta somos capaces de percibir mejor nuestro entorno; esto lleva a que aquellos que tienen más desarrollada su inteligencia emocional tengan la capacidad de proyectarse en otros, en otras palabras, tener una mayor alteridad.
	
	\item \textbf{ ¿Qué opinas sobre el autocontrol que tuvo la niña? ¿Consideras que la madre tuvo un manejo y comunicación asertiva ante la situación? }
	
	El autocontrol de la niña claramente puede mejorar, esto lo digo porque no pudo seguir lo que alguien con más experiencia, lo que me imagino que es su madre o quien este a cargo de ella, le pidió que hiciera. Por parte de la madre me parece que puedo haberle explicado a la niña desde un inicio que no sería el fin del ser al que apreciaba; sin embargo, su forma de hacerle notar que no era el fin parecía necesario para que ella lo viviera y así pudiera comprendelo mejor.
	
	\item \textbf{ ¿Consideras que el proceso de la negociación y conciliación es parte de la inteligencia emocional? ¿Cuál par de personajes resolvió de mejor manera la situación y tuvo una mejor comunicación asertiva? }
	
	Si lo considero, creo que es parte relevante. El negociar es un acto de dos personas, donde se requiere tener conocimiento de lo que la otra parte de la negociación quiere, o necesita en cierto momento, para encontrar un punto intermedio donde ambas partes de la negociación estén conformes con lo que obtengan. Creo que la segunda, dado que fueron los que lograron cruzar exitosamente el puente a pesar de que no había suficiente espacio para ambos; a pesar de esto creo que su manera de quitar a los animales más grandes no fue la mejor.
	
	\item \textbf{ ¿Cómo fue la autoestima del personaje principal al momento de notar que no tenía su cabello? ¿Cuál fue el mensaje sobre la autoestima? }
	
	Suponemos que el personaje principal es el humano, con esto continuamos el análisis. Cuando el humano perdió su cabello, parecía que no se sentía el mismo sin él, por lo que parece que su autoestima bajo de manera considerable por algunos momentos; luego, cunado su cabello regreso y se volvió a ir, pero el de la mujer también, se dio cuenta de que su cabello no lo definía. Creo que este cortometraje nos ayuda a darnos cuenta de que nuestra autoestima, y nuestro valor, nunca debe estar sustentada en nuestra apariencia física.
	
	\item \textbf{ ¿Qué acciones causaron que el personaje tuviera una baja autoestima? ¿Quién sacó de la depresión y mejoró la imagen del personaje? ¿Qué representan las cadenas? }
	
	Creo que la mala autoestima estuvo causada por presiones sociales que fueron ocasionadas por redes sociales, además de que puede ser qe esto haya sido acentuado por que estuvo sola en su cumpleaños. Ella misma con ayuda del personaje con pelo rizado le ayudó a salir de la depresión al mostrarle que ella podía estar presente. Para mi, las cadenas representaron las limitaciones que el personaje determinaba pera si misma, causadas por prestarle demasiada atención a lo que los demás decían de ella.
	
\end{enumerate}



\section{Situación}

	\par Jacinto está trabajando en su proyecto terminal bajo la dirección del Dr. Sabiondo en el laboratorio del centro de ciencias genómicas de la UNAM, en Morelos. Cuando Jacinto estaba estudiando haciendo el chequeo mensual del estado del arte de su proyecto, noto que los registros del proyecto en el CB del IBt en el que estaba colaborando no coincidían con lo que estaba trabajando. Cuando se lo comento al Dr este le dijo, de una manera irrespetuosa, que para no perder los fondos de la iniciativa privada y los de la UNAM, había tenido que registrar un proyecto menos polémico en la UNAM que fuera acreditado más rápido; además le dijo que si quería terminar su trabajo a tiempo no mencionara eso con alguien más. Como Jacinto está por terminar su proyecto, decide no llevar esto más lejos y seguir trabajando como hasta ahora lo estaba haciendo.

\subsection{Análisis del reglamento}

	\par Sabemos que el código \cite{codigo unam} se basa en 4 valores, trataremos de revisar superficialmente cada uno de los puntos para ver si la situación cumple, o no, con este.
	\begin{quote}
		\begin{description}
			\item [Rigor y honestidad] El Dr no cumplió con la parte de no \textit{ engañar } y, dependiendo de como este recibiendo el resto de su financiamiento publico y privado, puede ser que este incurriendo en alguna \textit{ practica corrupta }.

			\item [Respeto] El Dr no trató con respeto a Jacinto a pesar de que este se estaba conduciendo a él con respeto y con un tema de interés.

			\item [Responsabilidad] Dado el procedimiento dudoso que siguió el Dr para el registro de su proyecto puede ser que el proyecto que realmente este haciendo no este aprobado por el Comité de Bioética.

			\item [Integridad] El Dr mostró un abuso de poder al amenazar a Jacinto para que no fuera honesto.
			
		\end{description}
	\end{quote}

	\par De manera que, como varios de los puntos del código no están siendo cumplidos, esta situación no cumple con el mismo.



%%%%%%%%%%%%%%%%%%%%%%%%%%%%%%%%
%%         Bibliografia        %%
%%%%%%%%%%%%%%%%%%%%%%%%%%%%%%%%%%

\begin{thebibliography}{X}
	\bibitem{bib 1} \textit{“Bridge” by Ting Chian Tey $|$ Disney Favorite}. (2013, agosto 26). \url{https://www.youtube.com/watch?v=_X_AfRk9F9w}

	\bibitem{codigo unam} Consejo Interno del IBt. (2019). \textit{ Código de Ética e Integridad Científica del Instituto de Biotecnología de la UNAM}. UNAM. \url{http://www.ibt.unam.mx/computo/pdfs/Codigo-de-Etica-e-Integridad-Cientifica-del-IBt-de-la-UNAM.pdf}
	
	\bibitem{bib 3} \textit{Cortometrajes animados / EL NÁUFRAGO}. (2014, septiembre 28). \url{ https://www.youtube.com/watch?v=GQRw4sSnwr0}

	\bibitem{bib 4} \textit{El mejor amigo del hombre}. (2016, agosto 29). \url{https://www.youtube.com/watch?v=Ti0ANyxl6_c}

	\bibitem{bib 5} \textit{Overcomer Animated Short $|$ Hannah Grace}. (2016, abril 25). \url{https://www.youtube.com/watch?v=V6ui161NyTg}

\end{thebibliography}

\end{document}