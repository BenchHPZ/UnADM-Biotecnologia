\documentclass[12pt]{article}
\usepackage[spanish]{babel}

%%%%%%%%%%%%%%%%%%%%%%%%%%%%%%%%%%
%%%%%%%%%%%%%%%%%%%%%%%%%%%%%   %%
%%        Datos Trabajo     %%  %%
%%%%%%%%%%%%%%%%%%%%%%%%%%%%%%%%%%
\newcommand{\titulo}[0]{Evidencia de aprendizaje: Una mirada alternativa.}
\newcommand{\materia}[0]{Desarrollo Humano}
\newcommand{\grupo}[0]{BI-BDHU-2002-B2-012}
\newcommand{\unidad}[0]{Unidad 2}


%%%%%%%%%%%%%%%%%%%%%%%%%%%%%%%%%%
%%%%%%%%%%%%%%%%%%%%%%%%%%%%%%%%%%
\usepackage{amssymb}
\usepackage{enumerate}
\usepackage{geometry}
\usepackage{mathtools}
\usepackage{multicol}
\usepackage{soul}

\usepackage{graphicx}
	\graphicspath{ {assets/} }

\usepackage{hyperref}
	\hypersetup{
			pdftex,
		        pdfauthor={bench},
		        pdftitle={\titulo},
		        pdfsubject={\materia},
		        pdfkeywords={\grupo, \unidad, UnADM},
		        pdfproducer={Latex with hyperref, Ubuntu},
		        pdfcreator={pdflatex, or other tool},
			colorlinks=true,
				linkcolor=red,
				urlcolor=cyan,
				filecolor=green,
				citecolor=blue}

%%%%%%%%%%%%%%%%%%%%%%%%%%%%%%%%%%
%%%%%%%%%%%%%%%%%%%%%%%%%%%%%%%%%%

\title{
	%\includegraphics{../../../assets/logo-unadm} \\
	\ \\ Benjam\'in Rivera \\
	\bf{\titulo}\\\ \\}

\author{
	{\Huge Universidad Abierta y a Distancia de México}\\
	TSU en Biotecnolog\'ia \\
	\textit{Materia:} \materia \\
	\textit{Grupo:} \grupo \\
	\textit{Unidad:} \unidad \\
	\\
	\textit{Matricula:} ES202105994 }

\date{\textit{Fecha de entrega:} \today}


%%%%%%%%%%%%%%%%%%%%%%%%%%%%%
%%        Documento         %%
%%%%%%%%%%%%%%%%%%%%%%%%%%%%%%%
\begin{document}
\maketitle\newpage

\section{Videos}

\begin{enumerate}
	\item \textbf{ ¿Cuál de los dos personajes mostró una madurez en su inteligencia emocional y cuál tuvo mayor empatía? ¿Por qué la inteligencia emocional genera que haya una mayor alteridad hacia los demás? }
	
	Me parece que el personaje calvo mostró una mayor inteligencia emocional, lo que en este caso se mostró como más empatía. Y creo que la inteligencia emocional esta directamente ligada a que tanto somos capaces de pensar en lo que nos rodea, mientras mayor sea esta somos capaces de percibir mejor nuestro entorno; esto lleva a que aquellos que tienen más desarrollada su inteligencia emocional tengan la capacidad de proyectarse en otros, en otras palabras, tener una mayor alteridad.
	
	\item \textbf{ ¿Qué opinas sobre el autocontrol que tuvo la niña? ¿Consideras que la madre tuvo un manejo y comunicación asertiva ante la situación? }
	
	El autocontrol de la niña claramente puede mejorar, no pudo seguir lo que alguien con más experiencia le pidió que hiciera. Por parte de la madre me parece que puedo haberle explicado a la niña desde un inicio que no sería el fin del ser al que apreciaba; sin embargo, su forma de hacerle notar que no era el fin parecía necesario para que ella lo viviera y así pudiera comprendelo mejor.
	
	\item \textbf{ ¿Consideras que el proceso de la negociación y conciliación es parte de la inteligencia emocional? ¿Cuál par de personajes resolvió de mejor manera la situación y tuvo una mejor comunicación asertiva? }
	
	Si lo considero, el negociar implica comprender lo que el otro quiere, o necesita, para encontrar un punto intermedio donde ambos estén conformes con lo que obtengan. Claro que la segunda, dado que fueron quienes lograron cruzar exitosamente el puente a pesar de que no había suficiente espacio para ambos.
	
	\item \textbf{ ¿Cómo fue la autoestima del personaje principal al momento de notar que no tenía su cabello? ¿Cuál fue el mensaje sobre la autoestima? }
	
	Suponemos que el personaje principal es el humano. El humano considero no era quien quería ser sin su cabello, por lo que su autoestima fue muy bajo por unos momentos, luego se recupero algo. Que nuestra autoestima nunca debe estar sustentada en nuestra apariencia física.
	
	\item \textbf{ ¿Qué acciones causaron que el personaje tuviera una baja autoestima? ¿Quién sacó de la depresión y mejoró la imagen del personaje? ¿Qué representan las cadenas? }
	
	Creo que la mala autoestima estuvo causada por presiones sociales ocasionadas por redes sociales, además de que estuvo sola en su cumpleaños. Ella misma con ayuda del personaje con perlo rizado le ayudo a salir de la depresión al mostrarle que ella podía estar presente. Para mi, las cadenas representaron las limitaciones que el personaje determinaba pera si misma, causadas por prestarle demasiada atención a lo que los demás decían de ella.
\end{enumerate}



\section{Situaci\'on}

Jacinto esta trabajando en su proyecto terminal bajo la dirección del Dr. Sabiondo en el laboratorio del centro de ciencias genómicas de la UNAM, en Morelos. Cuando Jacinto estaba estudiando haciendo el chequeo mensual del estado del arte de su proyecto, noto que los registros del proyecto en el CB del IBt en el que estaba colaborando no coincidían con lo que estaba trabajando. Cuando se lo comento al Dr este le dijo que para no perder los fondos de la iniciativa privada y los de la UNAM, había tenido que registrar un proyecto menos polémico en la UNAM que fuera acreditado más rápido. Como Jacinto esta por terminar su proyecto, decide no llevar esto más lejos y seguir trabajando como hasta ahora lo estaba haciendo.




%%%%%%%%%%%%%%%%%%%%%%%%%%%%%%%%
%%         Bibliografia        %%
%%%%%%%%%%%%%%%%%%%%%%%%%%%%%%%%%%
\newpage
\begin{thebibliography}{X}
	\bibitem{bib 1} \textit{“Bridge” by Ting Chian Tey $|$ Disney Favorite}. (2013, agosto 26). \url{https://www.youtube.com/watch?v=_X_AfRk9F9w}

	\bibitem{bib 2} Consejo Interno del IBt. (2019). \textit{ Código de Ética e Integridad Científica del Instituto de Biotecnología de la UNAM}. UNAM. \url{http://www.ibt.unam.mx/computo/pdfs/Codigo-de-Etica-e-Integridad-Cientifica-del-IBt-de-la-UNAM.pdf}
	
	\bibitem{bib 3} \textit{Cortometrajes animados / EL NÁUFRAGO}. (2014, septiembre 28). \url{ https://www.youtube.com/watch?v=GQRw4sSnwr0}

	\bibitem{bib 4} \textit{El mejor amigo del hombre}. (2016, agosto 29). \url{https://www.youtube.com/watch?v=Ti0ANyxl6_c}

	\bibitem{bib 5} \textit{Overcomer Animated Short $|$ Hannah Grace}. (2016, abril 25). \url{https://www.youtube.com/watch?v=V6ui161NyTg}

\end{thebibliography}

\end{document}