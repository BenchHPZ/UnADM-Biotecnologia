\documentclass[12pt]{article}
\usepackage[spanish]{babel}

%%%%%%%%%%%%%%%%%%%%%%%%%%%%%%%%%%
%%%%%%%%%%%%%%%%%%%%%%%%%%%%%   %%
%%        Datos Trabajo     %%  %%
%%%%%%%%%%%%%%%%%%%%%%%%%%%%%%%%%%
\newcommand{\titulo}[0]{Evidencia de aprendizaje:\\ Proyecto de vida.}
\newcommand{\materia}[0]{Desarrollo humano}
\newcommand{\grupo}[0]{BI-BDHU-2002-B2-012}
\newcommand{\unidad}[0]{Unidad 3}


%%%%%%%%%%%%%%%%%%%%%%%%%%%%%%%%%%
%%%%%%%%%%%%%%%%%%%%%%%%%%%%%%%%%%
\usepackage{amssymb}
\usepackage{enumerate}
\usepackage{geometry}
\usepackage{mathtools}
\usepackage{multicol}
\usepackage{soul}

\usepackage{graphicx}
	\graphicspath{ {assets/} }

\usepackage{hyperref}
	\hypersetup{
			pdftex,
		        pdfauthor={bench},
		        pdftitle={\titulo},
		        pdfsubject={\materia},
		        pdfkeywords={\grupo, \unidad, UnADM},
		        pdfproducer={Latex with hyperref, Ubuntu},
		        pdfcreator={pdflatex, or other tool},
			colorlinks=true,
				linkcolor=red,
				urlcolor=cyan,
				filecolor=yellow}

%%%%%%%%%%%%%%%%%%%%%%%%%%%%%%%%%%
%%%%%%%%%%%%%%%%%%%%%%%%%%%%%%%%%%

\title{
	%\includegraphics{../../../assets/logo-unadm} \\
	{\Huge Universidad Abierta y a Distancia de M\'exico}
	\ \\\ \\\ \\ {\Large Benjam\'in Rivera} \\
	\bf{\titulo}\\\ \\}

\author{
	TSU en Biotecnolog\'ia \\
	\textit{Materia:} \materia \\
	\textit{Grupo:} \grupo \\
	\textit{Unidad:} \unidad \\
	\\
	\textit{Matricula:} ES202105994 }

\date{\textit{Fecha de entrega:} \today}


%%%%%%%%%%%%%%%%%%%%%%%%%%%%%
%%        Documento         %%
%%%%%%%%%%%%%%%%%%%%%%%%%%%%%%%
\begin{document}
\maketitle\newpage

%%%%%%%%%%%%%%%%%%%%%%%%%%%%%%%%%%%
%%           Contenido            %%
%%%%%%%%%%%%%%%%%%%%%%%%%%%%%%%%%%%%%
\section*{Preguntas}

\noindent \textbf{Chequeo personal}. Reflexiona sobre tu situación actual de vida, quién eres, descríbete en diferentes ámbitos de las dimensiones humanas, y agrega tus fortalezas y debilidades personales.
\par \
	\par Físicamente soy una persona alta, gruesa y con cachetes muy grandes; como fortaleza pondría mi condición física y como debilidad que ya me duelen algunas articulaciones. Desde el aspecto mental tengo como principal fortaleza que puedo recordar cosas fácilmente, aunque como debilidad esta que me distraigo con bastante facilidad. Desde el aspecto emocional me cuesta bastante socializar con las personas, aunque creo que puedo leerlas con facilidad. Y desde el aspecto espiritual por el momento no tengo nada definido, recientemente tuve una crisis de fe de la que apenas estoy saliendo y no he definido en lo que espiritualmente creo y confío.


\par \ \\
\par 
\noindent \textbf{Bases y fundamentos}. Reflexiona sobre las decisiones tomadas, la manera en la que tomas las decisiones, y revisa lo que te ha influenciado a llegar hasta este punto e ingresar a la universidad.
\par \
	\par Creo que hay dos clases de decisiones, las inmediatas y las esperadas. Las decisiones inmediatas son aquellas que no esperabas, que aparecen en el día a día; para estas decisiones trato de tener protocolos de decisiones generales para poder reaccionar lo más rápido posible pero manteniéndome dentro de limites manejables que luego pueda pasar a una decisión esperada; en otras palabras, una decisión de emergencia para poder obtener más tiempo y pensarlo mejor. Por otro lado, las decisiones esperadas son aquellas que, con al menos 12 horas de anticipación, ya sabes que tienes que tomar, en otras palabras, son aquellas que tienes suficiente tiempo para pensar y definir de una manera argumentativa; estas decisiones trato de tomarlas de la manera más informada y calculadora posible.


\par \ \\
\par 
\noindent \textbf{Anhelos y sueños}. Describe cuáles son tus anhelos y sueños universitarios, profesionales y laborales, indica hacia dónde quieres ir, tus propósitos y en dónde te gustaría estar.
\par \
	\par A nivel personal anhelo poder mejorar mi capacidad para socializar con las personas, tanto en mis nuevas relaciones, como en las relaciones con las personas a las que ya conozco y parezco importarles. 
	\par A nivel académico, espero poder terminar mi licenciatura y empezar con un posgrado inmediatamente para tratar de optimizar mi tiempo, del cual ya perdí bastante en años pasados por cosas que ahora me doy cuenta que son irrelevantes. 
	\par A nivel profesional, ya empece a tener algunas experiencias, espero poder aumentar el nivel y la importancia que mis aportaciones profesionales hacen a los grupos afectados. Después de acabar el posgrado, aún no estoy muy seguro de lo que quiero hacer; por un lado esta la opción de trabajar en una empresa ya consolidada, idea que me desagrada en lo absoluto y ya tengo varias opciones a la vista, y por le otro crear mi propio legado al emprender mi idea y tener mi propia empresa (incluyendo la opción de ser un \textit{freelancer}).


\par \ \\
\par 
\par \textbf{Recursos}. Describe con qué recursos cuentas para alcanzar los anhelos y
sueños. Los recursos no únicamente son materiales.
\par \
	\par Mi principal recurso soy yo, claro que hasta después de que termine de reconstruirme; después de eso estaré listo para empezar a trabajar sobre mis anhelos y sueños. Ya con esta mejora, si es que aún sigue ahí, uno de mis principales apoyos (como ha sido hasta ahora) sera mi novia, quien ha estado ahí para mi en muchos momentos y parece que no se quiere ir (para mi suerte). También quiero crear una nueva herramienta, un grupo más grande de apoyo para no recargar todo en ella. Y por ultimo (aunque no menos importante) mi mascota, quien se que estará por un tiempo más que suficiente para salir de todo esto.
	\par Creo que con estas herramientas debería de ser suficientes para poder trabajar mis anhelos y sueños.





\section*{Trello (Plan de vida)}

\begin{center}
	\url{https://trello.com/b/gOQDKfSS}
\end{center}


%%%%%%%%%%%%%%%%%%%%%%%%%%%%%%%%
%%         Bibliografia        %%
%%%%%%%%%%%%%%%%%%%%%%%%%%%%%%%%%%

\begin{thebibliography}{X}
	
	\bibitem{github} bench. (s/f). BenchHPZ/UnADM-Biotecnologia. GitHub. Recuperado el \today, de \url{https://github.com/BenchHPZ/UnADM-Biotecnologia}

\end{thebibliography}

\end{document}