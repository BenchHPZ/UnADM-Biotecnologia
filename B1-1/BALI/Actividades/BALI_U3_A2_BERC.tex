\documentclass[12pt]{article}
\usepackage[spanish]{babel}

%%%%%%%%%%%%%%%%%%%%%%%%%%%%%%%%%%
%%%%%%%%%%%%%%%%%%%%%%%%%%%%%   %%
%%        Datos Trabajo     %%  %%
%%%%%%%%%%%%%%%%%%%%%%%%%%%%%%%%%%
\newcommand{\titulo}[0]	{Actividad 2: Regla de Cramer}
\newcommand{\materia}[0]{Álgebra Lineal}
\newcommand{\grupo}[0]	{BI-BALI-2002-B1-012}
\newcommand{\unidad}[0]	{Unidad 3}


%%%%%%%%%%%%%%%%%%%%%%%%%%%%%%%%%%
%%%%%%%%%%%%%%%%%%%%%%%%%%%%%%%%%%
\usepackage{amssymb}
\usepackage{enumerate}
\usepackage[margin=1.5cm]{geometry}
\usepackage{mathtools}
\usepackage{multicol}
\usepackage{soul}

\usepackage{graphicx}
	\graphicspath{ {assets/} }

\usepackage{hyperref}
	\hypersetup{
			pdftex,
		        pdfauthor={bench},
		        pdftitle={\titulo},
		        pdfsubject={\materia},
		        pdfkeywords={\grupo, \unidad, UnADM},
		        pdfproducer={Latex with hyperref, Ubuntu},
		        pdfcreator={pdflatex, or other tool},
			colorlinks=true,
				linkcolor=red,
				urlcolor=cyan,
				filecolor=yellow}

%%%%%%%%%%%%%%%%%%%%%%%%%%%%%%%%%%
%%%%%%%%%%%%%%%%%%%%%%%%%%%%%%%%%%

\title{
	\includegraphics{../../../assets/logo-unadm} \\
	\ \\ Benjam\'in Rivera \\
	\bf{\titulo}\\\ \\}

\author{
	Universidad Abierta y a Distancia de México \\
	TSU en Biotecnolog\'ia \\
	\textit{Materia:} \materia \\
	\textit{Grupo:} \grupo \\
	\textit{Unidad:} \unidad \\
	\\
	\textit{Matricula:} ES202105994 }

\date{\textit{Fecha de entrega:} \today}


%%%%%%%%%%%%%%%%%%%%%%%%%%%%%
%%        Documento         %%
%%%%%%%%%%%%%%%%%%%%%%%%%%%%%%%
\usepackage{amsmath}
\begin{document}
\maketitle\newpage

\subsection*{Calcula los determinantes de las siguientes matrices por el Método de Cramer}

	\par Dado que las instruccinoes son un poco confusas (porque el metodo de Cramer sirve para obtener las soluciones de un sisitema y nos estan pidiendo la determinante de las matrices) y como no nos dan un vector $X$ para resolver las ecuaciones, entonces, \'unicamente obtendremos las determinantes por el m\'etodo de \textit{menores y cofactores}.

\begin{multicols}{2}
	\begin{enumerate}[A.]
		\item Sea
		
			$$A = \begin{pmatrix}
				5 & 4 & 3\\
				7 & 8 & 6\\
				6 & 3 & 1\\
			\end{pmatrix}$$
			
		,decidimos utilizar el m\'etodo por columnas, y tomamos $j=1$. Siguiendo lo anterior sabemos que tendremos que trabajar con las submatrices $a_1, a_2$ y $a_3$, las cuales quedan definidas por
		\begin{eqnarray*}
			a_3 &=&	\begin{pmatrix}
						4 & 3\\
						8 & 6\\
					\end{pmatrix} \\
			a_2 &=&	\begin{pmatrix}
						4 & 3\\
						3 & 1\\
					\end{pmatrix} \\
			a_1 &=&	\begin{pmatrix}
						8 & 6\\
						3 & 1\\
					\end{pmatrix} 
		\end{eqnarray*}
		de eso seguimos y aplicando nuestra formula obtenemos que
		\begin{eqnarray*}
			det(A) &=& \sum_{i=1}^m a_{ij} \cdot C_{ij} \\
				&=&	 5 \left|\begin{array}{rr}
							8 & 6 \\
							3 & 1
						\end{array}\right| 
						-7 \left|\begin{array}{rr}
							4 & 3 \\
							3 & 1
						\end{array}\right| 
						+6 \left|\begin{array}{rr}
							4 & 3 \\
							8 & 6
						\end{array}\right|   \\
				&=&	 5(-10) -7(-5) +6(0) = -15 
		\end{eqnarray*}	
	
	\hrule\ \\\hrule
		
		\item Sea
		
			$$B = \left(\begin{array}{rrr}
				-3 & -5 & -3 \\
				4 & 5 & 6 \\
				7 & 8 & 3
				\end{array}\right) 
			$$
			
		,decidimos utilizar el m\'etodo por renglones, y tomamos $i=2$. Siguiendo lo anterior sabemos que tendremos que trabajar con las submatrices $b_1, b_2$ y $b_3$, las cuales quedan definidas por
		\begin{eqnarray*}
			b_1 &=&	\begin{pmatrix}
						5 & 6\\
						8 & 3\\
					\end{pmatrix} \\
			b_2 &=&	\begin{pmatrix}
						4 & 6\\
						7 & 3\\
					\end{pmatrix} \\
			b_3 &=&	\begin{pmatrix}
						4 & 5\\
						7 & 8\\
					\end{pmatrix} 
		\end{eqnarray*}
		de eso seguimos y aplicando nuestra formula obtenemos que
		\begin{eqnarray*}
			det(B) &=& \sum_{j=1}^m b_{ij} \cdot C_{ij} \\
				&=&	-3 \left|\begin{array}{rr}
							5 & 6 \\
							8 & 3
						\end{array}\right| 
					+5 \left|\begin{array}{rr}
							4 & 6 \\
							7 & 3
						\end{array}\right| 
					-3 \left|\begin{array}{rr}
							4 & 5 \\
							7 & 8
						\end{array}\right|   \\
				&=&	  3(33) -5(30) +3(3) = -42
		\end{eqnarray*}		

		\hrule\ \\\hrule
		
	\end{enumerate}
\end{multicols}



\subsection*{Calcula los determinantes de los siguientes sistemas de ecuaciones}

	\par Para este ejercicio primero calcularemos la determinante del sistema, y de cada una de las variables, con el m\'etodo de menores y coproductos. Despu\'es usaremos el m\'etodo de cramer para obtener las soluciones del sistema.
	
\begin{multicols}{2}
	\begin{enumerate}[A.]
		\item Sea el sistema de ecuaciones
			\begin{eqnarray*}
				x-y+z&=&7 \\
				4x+2y-2z&=&10 \\
				2x+3y+z&=&23
			\end{eqnarray*}
		obtenemos la matriz extendida
			$$\begin{pmatrix}
				1  & -1 &  1 &  7\\
				4  &  2 & -2 & 10\\
				2  &  3 &  1 & 23\\
			\end{pmatrix} $$
		para la cual debemos obtener los determinantes de
			$$\delta = \begin{pmatrix}
				1  & -1 &  1 \\
				4  &  2 & -2 \\
				2  &  3 &  1 
			\end{pmatrix} $$
			$$\delta_x = \begin{pmatrix}
				 7  & -1 &  1 \\
				10  &  2 & -2 \\
				23  &  3 &  1 \\
			\end{pmatrix} $$
			$$\delta_y = \begin{pmatrix}
				1  &  7 &  1 \\
				4  & 10 & -2 \\
				2  & 23 &  1 \\
			\end{pmatrix} $$
			$$\delta_z = \begin{pmatrix}
				-1 &  1 &  7\\
				 2 & -2 & 10\\
				 3 &  1 & 23\\
			\end{pmatrix} $$
		para encontrar los determinates trabajaremos sobre columnas bloqueando la fila 1, por lo que
		\begin{eqnarray*}
			\Delta_x &=& \det\left(\begin{array}{rrr}
							7 & -1 & 1 \\
							10 & 2 & -2 \\
							23 & 3 & 1
						\end{array}\right) \\
				&=& 7 \left|\begin{array}{rr}
						2 & -2 \\
						3 & 1
					\end{array}\right| 
					+1 \left|\begin{array}{rr}
						10 & -2 \\
						23 & 1
					\end{array}\right| 
					+1 \left|\begin{array}{rr}
						10 & 2 \\
						23 & 3
					\end{array}\right| \\
				&=&  7(8) +1(56) +1(-16) = 96
		\end{eqnarray*}
		
		\begin{eqnarray*}
			\Delta_y &=& \det\left(\begin{array}{rrr}
						1 & 7 & 1 \\
						4 & 10 & -2 \\
						2 & 23 & 1
						\end{array}\right)  \\
				&=& 1 \left|\begin{array}{rr}
						10 & -2 \\
						23 & 1
					\end{array}\right| 
					-7 \left|\begin{array}{rr}
						4 & -2 \\
						2 & 1
					\end{array}\right|
					 +1 \left|\begin{array}{rr}
						4 & 10 \\
						2 & 23
					\end{array}\right| \\
				&=&   1(56) -7(8) +1(72) = 72
		\end{eqnarray*}
		
		\begin{eqnarray*}
			\Delta_z &=& \det\left(\begin{array}{rrr}
						1 & -1 & 7 \\
						4 & 2 & 10 \\
						2 & 3 & 23
						\end{array}\right)  \\
				&=& 1 \left|\begin{array}{rr}
						2 & 10 \\
						3 & 23
					\end{array}\right| 
					+1 \left|\begin{array}{rr}
						4 & 10 \\
						2 & 23
					\end{array}\right| 
					+7 \left|\begin{array}{rr}
						4 & 2 \\
						2 & 3
					\end{array}\right| \\
				&=&  1(16) +1(72) +7(8) = 144
		\end{eqnarray*}
		
		\begin{eqnarray*}
			\Delta &=& \det\left(\begin{array}{rrr}
						1 & -1 & 1 \\
						4 & 2 & -2 \\
						2 & 3 & 1
						\end{array}\right) \\
				&=& 1 \left|\begin{array}{rr}
						2 & -2 \\
						3 & 1
					\end{array}\right|
					+1 \left|\begin{array}{rr}
						4 & -2 \\
						2 & 1
					\end{array}\right| 
					+1 \left|\begin{array}{rr}
						4 & 2 \\
						2 & 3
					\end{array}\right|  \\
				&=&  1(8) +1(8) +1(8) = 24
		\end{eqnarray*}
		
		Y con estas determinantes calculadas podemos utilizar el m\'etodo de Cramer, por lo que
		
		\begin{eqnarray*}
			x &=& \frac{\Delta_x}{\Delta} = \frac{96}{24} = 4 \\
			y &=& \frac{\Delta_y}{\Delta} = \frac{72}{24} = 3 \\
			z &=& \frac{\Delta_z}{\Delta} = \frac{144}{24} = 6
		\end{eqnarray*}
		
		\hrule\ \\\hrule
		
		\item Sea el sistema de ecuaciones
			\begin{eqnarray*}
				2x + 3y + z&=&  4  \\
				4x+ 2y - 2z&=&  10 \\
				x - 3y - 3z&=&  3
			\end{eqnarray*}
		obtenemos la matriz extendida
			$$\begin{pmatrix}
				2  &  3 &  1 &  4\\
				4  &  2 & -2 & 10\\
				1  & -3 & -3 &  3\\
			\end{pmatrix} $$
		para la cual debemos obtener el determinante de
			$$\delta = \begin{pmatrix}
				2  &  3 &  1 \\
				4  &  2 & -2 \\
				1  & -3 & -3 
			\end{pmatrix} $$
			$$\delta_x = \begin{pmatrix}
				 4  &  3 &  1 \\
				10 &  2 & -2  \\
				 3 & -3 & -3  \\
			\end{pmatrix} $$
			$$\delta_y = \begin{pmatrix}
				2  &  4 &  1  \\
				4  & 10 & -2  \\
				1  &  3 & -3  \\
			\end{pmatrix} $$
			$$\delta_z = \begin{pmatrix}
				 3 &  1 &  4 \\
				 2 & -2 & 10 \\
				-3 & -3 &  3 \\
			\end{pmatrix} $$
		para encontrar los determinates, al igual que con el ejercicio anterior, trabajaremos sobre las columnas, bloqueando la fila 1, de donde obtenemos que
		\begin{eqnarray*}
			\Delta_x &=& \det \left(\begin{array}{rrr}
						4 & 3 & 1 \\
						10 & 2 & -2 \\
						3 & -3 & -3
						\end{array}\right)  \\
				&=& 4 \left|\begin{array}{rr}
						2 & -2 \\
						-3 & -3
					\end{array}\right| 
					-3 \left|\begin{array}{rr}
						10 & -2 \\
						3 & -3
					\end{array}\right| 
					+1 \left|\begin{array}{rr}
						10 & 2 \\
						3 & -3
					\end{array}\right| \\
				&=&  4(-12) -3(-24) +1(-36) = -12
		\end{eqnarray*}
		
		\begin{eqnarray*}
			\Delta_y &=& \det \left(\begin{array}{rrr}
						2 & 4 & 1 \\
						4 & 10 & -2 \\
						1 & 3 & -3
						\end{array}\right)   \\
				&=& 2 \left|\begin{array}{rr}
						10 & -2 \\
						3 & -3
					\end{array}\right| 
					-4 \left|\begin{array}{rr}
						4 & -2 \\
						1 & -3
					\end{array}\right| 
					+1 \left|\begin{array}{rr}
						4 & 10 \\
						1 & 3
					\end{array}\right| \\
				&=&   2(-24) -4(-10) +1(2) = -6
		\end{eqnarray*}
		
		\begin{eqnarray*}
			\Delta_z &=& \det\left(\begin{array}{rrr}
						2 & 3 & 4 \\
						4 & 2 & 10 \\
						1 & -3 & 3
						\end{array}\right)   \\
				&=& 2 \left|\begin{array}{rr}
						2 & 10 \\
						-3 & 3
					\end{array}\right| 
					-3 \left|\begin{array}{rr}
						4 & 10 \\
						1 & 3
					\end{array}\right| 
					+4 \left|\begin{array}{rr}
						4 & 2 \\
						1 & -3
					\end{array}\right| \\
				&=&  2(36) -3(2) +4(-14) = 10
		\end{eqnarray*}
		
		\begin{eqnarray*}
			\Delta &=& \det \left(\begin{array}{rrr}
						2 & 3 & 1 \\
						4 & 2 & -2 \\
						1 & -3 & -3
						\end{array}\right) \\
				&=& 2 \left(\begin{array}{rr}
						2 & -2 \\
						-3 & -3
					\end{array}\right| 
					-3 \left|\begin{array}{rr}
						4 & -2 \\
						1 & -3
					\end{array}\right| 
					+1 \left|\begin{array}{rr}
						4 & 2 \\
						1 & -3
					\end{array}\right|  \\
				&=&  2(-12) -3(-10) +1(-14) = -8
		\end{eqnarray*}
		
		Y con estas determinantes calculadas podemos utilizar el m\'etodo de Cramer, por lo que
		
		\begin{eqnarray*}
			x &=& \frac{\Delta_x}{\Delta} = \frac{-12}{-8} = \frac{3}{2} \\
			y &=& \frac{\Delta_y}{\Delta} = \frac{-6}{-8} = \frac{3}{4} \\
			z &=& \frac{\Delta_z}{\Delta} = \frac{10}{-8} = -\frac{5}{4}
		\end{eqnarray*}
		
		\hrule\ \\\hrule

	
	\end{enumerate}
\end{multicols}


%%%%%%%%%%%%%%%%%%%%%%%%%%%%%%%%
%%         Bibliografia        %%
%%%%%%%%%%%%%%%%%%%%%%%%%%%%%%%%%%
\hrule \ \\ \hrule
\begin{thebibliography}{X}

	\bibitem{basica} UnADM. (2020). \textit{U3 $|$ Determinantes}. \today, de División de Ciencias de la Salud, Biológicas y Ambientales Sitio web: \url{https://dmd.unadmexico.mx/contenidos/DCSBA/BLOQUE1/BI/01/BALI/unidad_03/descargables/BALI_U3_Contenido.pdf}
	
	\bibitem{github} BenchHPZ. (2020). \textit{Unidad 3}. \today, de GitHub Sitio web: \url{https://github.com/BenchHPZ/UnADM-Biotecnologia/tree/master/B1-1/BALI/Actividades/BALI_U3_BERC.ipynb}
\end{thebibliography}

\end{document}