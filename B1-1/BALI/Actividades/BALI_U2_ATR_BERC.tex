\documentclass[12pt]{article}
\usepackage[spanish]{babel}

\usepackage{amssymb}
\usepackage{enumerate}
\usepackage{geometry}
	\geometry{margin = 2cm}
\usepackage{mathtools}
\usepackage{multicol}
\usepackage{soul}

\usepackage{graphicx}
	\graphicspath{ {assets/} }


%%%%%%%%%%%%%%%%%%%%%%%%%%%%%%%%%%
%%%%%%%%%%%%%%%%%%%%%%%%%%%%%   %%
%%        Datos Trabajo     %%  %%
%%%%%%%%%%%%%%%%%%%%%%%%%%%%%%%%%%
\newcommand{\titulo}[0]{Autorreflexión. Unidad 1 y 2}
\newcommand{\materia}[0]{\'Algebra Lineal}
\newcommand{\grupo}[0]{BI-BALI-2002-B1-012}
\newcommand{\unidad}[0]{Unidad 2}
%%%%%%%%%%%%%%%%%%%%%%%%%%%%%%%%%%
%%%%%%%%%%%%%%%%%%%%%%%%%%%%%%%%%%
\usepackage[pdftex,
            pdfauthor={bench},
            pdftitle={\titulo},
            pdfsubject={\materia},
            pdfkeywords={\grupo, \unidad, UnADM},
            pdfproducer={Latex with hyperref, or other system},
            pdfcreator={pdflatex, or other tool}]{hyperref}
%%%%%%%%%%%%%%%%%%%%%%%%%%%%%%%%%%
%%%%%%%%%%%%%%%%%%%%%%%%%%%%%%%%%%

\title{
	\includegraphics{../../../assets/logo-unadm} \\
	\ \\ Benjam\'in Rivera \\
	\bf{\titulo}\\\ \\}

\author{
	Universidad Abierta y a Distancia de México \\
	TSU en Biotecnolog\'ia \\
	\textit{Materia:} \materia \\
	\textit{Grupo:} \grupo \\
	\textit{Unidad:} \unidad \\
	\\
	\textit{Matricula:} ES202105994 }

\date{\textit{Fecha de entrega:} \today}


%%%%%%%%%%%%%%%%%%%%%%%%%%%%%
%%        Documento         %%
%%%%%%%%%%%%%%%%%%%%%%%%%%%%%%%
\begin{document}
\maketitle

\noindent\makebox[\linewidth]{\rule{\paperwidth}{0.4pt}}
	\ \\ \
	\par Responde lo que se te pide

\begin{enumerate}[\bf{Unidad} 1]
	\item \ \\
		\begin{enumerate}
			\item \textbf{¿Qué situación de la vida cotidiana puede describirse o explicarse por medio de vectores? Descríbela}. Siendo imaginativos, practicamente cualquier cosa que se mueva o cambie con el tiempo puede ser descrita mediante vectores. Un ejemplo de lo anterior es el movimiento y las trayectorias de los peatones en una intersección de transito (como un semaforo o un cruce), en este ejemplo se podría dibujar dos vectores principales para cada persona, uno que describa su tryectoría confirmada desde que empezo a ser detectado, y otro que trate de hacer una predicción respecto a la direción y velocidad que planea tomar a contnuación.
			
			\item \textbf{¿Por qué crees que es importante conocer el uso y las operaciones de los vectores en la Biotecnología?} En biotecnología deberemos de enfrentarnos en algún momento con \textit{Química Aanalítica}, la cual se puede servir de los vectores para describir ciertas propiedades. Además de eso, como biotecnólogo, existe la posibilidad de trabajar con un grupo de investigación enfocado en epidemiología, quienes sin duda alguna utilizan vectores constantemente.
			
			\item \textbf{¿Qué aprendí en esta unidad?} Aprendimos a manejar los vectores, sus propiedades y operaciones permitidas, también trabajamos con matrices y sistemas de ecuaciones, además de que se empezarón a introducir métodos de solución de sistemas de ecuaciones.
			
			\item \textbf{¿Cumplí con los propósitos de aprendizaje?} Considero que mi desempeño en la unidad 1 fué bastante bueno, cumplí con todas mias actividades en tiempo y forma.
		\end{enumerate}
	
	\item \ \\
		\begin{enumerate}
			\item \textbf{¿Consideras que las actividades realizadas para esta unidad te ayudan en tu carrera?} Definitivamente; a pesar de que no formará parte de mis actividades diarias, no dudo que en mi carrera profesional me encuentre en situaciones donde deba resolver sistemas de ecuaciones, y aunque no creo resolverlas a mano siempre (gracias a las tecnologías que ahora tenemos a nuestro alcance), el saber que es lo que estoy haciendo me permitra desarrollarme mejor como profesionista.
			
			\item \textbf{¿Cuáles son las dificultades que te encontraste al realizar las actividades? ¿Por qué?} Estos son ejercicios bastante tediosos, ya que una vez que logras entender el tema, los ejercicios son bastante sencillos, pero rquieren de muchos pasos ordenados y que afectan a los siguientes, lo que no da lugar a errores. Creo que esto se me hace tedioso ahora porque ya no realizó tantos ejericicios como antes.
			
		\end{enumerate}
\end{enumerate}

%%%%%%%%%%%%%%%%%%%%%%%%%%%%%%%%
%%         Bibliografia        %%
%%%%%%%%%%%%%%%%%%%%%%%%%%%%%%%%%%
\noindent\makebox[\linewidth]{\rule{\paperwidth}{0.4pt}}
\begin{thebibliography}{X}
	\bibitem{biblio} UnADM. (S/D). \textit{Primer semestre Algebra Lineal}. 13 de Julio de 2020, de Universidad Abierta y a Distancia de México | DCSBA Sitio web: \url{https://dmd.unadmexico.mx/contenidos/DCSBA/BLOQUE1/BI/01/BALI/unidad_01/descargables/BALI_U1_Contenido.pdf}
	\bibitem{github} BenchHPZ. (2020). \textit{Biotecnologia}. 24 de julio de 2020, de GitHub Sitio web: \url{https://github.com/BenchHPZ/UnADM-Biotecnologia}
\end{thebibliography}

\end{document}