\documentclass[12pt]{article}
\usepackage[spanish]{babel}

%%%%%%%%%%%%%%%%%%%%%%%%%%%%%%%%%%
%%%%%%%%%%%%%%%%%%%%%%%%%%%%%   %%
%%        Datos Trabajo     %%  %%
%%%%%%%%%%%%%%%%%%%%%%%%%%%%%%%%%%
\newcommand{\titulo}[0]{Asignacion a Cargo del Docente}
\newcommand{\materia}[0]{Algebra Lineal}
\newcommand{\grupo}[0]{GrupBI-BALI-2002-B1-012o}
\newcommand{\unidad}[0]{Final}


%%%%%%%%%%%%%%%%%%%%%%%%%%%%%%%%%%
%%%%%%%%%%%%%%%%%%%%%%%%%%%%%%%%%%
\usepackage{amssymb}
\usepackage{enumerate}
\usepackage{geometry}
\usepackage{mathtools}
\usepackage{multicol}
\usepackage{soul}

\usepackage{graphicx}
	\graphicspath{ {assets/} }

\usepackage{hyperref}
	\hypersetup{
			pdftex,
		        pdfauthor={bench},
		        pdftitle={\titulo},
		        pdfsubject={\materia},
		        pdfkeywords={\grupo, \unidad, UnADM},
		        pdfproducer={Latex with hyperref, Ubuntu},
		        pdfcreator={pdflatex, or other tool},
			colorlinks=true,
				linkcolor=red,
				urlcolor=cyan,
				filecolor=yellow}

%%%%%%%%%%%%%%%%%%%%%%%%%%%%%%%%%%
%%%%%%%%%%%%%%%%%%%%%%%%%%%%%%%%%%

\title{
	\includegraphics{../../../assets/logo-unadm} \\
	\ \\ Benjam\'in Rivera \\
	\bf{\titulo}\\\ \\}

\author{
	Universidad Abierta y a Distancia de México \\
	TSU en Biotecnolog\'ia \\
	\textit{Materia:} \materia \\
	\textit{Grupo:} \grupo \\
	\textit{Unidad:} \unidad \\
	\\
	\textit{Matricula:} ES202105994 }

\date{\textit{Fecha de entrega:} \today}


%%%%%%%%%%%%%%%%%%%%%%%%%%%%%
%%        Documento         %%
%%%%%%%%%%%%%%%%%%%%%%%%%%%%%%%
\begin{document}
\maketitle\newpage

\section*{Ejercicio 1}
	\par Pbten el determinante de la matriz $A$ por el \textit{m\'etodo de Cramer}
		$$A = \begin{pmatrix}
		    3 & 2 & 1 \\
		    4 &-2 & 2 \\
		    2 & 3 & 1
		\end{pmatrix}$$
    
    \paragraph{Soluci\'on} Primero elaboramos la matriz extendida correspondiente
		$$\left(\begin{array}{rrr}
			3 & 2 & 1 \\
			4 & -2 & 2 \\
			2 & 3 & 1 \\
			3 & 2 & 1 \\
			4 & -2 & 2
		\end{array}\right)$$
	de esta obtenemos las operaciones correspondientes, que son
		$$(3)(-2)(1) + (4)(3)(1) + (2)(2)(2)-((2)(-2)(1) + (3)(3)(2) + (4)(2)(1))$$
	y con esto obtenemos que el \textbf{determinante de la matriz $A$} es $\det = -8$



\section*{Ejercicio 2}
	\par Suma las siguientes matrices
		$$ A = \begin{pmatrix} 
		    3 & 4 & 1 \\ 5 & 3 & 1 \\ 2 & 3 & 1 
		    \end{pmatrix}, 
		B = \begin{pmatrix}
		    -2 & -3 & -1 \\ 4 & 3 & 2 \\ 10 & 4 & 2
		    \end{pmatrix} $$
		    
	\paragraph{Soluci\'on} Dado que las matrices son de las mismas dimensiones, sabemos que la operaci\'on suma esta definida correctamente y se realiza \textit{elemento a elemento}. De manera que
	\begin{eqnarray*}
		A + B &=& \begin{pmatrix} 
		    	3 & 4 & 1 \\ 5 & 3 & 1 \\ 2 & 3 & 1 
		    \end{pmatrix} +
		    \begin{pmatrix}
		   		-2 & -3 & -1 \\ 4 & 3 & 2 \\ 10 & 4 & 2
		    \end{pmatrix} \\
		    &=& \left(\begin{array}{rrr}
				1 & 1 & 0 \\
				9 & 6 & 3 \\
				12 & 7 & 3
			\end{array}\right)
	\end{eqnarray*}


\section*{Ejercicio 3}
	\par Resolver el siguiente sistema de ecuaciones mediante el \textbf{m\'etodo de Gauss}

    \begin{eqnarray*}
        3x + 2y + 2z &=& 10 \\
        2x + 3y +  z &=&  8 \\
        3x +  y + 5z &=& 11
    \end{eqnarray*}
    
    \paragraph{Soluci\'on} Primero debemos pasar el sistema de ecuaciones a formato matricial. Este queda
    $$\left(\begin{array}{rrrr}
		3 & 2 & 2 & 10 \\
		0 & \frac{5}{3} & -\frac{1}{3} & \frac{4}{3} \\
		0 & 0 & \frac{14}{5} & \frac{9}{5}
	\end{array}\right)$$
	Luego tenemos que modificar la matriz para obtener una \textit{matriz triangular invertida}, para esto hacemos que la fila 2 sea -2/3 veces la fila 1 mas la fila 2 y que la fila 3 sea -1 veces la fila 1 mas la fila 3
	$$\left(\begin{array}{rrrr}
		3 & 2 & 2 & 10 \\
		0 & \frac{5}{3} & -\frac{1}{3} & \frac{4}{3} \\
		0 & -1 & 3 & 1
	\end{array}\right)$$
	Despu\'es  hacemos que la fila 3 sea 3/5 veces la fila 2 mas la fila 3
	$$\left(\begin{array}{rrrr}
		3 & 2 & 2 & 10 \\
		0 & \frac{5}{3} & -\frac{1}{3} & \frac{4}{3} \\
		0 & 0 & \frac{14}{5} & \frac{9}{5}
	\end{array}\right)$$
	Y ya tenemos la matriz en la forma deseada. Con esta ya podemos ver que las soluciones para este sistema son
	$$ x = \frac{16}{7}, \quad y = \frac{13}{14}, \quad z = \frac{9}{14}$$


\section*{Pruebas}
	\par Todos los ejercicios fueron corroborados en \cite{github}

%%%%%%%%%%%%%%%%%%%%%%%%%%%%%%%%
%%         Bibliografia        %%
%%%%%%%%%%%%%%%%%%%%%%%%%%%%%%%%%%
\newpage
\begin{thebibliography}{X}
	\bibitem{github} BenchHPZ. (2020, 3 septiembre). \textit{BALI\_Z\_BERC}. GitHub. \url{https://github.com/BenchHPZ/UnADM-Biotecnologia/blob/master/B1-1/BALI/Actividades/BALI_Z_BERC.ipynb}
\end{thebibliography}

\end{document}