\documentclass[12pt]{article}
\usepackage[spanish]{babel}

%%%%%%%%%%%%%%%%%%%%%%%%%%%%%%%%%%
%%%%%%%%%%%%%%%%%%%%%%%%%%%%%   %%
%%        Datos Trabajo     %%  %%
%%%%%%%%%%%%%%%%%%%%%%%%%%%%%%%%%%
\newcommand{\titulo}[0]	{
Actividad 3. Problemas sobre Determinantes}
\newcommand{\materia}[0]{Álgebra Lineal}
\newcommand{\grupo}[0]	{BI-BALI-2002-B1-012}
\newcommand{\unidad}[0]	{Unidad 3}


%%%%%%%%%%%%%%%%%%%%%%%%%%%%%%%%%%
%%%%%%%%%%%%%%%%%%%%%%%%%%%%%%%%%%
\usepackage{amssymb}
\usepackage{enumerate}
\usepackage[margin=2.5cm]{geometry}
\usepackage{mathtools}
\usepackage{multicol}
\usepackage{soul}

\usepackage{graphicx}
	\graphicspath{ {assets/} }

\usepackage{hyperref}
	\hypersetup{
			pdftex,
		        pdfauthor={bench},
		        pdftitle={\titulo},
		        pdfsubject={\materia},
		        pdfkeywords={\grupo, \unidad, UnADM},
		        pdfproducer={Latex with hyperref, Ubuntu},
		        pdfcreator={pdflatex, or other tool},
			colorlinks=true,
				linkcolor=red,
				urlcolor=cyan,
				filecolor=yellow}

%%%%%%%%%%%%%%%%%%%%%%%%%%%%%%%%%%
%%%%%%%%%%%%%%%%%%%%%%%%%%%%%%%%%%

\title{
	\includegraphics{../../../assets/logo-unadm} \\
	\ \\ Benjam\'in Rivera \\
	\bf{\titulo}\\\ \\}

\author{
	Universidad Abierta y a Distancia de México \\
	TSU en Biotecnolog\'ia \\
	\textit{Materia:} \materia \\
	\textit{Grupo:} \grupo \\
	\textit{Unidad:} \unidad \\
	\\
	\textit{Matricula:} ES202105994 }

\date{\textit{Fecha de entrega:} \today}


%%%%%%%%%%%%%%%%%%%%%%%%%%%%%
%%        Documento         %%
%%%%%%%%%%%%%%%%%%%%%%%%%%%%%%%

\usepackage{amsmath}
\newcommand{\dotsd}[0]	{\reflectbox{$\ddots$}}


\begin{document}
\maketitle\newpage

\subsection*{Obtener los determinantes de las siguientes matrices por los métodos que se indican:}

	\begin{quote}\begin{enumerate}[1.]
		\item Regla de Sarrus.
		\item Cofactores a primer rengl\'on.	\label{lst: cof ren 1}
		\item Cofactores a segunda columna.	\label{lst: cof col 2}
		\item Menores a primer rengl\'on.		\label{lst: men ren 1}
		\item Menores a segunda columna.		\label{lst: men col 2}
	\end{enumerate}\end{quote}

	\par Dado que en \cite[p\'ag. 10]{basica} se indica que los cofactores se utilizan en el m\'etodo de menores, entonces consideraremos que (\ref{lst: cof ren 1}, \ref{lst: men ren 1}) y (\ref{lst: cof col 2}, \ref{lst: men col 2}) son los mismos ejercicios. De manera que en este ejercicio usaremos en total tres metodos sobre cada matr\'iz para obtener el determinante, los cuales seran
	
	\begin{quote}\begin{enumerate}[1.]
		\item Regla de Sarrus.	
		\item Menores y cofactores sobre primer rengl\'on.	
		\item menores y cofactores sobre segunda columna.		
	\end{enumerate}\end{quote}
	
	\par El m\'etodo de cofactores ya lo hemos utilizado en otras tareas, por lo que seguiremos los pasos ah\'i descritos\footnote{En todo caso se puede consultar en \cite[p\'ag. 10 y 11]{basica}}. Para Sarrus, usaremos $M_n$ para denotar el renglon $n$ de la matriz $M$, entonces seguiremos los siguientes pasos:
	
	\begin{quote}\begin{enumerate}
		\item Generar la matriz 
			$$M' = 
				\begin{pmatrix}
					\ddots&\dots&\dotsd\\
					\dots & M& \dots	\\
					\dotsd&\dots&\ddots\\
					\dots & M_1 & \dots\\
					\dots & M_2 & \dots
				\end{pmatrix}$$
		\item A partir de $M'$, se cumple que
		\begin{eqnarray*}	
			\det{M} &=& {M'}_{1,1} {M'}_{2,2} {M'}_{3,3} + 
						{M'}_{2,1} {M'}_{3,2} {M'}_{4,3} +
						{M'}_{3,1} {M'}_{4,2} {M'}_{5,3} - \\ & &(
						{M'}_{5,1} {M'}_{4,2} {M'}_{3,3} +
						{M'}_{4,1} {M'}_{3,2} {M'}_{2,3} +
						{M'}_{3,1} {M'}_{2,2} {M'}_{1,3} )
		\end{eqnarray*}
	\end{enumerate}\end{quote}
	
	En esta actividad \'unicamente se corrobor\'o el valor de las determinantes, de cada matriz, en 
	\cite[Actividad 3]{notebook}.

\newpage
\begin{multicols}{2}
	\begin{enumerate}[A.]
		\item Sea
			$$A = \begin{pmatrix}
				 2 &  4 &  1\\
				 3 &  2 &  0\\
				 7 &  1 &  3\\
			\end{pmatrix}$$
		Para \textbf{Sarru} tenemos que
			$$A' = \begin{pmatrix}
				 2 &  4 &  1\\
				 3 &  2 &  0\\
				 7 &  1 &  3\\
				 2 &  4 &  1\\
				 3 &  2 &  0\\
			\end{pmatrix}$$
		
		por lo que
			\begin{eqnarray*}	
				\det{A} &=& (2)(2)(3) + (3)(1)(1) + (7)(4)(0) - \\ & &(  
							(3)(4)(3) + (2)(1)(0) + (7)(2)(1) ) \\
						&=& 12+3+0 - (36+0+14) = -35
			\end{eqnarray*}
			
		Para \textbf{renglon 1} usaremos los siguientes menores:
			$$\begin{vmatrix}
				 2 &  0 \\
				 1 &  3 \\
			\end{vmatrix},
			\begin{vmatrix}
				 3 &  0 \\
				 7 &  3 \\
			\end{vmatrix},
			\begin{vmatrix}
				 3 &  2 \\
				 7 &  1 \\
			\end{vmatrix}$$
		para usarlos con los cofactores y
			\begin{eqnarray*}
				\det(A) &=&
					2 \begin{vmatrix}
						 2 &  0 \\
						 1 &  3 \\
					\end{vmatrix} -
					4 \begin{vmatrix}
						 3 &  0 \\
						 7 &  3 \\
					\end{vmatrix} +
					1 \begin{vmatrix}
						 3 &  2 \\
						 7 &  1 \\
					\end{vmatrix} \\
					&=& 2(6) - 4(9) - 11 = -35
			\end{eqnarray*}
			
		Y para \textbf{columna 2} usaremos los menores
			$$\begin{vmatrix}
				 3 &  0 \\
				 7 &  3 \\
			\end{vmatrix},
			\begin{vmatrix}
				 2 &  1 \\
				 7 &  3 \\
			\end{vmatrix},
			\begin{vmatrix}
				 4 &  3 \\
				 1 &  0 \\
			\end{vmatrix}$$
		sobre los cofactores
			\begin{eqnarray*}
				\det(A) &=&
					4 \begin{vmatrix}
						 3 &  0 \\
						 7 &  3 \\
					\end{vmatrix} -
					2 \begin{vmatrix}
						 2 &  1 \\
						 7 &  3 \\
					\end{vmatrix} +
					1 \begin{vmatrix}
						 4 &  3 \\
						 1 &  0 \\
					\end{vmatrix} \\
					&=& - 4(9) + 2(-1) - (-3) = -35
			\end{eqnarray*}
		
		
		\item Sea B
			$$B = \begin{pmatrix}
				-1 &  2 &  3\\
				 2 & -1 &  3\\
				 3 & -1 &  2\\
			\end{pmatrix}$$
			
		Para \textbf{Sarrus} tenemos que 
			$$B = \begin{pmatrix}
				-1 &  2 &  3\\
				 2 & -1 &  3\\
				 3 & -1 &  2\\
				-1 &  2 &  3\\
				 2 & -1 &  3\\
			\end{pmatrix}$$
			
		por lo que 
			\begin{eqnarray*}	
				\det{B} &=& (-1)(-1)(2) + (2)(-1)(3) + (3)(2)(3) - \\ & &(  
							(2)(2)(2) + (-1)(-1)(3) + (2)(-1)(3) ) \\
						&=& 12
			\end{eqnarray*}
			
		Para \textbf{renglon 1} usaremos los siguientes menores:
			$$\begin{vmatrix}
				 -1 &  3 \\
				 -1 &  2 \\
			\end{vmatrix},
			\begin{vmatrix}
				 2 &  3 \\
				 3 &  2 \\
			\end{vmatrix},
			\begin{vmatrix}
				 2 & -1 \\
				 3 & -1 \\
			\end{vmatrix}$$
		para usarlos con los cofactores y
			\begin{eqnarray*}
				\det(B) &=&
					-1 \begin{vmatrix}
						 -1 &  3 \\
						 -1 &  2 \\
					\end{vmatrix} -
					2 \begin{vmatrix}
						 2 &  3 \\
						 3 &  2 \\
					\end{vmatrix} +
					3 \begin{vmatrix}
						 2 & -1 \\
						 3 & -1 \\
					\end{vmatrix} \\
					&=& -1(1) - 2(-5) + 3 = 12
			\end{eqnarray*}
			
		Y para \textbf{columna 2} usaremos los menores
			$$\begin{vmatrix}
				 2 &  3 \\
				 3 &  2 \\
			\end{vmatrix},
			\begin{vmatrix}
				-1 &  3 \\
				 3 &  2 \\
			\end{vmatrix},
			\begin{vmatrix}
				-1 &  3 \\
				 2 &  3 \\
			\end{vmatrix}$$
		sobre los cofactores
			\begin{eqnarray*}
				\det(B) &=&
					- 2 \begin{vmatrix}
						 2 &  3 \\
						 3 &  2 \\
					\end{vmatrix} -
					1 \begin{vmatrix}
						-1 &  3 \\
						 3 &  2 \\
					\end{vmatrix} +
					1 \begin{vmatrix}
						-1 &  3 \\
						 2 &  3 \\
					\end{vmatrix} \\
					&=& - 2(-5) - 1(-11) + 1(-9) = 12
			\end{eqnarray*}
		
	\end{enumerate}
\end{multicols}



%%%%%%%%%%%%%%%%%%%%%%%%%%%%%%%%
%%         Bibliografia        %%
%%%%%%%%%%%%%%%%%%%%%%%%%%%%%%%%%%
\newpage
\begin{thebibliography}{X}

	\bibitem{basica} UnADM. (2020). \textit{U3 $|$ Determinantes}. \today, de División de Ciencias de la Salud, Biológicas y Ambientales Sitio web: \url{https://dmd.unadmexico.mx/contenidos/DCSBA/BLOQUE1/BI/01/BALI/unidad_03/descargables/BALI_U3_Contenido.pdf}
	
	\bibitem{github}BenchHPZ. (2020). \textit{Biotecnolog\'ia}. \today, de GitHub Sitio web: \url{https://github.com/BenchHPZ/UnADM-Biotecnologia/tree/master/B1-1/BALI}
	
	\bibitem{notebook} BenchHPZ. (2020). \textit{Unidad 3}. \today, de GitHub Sitio web: \url{https://github.com/BenchHPZ/UnADM-Biotecnologia/tree/master/B1-1/BALI/Actividades/BALI_U3_BERC.ipynb}
\end{thebibliography}

\end{document}