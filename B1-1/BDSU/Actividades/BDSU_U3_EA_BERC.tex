\documentclass[12pt]{article}
\usepackage[spanish]{babel}

%%%%%%%%%%%%%%%%%%%%%%%%%%%%%%%%%%
%%%%%%%%%%%%%%%%%%%%%%%%%%%%%   %%
%%        Datos Trabajo     %%  %%
%%%%%%%%%%%%%%%%%%%%%%%%%%%%%%%%%%
\newcommand{\titulo}[0]{Evidencia de aprendizaje: Problemáticas de su entorno}
\newcommand{\materia}[0]{Desarrollo Sustentable}
\newcommand{\grupo}[0]{ES202105994}
\newcommand{\unidad}[0]{Unidad 3}


%%%%%%%%%%%%%%%%%%%%%%%%%%%%%%%%%%
%%%%%%%%%%%%%%%%%%%%%%%%%%%%%%%%%%
\usepackage{amssymb}
\usepackage{enumerate}
\usepackage{geometry}
\usepackage{mathtools}
\usepackage{multicol}
\usepackage{soul}

\usepackage{graphicx}
	\graphicspath{ {assets/} }

\usepackage{hyperref}
	\hypersetup{
			pdftex,
		        pdfauthor={bench},
		        pdftitle={\titulo},
		        pdfsubject={\materia},
		        pdfkeywords={\grupo, \unidad, UnADM},
		        pdfproducer={Latex with hyperref, Ubuntu},
		        pdfcreator={pdflatex, or other tool},
			colorlinks=true,
				linkcolor=red,
				urlcolor=cyan,
				filecolor=yellow}

%%%%%%%%%%%%%%%%%%%%%%%%%%%%%%%%%%
%%%%%%%%%%%%%%%%%%%%%%%%%%%%%%%%%%

\title{
	\includegraphics{../../../assets/logo-unadm} \\
	\ \\ Benjam\'in Rivera \\
	\bf{\titulo}\\\ \\}

\author{
	Universidad Abierta y a Distancia de México \\
	TSU en Biotecnolog\'ia \\
	\textit{Materia:} \materia \\
	\textit{Grupo:} \grupo \\
	\textit{Unidad:} \unidad \\
	\\
	\textit{Matricula:} ES202105994 }

\date{\textit{Fecha de entrega:} \today}


%%%%%%%%%%%%%%%%%%%%%%%%%%%%%
%%        Documento         %%
%%%%%%%%%%%%%%%%%%%%%%%%%%%%%%%
\begin{document}
\maketitle\newpage

\subsection*{Problem\'atica anterior}

	\par En \cite{unidad1} describ\'i la el problema causado por las ladrilleras en mi comunidad, aunque por desgracia no es el \'unico ecosistema donde esto sucede.
	\par Este problema es causado porque los artesanos, que actualmente se dedican a la fabricación, y posterior comercio, de los ladrillos. Ellos, por desgracia, no han podido recibir todo el apoyo necesario, ni la educación, para que comprendan y realizar la inversión necesaria para cambiar su fuente de energía actual por una que tenca un efecto menor al ecosistema y la población afectada.



\subsection*{Dimensi\'on de Desarrollo Sustentable}
	
	\par En esta secci\'on tratare de ampliar mi entendimiento de esta porblem\'atica desde la \textbf{dimensi\'on ambiental}. Esto dado que fue lo que m\'as se me complico en la secci\'on anterior.
	\par Esta \textit{dimensi\'on} promueve la protección de los recursos naturales necesarios para la seguridad alimentaria y energética y, al mismo tiempo, comprende el requerimiento de la expansión de la producción para satisfacer a las poblaciones en crecimiento demográfico.
	\par Aunque, a pesar de que nos centraremos en esta, todas las dimensiones estan fuertemente relacinoadas, por lo que no sera la \'unica que se mencione.



\subsection*{Indicadores y soluciones}

	\par Algunos de los indicadores que encuentro relevantes para esta problem\'atica son
	
	\begin{quote}\begin{description}
	
		\item [Calidad del aire] Es el indicador que se encarga de medir el bienestar del aire; para esto utiliza variables como la concentraci\'on de $SO_2$, $NO_2$ y $TSP$
		\item [Uso de la tierra] Este indicador es el ecargadp de medir la calidad del uso que se le esta dando a la tierra; esto se basa en el \textit{porcentaje de tierra con bajo impacto antropog\'enico}
		\item [Ciencia y tecnolog\'ia] Este indicador verifica el estado del arte de la t\'ecnias exitentes y la teor\'ia aceptada alrededor de una proble\'atica; este indicador esta directamente relacionado con los \textit{indices de innovaci\'on y logros tecnol\'ogicos}, dentro de estos esta la \textit{eficiencia energ\'etica} de la tecnolog\'ia actual.
		
	\end{description}\end{quote}
	claramente cada uno de estos indicadores tiene medidas negativas en la problem\'atica que ecogimos en [unidad1], y sobre la cual se centra este trabajo.
	\par A continuiaci\'on se presentaran una opci\'on por indicador para tratar de mejorar la medida que actualmente existe de \'el

	\begin{quote}\begin{description}

		\item [Uso de la tierra] Como mencione en \cite{actividad3}, una de las solucinoes menos complejas ser\'ia elaborar un estudio donde se verificara las zonas geogr\'aficas que podr\'ian causar un menor impacto a su entorno con las t\'ecnicas actuales, y las que puedan venir. Y como tambi\'en se menciona ah\'i, si \'unicamente se trata de solucionar este indicador, \'unicamente ser\'ia una soluci\'on temporal; aunque esto no quiere decir que no se importante para una buena soluci\'on integtral.

		\item [Ciencia y tecnolog\'ia] Este es uno de los indicadores que no pertenece directamente a la \textit{dimensi\'on ambiental}, es m\'as cercano a las \textit{dimensiones sociales y econ\'omicas}, sin embargo, es relevante para poder solucionar los inconvenientes de la \textit{dimensi\'on ambiental}. Para poder mejorar la medida de este indicador es necesario impulsar el desarrollo de las ciencias y tecnolog\'ias que puedan atacar estas problem\'aticas; ejemplos de quienes podr\'ian ayudar son las \textit{ingiener\'ias ambientales y ecol\'ogicas}, como la \textit{bioteconolog\'ia}, o la \textit{Qu\'imica}, que es una ciencia exacta.

		\item [Calidad del aire] Este indicador es un poco m\'as complicado de atacar, ya que debe ser una soluci\'on integral, y aunque la consecuencia inmediata sera en la \textit{dimensi\'on ambiental}, require que su soluci\'on utilice soluciones de las \textit{tres dimensiones} necesarias para un desarrollo sustentable.
		\par Para poder mejorar la \textbf{calidad del aire} es necesario encontrar una localizaci\'on geogr\'afica con un flujo constante de aire, y que dicho flujo no se dirija a zonas sensibles. Adem\'as se necesita cambiar la manera de coser los ladrillos; usando t\'ecnicas, y por lo tanto combustibles, que liberen una menor cantidad de gases nocivos para el ecosistema. Y para que sea posible cambiar las t\'ecnicas utilizadas son necesarias dos cosas, innovar y mejorar las actuales y promover apoyos econ\'omicos, profesionales y culturales para que los actuales artesanos puedan y se interesen por cambiar sus t\'ecnicas actuales. 

	\end{description}\end{quote}
	
	\par Podemos ver que al final cualquier soluci\'on debe tener accinoes que integren todas las dimensiones, ya que de otra manera, \'unicamente se esta solucionando parcialmente el problema, y se trata de posponer lo inevitable.

\subsection*{Conclusi\'on}	
	\par Me parece interesante notar que, a pesar de que se trate de hacer \'enfasis en una de las dimensiones, las otras terminan haciendo precencia en cualquier soluci\'on que sea trate de ser integral y realmente solucionar las problem\'aticas causadas.



%%%%%%%%%%%%%%%%%%%%%%%%%%%%%%%%
%%         Bibliografia        %%
%%%%%%%%%%%%%%%%%%%%%%%%%%%%%%%%%%

\begin{thebibliography}{X}

	\bibitem{basica} S/D. (2020). \emph{U3 \textbar{} Indicadores y dimensiones sustentables en la promoción de alternativas sustentables}. 14 de agosto de 2020, de UnADM Sitio web: \\\url{https://dmd.unadmexico.mx/contenidos/DCSBA/BLOQUE1/BI/01/BDSU/unidad_03/descargables/BDSU_U3_Contenido.pdf}
	\bibitem{unidad1} Bench. (2020) \emph{Evidencia de aprendizaje: Problemáticas de su entorno}. 14 de agosto de 2020, de GitHub Sitio Web: \url{https://github.com/BenchHPZ/UnADM-Biotecnologia/blob/master/B1-1/BDSU/Actividades/BDSU_U1_EA_BERC.pdf}
	\bibitem{actividad3} Bench. (2020) \emph{Actividad 3. U3}. 14 de agosto de 2020, de GitHub Sitio Web: \url{https://github.com/BenchHPZ/UnADM-Biotecnologia/blob/master/B1-1/BDSU/Actividades/BDSU_U3_A3_BERC.md}

	\bibitem{eadic} EQUIPO COMUNICACIÓN. (2002). \emph{Los indicadores de sostenibilidad ambiental}. 14 de agosto de 2020, de eadic Sitio web: \\\url{https://www.eadic.com/los-indicadores-de-sostenibilidad-ambiental/}
	\bibitem{uci} Achkar, M.. (2005). \emph{Indicadores de sustentabilidad}. 14 de agosto de 2020, de Universidad para la Cooperación Internacional Sitio web: \url{http://www.ucipfg.com/Repositorio/MLGA/MLGA-03/semana2/Indicadores_de_sostenibilidad.pdf}
	\bibitem{uabcs} Ibáñez Pérez, R.. (2012) \emph{Indicadores y sustentabilidad: utilidades y limitaciones}. 14 de agosto de 2020, de Universidad Autónoma de Baja California Sur Sitio web \url{https://www.redalyc.org/pdf/4561/456145105006.pdf}
	
\end{thebibliography}

\end{document}