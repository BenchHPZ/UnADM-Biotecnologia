\documentclass[12pt]{article}
\usepackage[spanish]{babel}

%%%%%%%%%%%%%%%%%%%%%%%%%%%%%%%%%%
%%%%%%%%%%%%%%%%%%%%%%%%%%%%%   %%
%%        Datos Trabajo     %%  %%
%%%%%%%%%%%%%%%%%%%%%%%%%%%%%%%%%%
\newcommand{\titulo}	[0]{Autorreflexión}
\newcommand{\materia}	[0]{Desarrollo Sustentable}
\newcommand{\grupo}		[0]{BI-BDSU-2002-B1-012}
\newcommand{\unidad}	[0]{Unidad 2}


%%%%%%%%%%%%%%%%%%%%%%%%%%%%%%%%%%
%%%%%%%%%%%%%%%%%%%%%%%%%%%%%%%%%%
\usepackage{amssymb}
\usepackage{enumerate}
\usepackage{geometry}
\usepackage{mathtools}
\usepackage{multicol}
\usepackage{soul}

\usepackage{graphicx}
	\graphicspath{ {assets/} }

\usepackage{hyperref}
	\hypersetup{
			pdftex,
		        pdfauthor	={bench},
		        pdftitle	={\titulo},
		        pdfsubject	={\materia},
		        pdfkeywords	={\grupo, \unidad, UnADM},
		        pdfproducer	={Latex with hyperref, Ubuntu},
		        pdfcreator	={pdflatex},
			colorlinks	=true,
				linkcolor	=red,
				urlcolor	=cyan,
				filecolor	=yellow}

%%%%%%%%%%%%%%%%%%%%%%%%%%%%%%%%%%
%%%%%%%%%%%%%%%%%%%%%%%%%%%%%%%%%%

\title{
	\includegraphics{../../../assets/logo-unadm} \\
	\ \\ Benjam\'in Rivera \\
	\bf{\titulo}\\\ \\}

\author{
	Universidad Abierta y a Distancia de México \\
	TSU en Biotecnolog\'ia \\
	\textit{Materia:} \materia \\
	\textit{Grupo:} \grupo \\
	\textit{Unidad:} \unidad \\
	\\
	\textit{Matricula:} ES202105994 }

\date{\textit{Fecha de entrega:} \today}


%%%%%%%%%%%%%%%%%%%%%%%%%%%%%
%%        Documento         %%
%%%%%%%%%%%%%%%%%%%%%%%%%%%%%%%
\begin{document}
\maketitle\newpage

\subsection*{¿Consideras que, para un Ingeniero en Biotecnología, es importante conocer las dimensiones del desarrollo sustentable para realizar un proyecto? ¿Por qué?}
	
	\par Por supuesto que si, todo profesional debe tener conciencia de lo que un desarrollo sustentable implica; de esta manera todos podremos contribuir a que en sociedad tengamos un desarrollo sustentable. Especificaente, los \textbf{profesionales en Biotecnología}, estan encargados de proyectos como

\begin{itemize}
	\item Cultivos de bacterias y levaduras (que apoyan a los demás proyectos).
	\item Producción de biocombustibles (apoyan a la reducción de liberación de gases nocivos).
 	\item Plantas transgénicas (reducren la cantidad de agua necesaria para su producción).
	\item Plásticos biodegradables (reducen el impacto de los residuos generados). 
\end{itemize}
podemos ver que estos afectan directamente a nuestra capcidad de coexistir con un \textit{desarrollo sustentable}.
	
	
\subsection*{¿Cuáles conceptos te resultaron más difíciles de comprender?}
	
	\par Me costo un poco comprender el alcance de la \textbf{dimensión social}, la mezclaba un poco con las otras dos dimensiones necesarias por el desarrollo sustentable. El resto del material lo sent\'i bastante digerible, me ayudaron mucho el material extra que se nos proporcionó.
	
	
%%%%%%%%%%%%%%%%%%%%%%%%%%%%%%%%
%%         Bibliografia        %%
%%%%%%%%%%%%%%%%%%%%%%%%%%%%%%%%%%
	\begin{thebibliography}{X}
		
		\bibitem{basica} S/D. (2020). \textit{U2 $|$ Dimensiones y retos de la sustentabilidad}. \today, de UnADM. Sitio web: \url{https://dmd.unadmexico.mx/contenidos/DCSBA/BLOQUE1/BI/01/BDSU/unidad_02/descargables/BDSU_U2_Contenido.pdf} 

	\end{thebibliography}
\end{document}