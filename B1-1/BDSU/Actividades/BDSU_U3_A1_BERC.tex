\documentclass[12pt]{article}
\usepackage[spanish]{babel}

%%%%%%%%%%%%%%%%%%%%%%%%%%%%%%%%%%
%%%%%%%%%%%%%%%%%%%%%%%%%%%%%   %%
%%        Datos Trabajo     %%  %%
%%%%%%%%%%%%%%%%%%%%%%%%%%%%%%%%%%
\newcommand{\titulo}[0]{Actividad 1. Análisis de los indicadores del desarrollo sustentable}
\newcommand{\materia}[0]{Desarrollo Sustentable}
\newcommand{\grupo}[0]{ES202105994}
\newcommand{\unidad}[0]{Unidad 3}


%%%%%%%%%%%%%%%%%%%%%%%%%%%%%%%%%%
%%%%%%%%%%%%%%%%%%%%%%%%%%%%%%%%%%
\usepackage{amssymb}
\usepackage{enumerate}
\usepackage{geometry}
\usepackage{mathtools}
\usepackage{multicol}
\usepackage{soul}

\usepackage{graphicx}
	\graphicspath{ {assets/} }

\usepackage{hyperref}
	\hypersetup{
			pdftex,
		        pdfauthor={bench},
		        pdftitle={\titulo},
		        pdfsubject={\materia},
		        pdfkeywords={\grupo, \unidad, UnADM},
		        pdfproducer={Latex with hyperref, Ubuntu},
		        pdfcreator={pdflatex, or other tool},
			colorlinks=true,
				linkcolor=red,
				urlcolor=cyan,
				filecolor=yellow}

%%%%%%%%%%%%%%%%%%%%%%%%%%%%%%%%%%
%%%%%%%%%%%%%%%%%%%%%%%%%%%%%%%%%%

\title{
	\includegraphics{../../../assets/logo-unadm} \\
	\ \\ Benjam\'in Rivera \\
	\bf{\titulo}\\\ \\}

\author{
	Universidad Abierta y a Distancia de México \\
	TSU en Biotecnolog\'ia \\
	\textit{Materia:} \materia \\
	\textit{Grupo:} \grupo \\
	\textit{Unidad:} \unidad \\
	\\
	\textit{Matricula:} ES202105994 }

\date{\textit{Fecha de entrega:} \today}


\usepackage{longtable}

%%%%%%%%%%%%%%%%%%%%%%%%%%%%%
%%        Documento         %%
%%%%%%%%%%%%%%%%%%%%%%%%%%%%%%%
\begin{document}
\maketitle\newpage

	\par Los indicadores de sustentabilidad son tan variados y dependen de una infinidad de variables, es importante considerar siempre los m\'as relevantes para cada proyecto; seleccionar menos nos dara una vista incompleta del problema, seleccionar m\'as nos dara demasiado trabajo incesario.
	
	\par Para poder seleccionar los indicadores adecuados es necesario comprenderlos completamente.

\begin{longtable}[]{@{}rll@{}}

	\begin{minipage}[b]{0.37\columnwidth}\raggedleft\bf
	Indicador\strut
	\end{minipage} & \begin{minipage}[b]{0.27\columnwidth}\raggedright\bf
	Característica\strut
	\end{minipage} & \begin{minipage}[b]{0.27\columnwidth}\raggedright\bf
	Relación con dimensión\strut
	\end{minipage}\tabularnewline

	\endhead
	\begin{minipage}[t]{0.37\columnwidth}\raggedleft
	Indicadores ambientales\strut
	\end{minipage} & \begin{minipage}[t]{0.27\columnwidth}\raggedright
	Aquellos que se encargan de medir el estado de factores ambientales; por
	ejemplo: Calidad del aire, calidad del agua, biodiversidad, entre
	otros\strut
	\end{minipage} & \begin{minipage}[t]{0.27\columnwidth}\raggedright
	Dimensión ambiental\strut
	\end{minipage}\tabularnewline
	\begin{minipage}[t]{0.37\columnwidth}\raggedleft
	Indicadores del impacto humano sobre el ambiente\strut
	\end{minipage} & \begin{minipage}[t]{0.27\columnwidth}\raggedright
	Son aquellos que miden el impacto del desarrollo humano en su
	ecosistema; por ejemplo: Niveles de gases, densidad poblacional, entre
	otros\strut
	\end{minipage} & \begin{minipage}[t]{0.27\columnwidth}\raggedright
	Dimensión ambiental y social\strut
	\end{minipage}\tabularnewline
	\begin{minipage}[t]{0.37\columnwidth}\raggedleft
	Indicadores de biodiversidad\strut
	\end{minipage} & \begin{minipage}[t]{0.27\columnwidth}\raggedright
	Aquellos que miden la salud de algún ecosistema; por ejemplo: Cantidad
	de especies de plantas y animales, cantidad de especies endémicas,
	densidad de biodiversidad, entre otras\strut
	\end{minipage} & \begin{minipage}[t]{0.27\columnwidth}\raggedright
	Dimensión ambiental\strut
	\end{minipage}\tabularnewline
	\begin{minipage}[t]{0.37\columnwidth}\raggedleft
	Indicadores de nivel social\strut
	\end{minipage} & \begin{minipage}[t]{0.27\columnwidth}\raggedright
	Aquellos que miden el bienestar social y económico de una región; por
	ejemplo: Densidad poblacional, indice de desempleo, slario promedio,
	entre otros\strut
	\end{minipage} & \begin{minipage}[t]{0.27\columnwidth}\raggedright
	Dimensión ambiental y económica\strut
	\end{minipage}\tabularnewline
	\begin{minipage}[t]{0.37\columnwidth}\raggedleft
	Indicadores de salud\strut
	\end{minipage} & \begin{minipage}[t]{0.27\columnwidth}\raggedright
	Aquellos que indican la salud medica de una región; por ejemplo:
	promedio de vida, cantidad de instituciones de salud, enfermedades
	contagiosas identificadas, entre otros\strut
	\end{minipage} & \begin{minipage}[t]{0.27\columnwidth}\raggedright
	Dimensión social y económica\strut
	\end{minipage}\tabularnewline
	\begin{minipage}[t]{0.37\columnwidth}\raggedleft
	Indicadores económicos\strut
	\end{minipage} & \begin{minipage}[t]{0.27\columnwidth}\raggedright
	Son los que indican el bienestar económico de una región; por ejemplo:
	PIB, tasa de paro, tasa de desempleo, entre otros\strut
	\end{minipage} & \begin{minipage}[t]{0.27\columnwidth}\raggedright
	Dimensión económica\strut
	\end{minipage}\tabularnewline
	\begin{minipage}[t]{0.37\columnwidth}\raggedleft
	Indicadores culturales\strut
	\end{minipage} & \begin{minipage}[t]{0.27\columnwidth}\raggedright
	Aquellos que miden la compenetración cultural de una región; por
	ejemplo: indice de analfabetismo, viabilidad de asoceación de grupos,
	entre otros\strut
	\end{minipage} & \begin{minipage}[t]{0.27\columnwidth}\raggedright
	Dimensión cultural\strut
	\end{minipage}\tabularnewline

\end{longtable}


\subsection*{Referencias}

\begin{enumerate}[I]

	\item S/D. (2020). \emph{U3 \textbar{} Indicadores y dimensiones sustentables en la promoción de alternativas sustentables}. 14 de agosto de 2020, de UnADM Sitio web: \\\url{https://dmd.unadmexico.mx/contenidos/DCSBA/BLOQUE1/BI/01/BDSU/unidad_03/descargables/BDSU_U3_Contenido.pdf}
	\item EQUIPO COMUNICACIÓN. (2002). \emph{Los indicadores de sostenibilidad ambiental}. 14 de agosto de 2020, de eadic Sitio web: \\\url{https://www.eadic.com/los-indicadores-de-sostenibilidad-ambiental/}
	\item Achkar, M.. (2005). \emph{Indicadores de sustentabilidad}. 14 de agosto de 2020, de Universidad para la Cooperación Internacional Sitio web: \url{http://www.ucipfg.com/Repositorio/MLGA/MLGA-03/semana2/Indicadores_de_sostenibilidad.pdf}
	\item Ibáñez Pérez, R.. (2012) \emph{Indicadores y sustentabilidad: utilidades y limitaciones}. 14 de agosto de 2020, de Universidad Autónoma de Baja California Sur Sitio web \url{https://www.redalyc.org/pdf/4561/456145105006.pdf}

\end{enumerate}

\end{document}