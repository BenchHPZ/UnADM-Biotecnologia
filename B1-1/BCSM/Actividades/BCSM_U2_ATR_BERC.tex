\documentclass[12pt]{article}
\usepackage[spanish]{babel}

%%%%%%%%%%%%%%%%%%%%%%%%%%%%%%%%%%
%%%%%%%%%%%%%%%%%%%%%%%%%%%%%   %%
%%        Datos Trabajo     %%  %%
%%%%%%%%%%%%%%%%%%%%%%%%%%%%%%%%%%
\newcommand{\titulo}[0]{Autorreflexiones de la unidad 2.}
\newcommand{\materia}[0]{Contexto Socioeconómico de México}
\newcommand{\grupo}[0]{BI-BCSM-2002-B1-012}
\newcommand{\unidad}[0]{Unidad 2}


%%%%%%%%%%%%%%%%%%%%%%%%%%%%%%%%%%
%%%%%%%%%%%%%%%%%%%%%%%%%%%%%%%%%%
\usepackage{amssymb}
\usepackage{enumerate}
\usepackage{geometry}
\usepackage{mathtools}
\usepackage{multicol}
\usepackage{soul}

\usepackage{graphicx}
	\graphicspath{ {assets/} }

\usepackage{hyperref}
	\hypersetup{
			pdftex,
		        pdfauthor={bench},
		        pdftitle={\titulo},
		        pdfsubject={\materia},
		        pdfkeywords={\grupo, \unidad, UnADM},
		        pdfproducer={Latex with hyperref, Ubuntu},
		        pdfcreator={pdflatex, or other tool},
			colorlinks=true,
				linkcolor=red,
				urlcolor=cyan,
				filecolor=yellow}

%%%%%%%%%%%%%%%%%%%%%%%%%%%%%%%%%%
%%%%%%%%%%%%%%%%%%%%%%%%%%%%%%%%%%

\title{
	\includegraphics{../../../assets/logo-unadm} \\
	\ \\ Benjam\'in Rivera \\
	\bf{\titulo}\\\ \\}

\author{
	Universidad Abierta y a Distancia de México \\
	TSU en Biotecnolog\'ia \\
	\textit{Materia:} \materia \\
	\textit{Grupo:} \grupo \\
	\textit{Unidad:} \unidad \\
	\\
	\textit{Matricula:} ES202105994 }

\date{\textit{Fecha de entrega:} \today}


%%%%%%%%%%%%%%%%%%%%%%%%%%%%%
%%        Documento         %%
%%%%%%%%%%%%%%%%%%%%%%%%%%%%%%%
\begin{document}
\maketitle\newpage

\subsection*{¿Por qué es importante conocer el contexto socioeconómico de México?}

\textit{El que no conce su historia esta condenado a repetirla}\footnote{frase atribuida a Napoleon}. Me parece que esta es la frase que describe con gran precición y objetividad la razón por la cual el estudio de la historia, tanto los aspectos sociales como económicos, es una actividad necesaria para cualquier persona, ya que no podemos ignorar las situaciones por las que ya hemos pasado. Especificamente, nosotros como estudiantes de una institución mexicana y como posibles futuros profesionistas en México, debemos conocer todos los procesos que nos han llevado a la situación actual, para saber que caminos evitar en el futuro; aunque esta decisión tambien deba ser con cautela, para evitar entrar en algun ciclo vicioso del cual ya hayamos salido.


\subsection*{¿Qué opinas de la estructura socioeconómica de México?}

Considero que en su momento fué de las mejores decisiones que pudimos haber tomado; sin embargo, la estructura de la nación, en la actualidad, ya esta empezando a quedar obsoleta, debemos empezar a considerar distintas maneras para acomodar nuestro arcáico modelo social, que lleva más de 100 años sin cambiar, en algo que se adapte mejor en las necesidades de la sociedad contemporánea. Creo que una posible solución es reducir el control federal sobte los estados; ya que el país ha crecido tanto, que cada los estados se han vuelto sistemas muy complejos, con necesidades muy distintas entre si.




%%%%%%%%%%%%%%%%%%%%%%%%%%%%%%%%
%%         Bibliografia        %%
%%%%%%%%%%%%%%%%%%%%%%%%%%%%%%%%%%

\begin{thebibliography}{X}
	\bibitem{biblio} S/D. (2020). Introducción al estudio histórico de México. 18 de Julio de 2020, de UnADM Sitio web: \url{https://campus.unadmexico.mx/contenidos/DCSBA/TC/CSM/unidad_01/descargables/CSM_U1_Contenido.pdf}
\end{thebibliography}

\end{document}