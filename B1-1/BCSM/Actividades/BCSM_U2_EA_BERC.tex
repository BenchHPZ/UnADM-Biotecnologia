\documentclass[12pt]{article}
\usepackage[spanish]{babel}

\usepackage{amssymb}
\usepackage{enumerate}
\usepackage{geometry}
\usepackage{mathtools}
\usepackage{multicol}
\usepackage{soul}

\usepackage{graphicx}
	\graphicspath{ {assets/} }


%%%%%%%%%%%%%%%%%%%%%%%%%%%%%%%%%%
%%%%%%%%%%%%%%%%%%%%%%%%%%%%%   %%
%%        Datos Trabajo     %%  %%
%%%%%%%%%%%%%%%%%%%%%%%%%%%%%%%%%%
\newcommand{\titulo}[0]{Evidencia de aprendizaje. Política de Estado en ciencia y tecnología}
\newcommand{\materia}[0]{Contexto Socioeconómico de México}
\newcommand{\grupo}[0]{BI-BCSM-2002-B1-012}
\newcommand{\unidad}[0]{Unidad 2}

%%%%%%%%%%%%%%%%%%%%%%%%%%%%%%%%%%
%%%%%%%%%%%%%%%%%%%%%%%%%%%%%%%%%%


\usepackage[pdftex,
		        pdfauthor={bench},
		        pdftitle={\titulo},
		        pdfsubject={\materia},
		        pdfkeywords={\grupo, \unidad, UnADM},
		        pdfproducer={Latex with hyperref, or other system},
		        pdfcreator={pdflatex, or other tool},
            colorlinks=true,
				linkcolor=red,
				urlcolor=cyan,
				filecolor=yellow
			]{hyperref}
%%%%%%%%%%%%%%%%%%%%%%%%%%%%%%%%%%
%%%%%%%%%%%%%%%%%%%%%%%%%%%%%%%%%%

\title{
	\includegraphics{../../../assets/logo-unadm} \\
	\ \\ Benjam\'in Rivera \\
	\bf{\titulo}\\\ \\}

\author{
	Universidad Abierta y a Distancia de México \\
	TSU en Biotecnolog\'ia \\
	\textit{Materia:} \materia \\
	\textit{Grupo:} \grupo \\
	\textit{Unidad:} \unidad \\
	\\
	\textit{Matricula:} ES202105994 }

\date{\textit{Fecha de entrega:} \today}


%%%%%%%%%%%%%%%%%%%
%%%%%%%%%%%%%%%%%%%%%
%%        Local     %% 
%%%%%%%%%%%%%%%%%%%%%%%

\newcommand{\periodo}[0]{Institucionalizaci\'on}

%%%%%%%%%%%%%%%%%%%%%%%%%%%%%
%%        Documento         %%
%%%%%%%%%%%%%%%%%%%%%%%%%%%%%%%
\begin{document}
\maketitle\newpage

\subsection*{Introducci\'on}	

	\par En este trabajo se tratara de indentificar y reflexionar acerca de las pol\'iticas que el estado mexiano tomo respoecto a la Ciencia y la Tecnolog\'ia durante el periodo conocido como \textbf{el Desarrollo y la Institucinalizaci\'on} del Estado Mexicano, que abarc\'o desde \textit{1920} hasta \textit{1940}.

	\par Durante el resto del documento llamaremos al periodo de estudio como \textbf{\periodo}.
	
	
\subsection*{Desarrollo}

	\par La \periodo fue el periodo que sigui\'o de la \textbf{Construcci\'on Nacional}. Al final de la Revoluci\'on Mexicana se necesitaba una forma de hacer legitima la revoluci\'on, la soluci\'on que encontraron a esto fu\'e la reforma a la previa constituci\'on, por lo que se promulgo la \textit{constituci\'on de 1917}. Esto dio pie a que el pais dejara a los heroes revolucionarios y pasara a regirse por instituciones, lo que conocemos como el proceso \textbf{del Caudillismo a la \periodo}, sin emabargo, antes de poder llegar a la \periodo, tuvimos que pasar por un breve periodo de Construcci\'on Nacional. Una vez hubo concluido este periodo, dio inicio la \textbf{\periodo}.
	
	\par Durante la \periodo, se buscaba enftizar un estado leg\'itimo que dependiera de un c\'odigo y donde todos tuvieran representaci\'on, como mencionamos en el parrafo anterior, se dio pie a esto en la constituci\'on de 1917. Adem\'as se empez\'o a producir una masiva migraci\'on de la poblaci\'on del campo a las ciudades, esto provoc\'o que hubiera mucha mano de obra disponible. 
	
	\par Para lograr la legitimizaci\'on del estado, se resolvi\'o crear instituciones encargadas de atender sectores especificos del desarrollo del pa\'is. Algunas de las instituciones que se crearon durante este periodo fuer\'on el \textit{Partido Nacional Revolucionario}
\footnote{Que psoteriormente se convertir\'ia en el \textbf{PRI}}
	la \textit{Confederaci\'on de Trabajadores de M\'exico}, el \textit{Banco de M\'exico} y la \textit{Confederaci\'on Nacional Campesina}.
	
	\par Dentro de las instituciones creadas, en el ambito de ciencia y tecnolog\'ia, podemos resaltar la creaci\'on de 
	
	\begin{itemize}
		\item la \textit{Secretaria de Educaci\'on Publica} en  1921 y
		\item el \textit{Instituto Politecnico Nacional} en 1936
	\end{itemize}
	
	todas estas instituciones que en cierta medidad se encrgan de promover e impulsar la ciencia y la tecnolog\'ia en el pa\'is. Un poco despu\'es de esta \'epoca, se fundo la maxima instituci\'on de ciencias y tecnolog\'ias, el \textit{Consejo Nacional de Ciencia y Tecnolog\'ia}, pero ya hasta 1970.
	
	\par En el ambito internacional, adoptamos politicas proteccionistas que priorizaban el producto mexicano sobre el extranjero, como impuestos a exportaciones y apoyos a econom\'ias locales, esto nos ayudo mucho a estar preparados durante la \textit{segunda guerra mundial} para poder satisfacer la demanda de los productos que las demas naciones hab\'ian dejado de lado por la industria b\'elica. Adem\'as, de acuerdo con las reformas de la revoluci\'on, nacionalizamos todos los recursos del territorio mexicano, esto se ve reflejado en acciones como la \textit{expropiaci\'on petrolera} en 1938.

	\par La fuerte indusrtializaci\'on de la naci\'on y la gran cantidad de mano de obra dispuesta a trabajar en las industrias ocasion\'o que la educaci\'on t\'ecnica tomara mucha importancia. Y que el desarrollo de las tecnolog\'ias fuera relevante.
	
	
\subsection*{Conclusiones}
	\par Creo que el enfoque que se le dio a las tecnolog\'ias y la industrializaci\'on durante la \periodo, fu\'e una gran estrategia en ese momento, al igual que la politica de proteccionismo; eran propios de un pa\'is en desarrollo. Desafortunadamente, la \textit{ciencia y la tecnolog\'ia} han sido dejadas de lado hoy en d\'ia, y seguimos aferrandonos al proteccionismo; una pol\'itica que desde mi perspectiva, en un mundo globalizado, ya no es deber\'ia de existir, en cambio, deber\'ia de haber una apertura a todas las ideas, acompa\~nado de una modernizaci\'on en nuestros ideales en la educaci\'on y las ciencias.
%%%%%%%%%%%%%%%%%%%%%%%%%%%%%%%%
%%         Bibliografia        %%
%%%%%%%%%%%%%%%%%%%%%%%%%%%%%%%%%%
\newpage
\begin{thebibliography}{X}
	\bibitem{biblio} UnADM. (S/D). \textit{U2 $|$ Historia econ\'omica y pol\'itica de M\'exico en el siglo XX}. \today, de Universidad Abierta y a Distancia de M\'exico $|$ DCSBA. Sitio web: \url{https://campus.unadmexico.mx/contenidos/DCSBA/TC/CSM/unidad_02/descargables/CSM_U2_Contenido.pdf}
\end{thebibliography}

\end{document}