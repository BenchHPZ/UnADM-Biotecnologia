\documentclass[12pt]{article}
\usepackage[spanish]{babel}

%%%%%%%%%%%%%%%%%%%%%%%%%%%%%%%%%%
%%%%%%%%%%%%%%%%%%%%%%%%%%%%%   %%
%%        Datos Trabajo     %%  %%
%%%%%%%%%%%%%%%%%%%%%%%%%%%%%%%%%%
\newcommand{\titulo}[0]{Autorreflexiones de la unidad 3.}
\newcommand{\materia}[0]{Contexto Socioeconómico de México}
\newcommand{\grupo}[0]{BI-BCSM-2002-B1-012}
\newcommand{\unidad}[0]{Unidad 3}


%%%%%%%%%%%%%%%%%%%%%%%%%%%%%%%%%%
%%%%%%%%%%%%%%%%%%%%%%%%%%%%%%%%%%
\usepackage{amssymb}
\usepackage{enumerate}
\usepackage{geometry}
\usepackage{mathtools}
\usepackage{multicol}
\usepackage{soul}

\usepackage{graphicx}
	\graphicspath{ {assets/} }

\usepackage{hyperref}
	\hypersetup{
			pdftex,
		        pdfauthor={bench},
		        pdftitle={\titulo},
		        pdfsubject={\materia},
		        pdfkeywords={\grupo, \unidad, UnADM},
		        pdfproducer={Latex with hyperref, Ubuntu},
		        pdfcreator={pdflatex, or other tool},
			colorlinks=true,
				linkcolor=red,
				urlcolor=cyan,
				filecolor=yellow}

%%%%%%%%%%%%%%%%%%%%%%%%%%%%%%%%%%
%%%%%%%%%%%%%%%%%%%%%%%%%%%%%%%%%%

\title{
	\includegraphics{../../../assets/logo-unadm} \\
	\ \\ Benjam\'in Rivera \\
	\bf{\titulo}\\\ \\}

\author{
	Universidad Abierta y a Distancia de México \\
	TSU en Biotecnolog\'ia \\
	\textit{Materia:} \materia \\
	\textit{Grupo:} \grupo \\
	\textit{Unidad:} \unidad \\
	\\
	\textit{Matricula:} ES202105994 }

\date{\textit{Fecha de entrega:} \today}


%%%%%%%%%%%%%%%%%%%%%%%%%%%%%
%%        Documento         %%
%%%%%%%%%%%%%%%%%%%%%%%%%%%%%%%
\begin{document}
\maketitle\newpage

\subsection*{¿Qué impacto tienen el Neoliberalismo y la Globalización en el ámbito de la Biotecnología?}

	\par Consideremos que la Globalización es un impacto del Neoliberalismo, de manera que únicamente hablaremos del impacto de la Globalización en la Biotecnología.

	\par Considero que la Biotecnología, como todas las ciencias e ingienerías, se ha beneficiado del fenómeno de masificación de la comunicación que vino como consecuencia de la globalización; gracias a esta modernización en la comunicación y a la nueva velocidad de transferencia de datos, información y noticias a lo largo del mundo, cualqueir investigador ahora tiene la capacidad de colaborar, en tiempo real, con alguien que no viva en la misma ciudad, país o incluso contiente. Sin esta velocidad a la que todo se  comunica hoy en día, tanto las ciencias como las ingienerías, y con esto la tecnología del mundo, se iría desarrollando mucho más lento y  habría demsiada redundancia en los témas de investigación, lo que casuaría que todos cometieramos los mismos errores.

\subsection*{En tu desarrollo personal, ¿De qué forma te ha servido conocer el contexto socio-económico de México?}

	\par En la vida diaria, me ayuda a ser un mejor ciudadano, a escoger con mayor información y por cultura general. Creo que la verdadera utilidad de esta clase en todos los aspectos esta contenida en la siguiente frase:
	\begin{quote}\begin{quote}\it
		El que no conoce su historia, esta condenado a repetirla.
	\end{quote}\end{quote}
	
	\par Todas las historias del mundo, tanto la del mundo, como la de los grandes paises y la de las personas, tienen cierta similaridad. La manera en que los paises resuelven sus conflictos, se puede aprecira también entre personas, y eventualmente lo apreciaremos entre mundos.


%%%%%%%%%%%%%%%%%%%%%%%%%%%%%%%%
%%         Bibliografia        %%
%%%%%%%%%%%%%%%%%%%%%%%%%%%%%%%%%%

\begin{thebibliography}{X}
	\bibitem{biblio} S/D. (2020). Introducción al estudio histórico de México. 3 de agosto de 2020, de UnADM Sitio web: \url{https://campus.unadmexico.mx/contenidos/DCSBA/TC/CSM/unidad_03/descargables/CSM_U3_Contenido.pdf}
\end{thebibliography}

\end{document}