\documentclass[12pt]{article}
\usepackage[spanish]{babel}

\usepackage{amssymb}
\usepackage{array}
\usepackage{enumerate}
\usepackage{geometry}
\usepackage{multicol}

\usepackage{graphicx}
	\graphicspath{ {assets/} }


%%%%%%%%%%%%%%%%%%%%%%%%%%%%%%%%%%
%%%%%%%%%%%%%%%%%%%%%%%%%%%%%   %%
%%        Datos Trabajo     %%  %%
%%%%%%%%%%%%%%%%%%%%%%%%%%%%%%%%%%
\newcommand{\titulo}[0]{Evidencias de aprendizaje. Per\'iodos y procesos hist\'oricos en M\'exico}
\newcommand{\materia}[0]{Contexto Socioecon[omico de M\'exico}
\newcommand{\grupo}[0]{BI-BCSM-2002-B1-012}
\newcommand{\unidad}[0]{Unidad 1}
%%%%%%%%%%%%%%%%%%%%%%%%%%%%%%%%%%
%%%%%%%%%%%%%%%%%%%%%%%%%%%%%%%%%%
\usepackage[pdftex,
            pdfauthor={bench},
            pdftitle={\titulo},
            pdfsubject={\materia},
            pdfkeywords={\grupo, \unidad, UnADM},
            pdfproducer={Latex with hyperref, or other system},
            pdfcreator={pdflatex, or other tool}]{hyperref}
%%%%%%%%%%%%%%%%%%%%%%%%%%%%%%%%%%
%%%%%%%%%%%%%%%%%%%%%%%%%%%%%%%%%%

\title{
	\includegraphics{../../../assets/logo-unadm} \\
	\ \\ Benjam\'in Rivera \\
	\bf{\titulo}\\\ \\}

\author{
	Universidad Abierta y a Distancia de México \\
	TSU en Biotecnolog\'ia \\
	\textit{Materia:} \materia \\
	\textit{Grupo:} \grupo \\
	\textit{Unidad:} \unidad \\
	\\
	\textit{Matricula:} ES202105994 }

\date{\textit{Fecha de entrega:} \today}


%%%%%%%%%%%%%%%%%%%%%%%%%%%%%
%%        Documento         %%
%%%%%%%%%%%%%%%%%%%%%%%%%%%%%%%
\begin{document}
\maketitle\newpage

\subsection*{Introducci\'on}
	
	\par El estudio de los cambios hist\'oricos, de cualquier regi\'on o pais, debe comprender una investigaci\'on integral que permita corroborar distintos puntos de vista y complementar las distintas fuentes de informaci\'on que utilicemos durante la misma. En esta unidad estudiamos tres enfoques de investigaci\'on que, siendo complementarios entre si, pueden ser aplicables en este ensayo. En la tabla de la figura \ref{tab:enfoques_investigacion} se trata de explicar y definir los enfoques antes mencionados.
	
\begin{figure}
	\centering
	
	\begin{tabular}{|m{2.5cm}||b{3cm}|b{3cm}|b{3cm}|b{3cm}|}
		\hline
			&\bf Objeto de estudio &\bf M\'etodo de investigaci\'on &\bf Aplicaci\'on pr\'actica &\bf Opini\'on personal \\
		\hline\hline
			\bf Investigaci\'on hist\'orica & 
			Busca realizar una reconstrucción social y cultural a partir de fuentes primarias y secundarias & 
			Recolección, evaluación, verificación y sintetización de evidencias para obtener conclusiones b\'asicas & 
			Censo poblacional o determinar características de politicas publicas según datos historicos & 
			 Aquel que no conzca su historia, esta condenado a repetirla \textit{(Frase atribuida a Napoleon Bonaparte)}\\
		\hline
			\bf Investifaci\'on econ\'omica & 
			Es aquella que tiene como prioridad principal entender los m\'etodos más óptimos para administrar recursos y maximizar el rendimiento &
			 Utiliza distintos metodos, como: Analítico, Inductivo, Deductivo y el Metodo Cientifico & 
			 Tomar decisiones óptimas para administrar los recursos de un grupo, administración pública & 
			 Considero que esta ha sido de las que más se ha abusado, y se ha ignorado demasiado a las historicas y social a la hora de tomar grandes decisiones\\
		\hline
			\bf Investigaci\'on social &
			Es la que estudia las interacciones de los individuos y grupos con grupos más grandes & 
			Método científico & 
			Psicología de masas, predicción de tendencias, análisis de necesidades de grupos sociales & 
			Creo que es de la más ignorada y, que cuando se usa, no se suele hacer con buenas intenciones\\
		\hline
	\end{tabular}

	\caption{Tabla comparativa de los enfoques de investigaci\'on. Producto de la \textit{Fase 1} de la actividad}
	\label{tab:enfoques_investigacion}
\end{figure}

	\par Antes de continuar, me parece es util que definamos los procesos historicos, de M\'exico, en los que se centrara este trabajo. \footnote{Utilizaremos los mencionados en \cite[p. 29]{basica}} En este ensayo se enfocara en analizar la:
	
	\begin{quote}\begin{itemize}
		\item Revoluci\'on Mexicana (1910-1920)
		\item Reconstrucci\'on Nacional (1920-1940)
		\item Desarrollo estabilizador (1940-1970)
		\item Neolibrealismo y Globalizaci\'on
	\end{itemize}\end{quote}
	
	\par Para cada uno de los anteriores identificaremos personajes relevantes, hechos relevantes, contexto social y econ\'omico. Este ensayo se enfocara principalmente econ\'omicos y sociales.
	
	
	

\subsection*{Desarrollo}
	\par A lo largo de esta secci\'on se tomara en cuenta el contenido de la tabla de la figura \ref{tab:enfoques_investigacion} para enfocarnos en los aspectos econ\'omicos y sociales de los acontecimientos relevantes.
	
	

	\subsubsection*{Revoluci\'on Mexicana}

		\par Fu\'e un acontecimiento armado, considerado entre los m\'as importantes de la historia moderna de nuestro pa\'is, e incluso de Am\'erica Latina. 
		\par La magnitud y alcance del mismo es muy amplia, al igual que los involucrados y afectados. Algunas de las concsecuencias m\'as importantes que podmeos identificar de este son:\\\ \\\
		\begin{quote}\begin{itemize}
			\item la instauraci\'on de una \textbf{democracia efectiva} en el pa\'is, 
			\item la creci\'on de una nueva constituci\'on,
			\item la declaraci\'on y constitucionalización de los derechos individuales,
			\item la \textbf{Ley de la Reforma Agraria}, 
			\item entre otros.
		\end{itemize}\end{quote}
				
		\par Sin embargo, para que todo eso fuera posible, hubo varios antecedentes que se juntaron para que se desencadenara todo. Por el \textbf{enfoque econ\'omico} podemos identificar que los extranjeros eran quienes m\'as estaban siendo beneficiados por toda la maquinaria econ\'omica existente en m\'exico. Incluso despu\'es de la independencia, se hab\'ia creado un grupo muy selecto beneficiado por el progreso, dentro de estas personas se encontraba Porfirio D\'iaz, quien priorizo el progreso sobre el bienestar de la sociedad. 
		\par El descontento creado entre el grueso de la poblaci\'on y la inconformidad de algunas personas en el grupo beneficiado fu\'e lo que termino por desencadenar todos lo hechos que conforman al proceso de la \textit{Revoluci\'on Mexicana}.
		\par Como antecedentes econ\'omicos podemos destacar
		
		\begin{quote}\begin{itemize}
			\item Alto porcentaje de tierras controladas por un peque\~no grupo selecto
			\item Altos impuestos para agricultores y artesanos
			\item Comienzo de la explotaci\'on petrolera
			\item Inflaci\'on de importaciones y devaluaci\'on de las exportaciones internacionales
		\end{itemize}\end{quote}
		y podemos agregar como sociales a
		
		\begin{quote}\begin{itemize}
			\item Estafa de sociedades indigeneas y personas no acostumbradas a la propiedad privada
			\item Analfabetización de las grandes masas
			\item Rebeliones indigenas y represi\'on gubernamental
		\end{itemize}\end{quote}
		
		\par Dentro de este proceso podemos destacar como personajes importantes a Porfirio D\'iaz, Francisco I. Mader, Venustiano Carranza, Bernardo Reyes, James Creelman. Algunos de los hechos m\'as destacables fueron la Renuncaia de Porfirio D\'Iaz, la presidencia de Madero, la Dictadura de Huerta, el Plan de Guadaluoe y el Congreso Constituyente


	\subsubsection*{Reconstrucci\'on Nacional (1920-1940)}
	
		\par Posterior al conflicto de revoluci\'on las carencias de la sociedad estaban muy marcadas, aunque hab\'ia un lienzo blanco donde empezar a escribir una nueva historia. Este periodo tambi\'en es conocido como el momento en que se paso del caudillismo al presidencialismo, adem\'as se fundaron muchas instituciones.
		\par Algunos de los hechos que resaltan en este priodo son
		
		\begin{quote}\begin{itemize}
			\item Fundaci\'on de multiples partidos pol\'iticos
			\item Creaci\'on de la \textit{Secretaria de Educaci\'on Publica}
			\item Se realiz\'o el reparto agrario
			\item Creaci\'on del banco de M\'exico
			\item La Guerra cristera de M\'exico (1926-1929)
			\item La exropiaci\'on petrolera
		\end{itemize}\end{quote}
		durante este periodo resaltan nombres como Lazaro Cardenas, Plutarco El\'ias Calles, Venustiano Carranza, Adolfo de la Huerta, Pascual Orozco, Jose Vasconcelos, entre otros.
		\par Todo este periodo tuvo un principal \textbf{causante social}, el \textbf{cansancio de la guerra}. Y la guerra tambi\'en desencadeno el principal \textbf{causante econ\'omico}, el desgaste de las arcas nacionales \textbf{redujo el poder econ\'omico} de la naci\'on.
		
	
	\subsubsection*{Desarrollo estabilizador (1940-1970)}
	
		\par El Desarrollo Estabilizador es tambi\'en conocido como el \textit{Milagro Mexicano}, y fu\'e el modelo econ\'omico que llevo a M\'exico a su mayor apogeo econ\'omico. Priorizaba buscar un desarrollo econ\'omico continuo y equitativo para toda la poblaci\'on.
		\par En este periodo destacan principalmente los presidentes de la naci\'on, que en dichas fechas fueron Manuel Ávila Camacho,Miguel Alemán Valdés, Adolfo Ruiz Cortines, Adolfo López Mateos y Gustavo Díaz Ordaz.
		\par Los antecedentes sociales m\'as representativos que apoyaron a que este period fuera posible son
		\begin{quote}\begin{itemize}
			\item Pueblo en paz y recibiendo los frutos de la revoluci\'on y la reconstrucci\'on nacional.
			\item Control del pa\'is democraticamente
			\item Fin de la Segunda Guerra Mundial
		\end{itemize}\end{quote}
		a estas podemos agregar en el ambito econ\'omico a
		
		\begin{quote} \begin{itemize}
			\item Capacidad de exportaci\'on en varias industrias
			\item Capacidad internacional en industria b\'elica
		\end{itemize}\end{quote}
	
\begin{figure}
	\centering
	
	\begin{tabular}{|b{3cm}||b{10cm}|}
		\hline
			\bf Periodo Hist\'orico &
			Desarrollo estabilizador (1940-1970) \\
		\hline
			\bf Personajes&
			Antonio Carrillo Flores, José López Portillo, Adolfo Ruiz Cortines, Miguel Alemán, Manuel Ávila Camacho, \\
		\hline
			\bf Hechos relevantes & 
			Crecimiento e indistrialización nacinoal, estabilidad política y un crecimiento económico, construye el Centro Hospitalario 20 de Noviembre del ISSSTE, se crea "Compañía Mexicana de Luz y Fuerza", Abundancia económica del pais, La Edad de Oro del Capitalismo, movimiento estudiantil mexicano de 1968, \\
		\hline
	\end{tabular}	
	\caption{Ficha de trabajo del Desarrollo Estabilizador. Producto de la \textit{Fase 2} de la actividad.}

\end{figure}



	\subsubsection*{Neoliberalismo y Globalizaci\'on}
	
	\par Siendo el \textit{neoliberalismo}, el sistema pol\'itico que busca impulasr las inversiones extranjeras en la regi\'on acelerando las fuerzas productivas, y la \textit{globalizaci\'on}, el procso hist\'orico presentado por la acelrada comunicaci\'on y comercio internacional, aspectos muy realcionados de la historia contemporanea de nuestro pa\'is, me parece correcto analizarlos en la misma secci\'on. 
	
	\par Ambos procesos son posibles de implementar en M\'exico gracias a las grandes expectaivas que dejo el pais despu\'es del Desarrollo Globalizador. Podmeos destacar
	\begin{quote}\begin{itemize}
		\item Alto nivel de industrializaci\'on nacional. \textbf{(Econ\'omico)}
		\item Alto promedio de alfabetizaci\'on social. \textbf{(Social)}
		\item Situacu\'on post guerra mundial internacional. \textbf{(Social)}
	\end{itemize}\end{quote}
	
	\par Considero que a\'un es demasiado fresca esta etapa de la historia como para poder identificar a personajes relevantes de ella, es algo que la historia esta a\'un por decidir. Aunque de hechos podemos mencionar el Tratado Atl\'antico Norte, acercamiento a instituciones internacionales como el Banco Mundial y el Fondo Monetario Internacional, tratados de libre comercio, entre otros.




\subsection*{Conclusi\'on}

	\par A lo largo de la historia y los procesos que estudiamos en esta unidad, hemos visto que cada una de las acciones que vamos realizando va repercutiendo m\'as adelante en la historia y es utilizado por los siguientes actores historicos para impulsar sus objetivos.
	\par Recientemente el futuro de la naci\'on era bastante incierto, aunque parec\'ia prometedor. Parec\'ia que ya hab\'iamos vuelto a tener un lugar para tener un impulso rumbo a una nueva de oro, hasta que la situaci\'on m\'as reciente nos detuvo en seco, y a todo el mundo tambi\'en.
	\par Considero que esta pandemia fu\'e un momento ideal donde podemos ver que ahora necesitamos una nueva campa\~na de alfabetizaci\'on, a nivel casi mundial, donde aprendamos a utilziar estas herramientas teconol\'ogicas que han prosperado a nuestro problema actual, y que al mismo tiempo han limitado el desarrollo de aquellos que no estaban preparados. 
	\par Definitivamente la vida virtual no sera jam\'as un sustituto de nuestras necesidades sociales, pero sera una ayuda para solventar cualquier otro inconveniente que le surga a la humanidad.

%%%%%%%%%%%%%%%%%%%%%%%%%%%%%%%%
%%         Bibliografia        %%
%%%%%%%%%%%%%%%%%%%%%%%%%%%%%%%%%%
\newpage
\begin{thebibliography}{X}
	\bibitem{basica} S/D. (2020). \textit{Introducción al estudio histórico de México}. 18 de Julio de 2020, de UnADM Sitio web: \url{https://campus.unadmexico.mx/contenidos/DCSBA/TC/CSM/unidad_01/descargables/CSM_U1_Contenido.pdf}
	\bibitem{Americas} Henr\'iquez, A..(2013) \textit{Investigacion historica} 18 de julio de 2020, de Universidad de las Améticas Sitio web: \url{https://es.slideshare.net/isabelolmeda/investigacion-historica}
	\bibitem{economica} Bravo, R..(1994) \textit{Metodología de la Investigación Económica}. Ciudad de México: Editorial Alhambra Mexicana.
	\bibitem{UNAM} Tello, C..(2010). \textit{Notas sobre el Desarrollo Estabilizador.} 19 de julio de 2020, de Facultad de Economía, UNAM Sitio web: \url{http://www.economia.unam.mx/publicaciones/econinforma/pdfs/364/09carlostello.pdf}
	\bibitem{fondo} Bloch, M.. (1982). \textit{Introducci[oon a la historia}. M\'exico: Fondo de Cultura Econ\'omica

\end{thebibliography}

\end{document}