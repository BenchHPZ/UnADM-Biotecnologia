\documentclass[12pt]{article}
\usepackage[spanish]{babel}

%%%%%%%%%%%%%%%%%%%%%%%%%%%%%%%%%%
%%%%%%%%%%%%%%%%%%%%%%%%%%%%%   %%
%%        Datos Trabajo     %%  %%
%%%%%%%%%%%%%%%%%%%%%%%%%%%%%%%%%%
\newcommand{\titulo}[0]{Actividad 2. Foro de construcción del conocimiento. Neoliberalismo y globalización.}
\newcommand{\materia}[0]{Contexto Socioeconómico de México}
\newcommand{\grupo}[0]{BI-BCSM-2002-B1-012}
\newcommand{\unidad}[0]{Unidad 3}

%%%%%%%%%%%%%%%%%%%%%%%%%%%%%%%%%%
%%%%%%%%%%%%%%%%%%%%%%%%%%%%%%%%%%
\usepackage{amssymb}
\usepackage{enumerate}
\usepackage{geometry}
\usepackage{mathtools}
\usepackage{multicol}
\usepackage{soul}

\usepackage{graphicx}
	\graphicspath{ {assets/} }

\usepackage{hyperref}
	\hypersetup{
			pdftex,
		        pdfauthor={bench},
		        pdftitle={\titulo},
		        pdfsubject={\materia},
		        pdfkeywords={\grupo, \unidad, UnADM},
		        pdfproducer={Latex with hyperref, Ubuntu},
		        pdfcreator={pdflatex, or other tool},
			colorlinks=true,
				linkcolor=red,
				urlcolor=cyan,
				filecolor=yellow}

%%%%%%%%%%%%%%%%%%%%%%%%%%%%%%%%%%
%%%%%%%%%%%%%%%%%%%%%%%%%%%%%%%%%%

\title{
	\includegraphics{../../../assets/logo-unadm} \\
	\ \\ Benjam\'in Rivera \\
	\bf{\titulo}\\}

\author{
	Universidad Abierta y a Distancia de México \\
	TSU en Biotecnolog\'ia \\
	\textit{Materia:} \materia \\
	\textit{Grupo:} \grupo \\
	\textit{Unidad:} \unidad \\
	\\
	\textit{Matricula:} ES202105994 }

\date{\textit{Fecha de entrega:} \today}


%%%%%%%%%%%%%%%%%%%%%%%%%%%%%
%%        Documento         %%
%%%%%%%%%%%%%%%%%%%%%%%%%%%%%%%


%%%%%%%%%%%%%%%%%%%%%%%%%%%%%
%%        Documento         %%
%%%%%%%%%%%%%%%%%%%%%%%%%%%%%%%
\begin{document}
\maketitle
\noindent\makebox[\linewidth]{\rule{\paperwidth}{0.4pt}}

	\par Primero me definiremos los conceptos importantes de la actividad, de manera que, para empezar podemos decir que
	\begin{quote}\it
		el \textbf{neoliberalismo} es el ideal político y económico que busca limitar la intervención del Estado en asuntos del pueblo
	\end{quote}
	además de introducir la teocracia a la forma de gobierno;especificamente en México, esta fue la etapa posterior al milagro mexicano donde buscabamos mantener una política proteccionista y eso provocó que el desarrollo nacional en el ámbito global fuera muy lento. Por otro lado
	\begin{quote}
		la \textbf{globalización} es el proceso político y económico mediante el cual se impulso la comunicación e interdependencia entre los paises del mundo.
	\end{quote}
	
	\par De manera que en México, el neoliberalismo y la globalización, estan relacionadas porque el primero retraso a la segunda. Sin embargo, una vez el neoliberaelismo se ajusto y la política nacional lo permitio, la union de los dos anteriores permitió algunas acciones, ejemplo de estas son
	\begin{itemize}
		\item El Tratado de Libre Comercio \textbf{(TLCAN)}
		\item Reducci\'on de empresas paraestatales
		\item Apertura a la inversi\'on privada
		\item Garantizaci\'on de la autonom\'ia del \textbf{Banco de M\'exico}
		\item Apertura a mercadis internacionales
	\end{itemize}
	estas acciones permitieron que el modelo neoliberal se fortaleciera en el país y tuvieramos una aprertura internacional que nos permitiera competir con los grandes y llegar a ser considerado país en vías de desarrollo.
	
	
%%%%%%%%%%%%%%%%%%%%%%%%%%%%%%%%
%%         Bibliografia        %%
%%%%%%%%%%%%%%%%%%%%%%%%%%%%%%%%%%

\noindent\makebox[\linewidth]{\rule{\paperwidth}{0.4pt}}
\begin{thebibliography}{X}
	\bibitem{basica} UnADM. (2020). \textit{U3 $|$ EL neoliberalismo y la política de Estado en ciencia y tecnología}. 2 de agosto de 2020, de División de Ciencias de la Salud, biológicas y Ambientales Sitio web: \url{https://campus.unadmexico.mx/contenidos/DCSBA/TC/CSM/unidad_03/descargables/CSM_U3_Contenido.pdf}
\end{thebibliography}

\end{document}