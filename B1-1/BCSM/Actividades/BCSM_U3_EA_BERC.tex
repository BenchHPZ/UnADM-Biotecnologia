\documentclass[12pt]{article}
\usepackage[spanish]{babel}

%%%%%%%%%%%%%%%%%%%%%%%%%%%%%%%%%%
%%%%%%%%%%%%%%%%%%%%%%%%%%%%%   %%
%%        Datos Trabajo     %%  %%
%%%%%%%%%%%%%%%%%%%%%%%%%%%%%%%%%%
\newcommand{\titulo}[0]{Evidencia de aprendizaje: Política de estado en ciencia y tecnología.}
\newcommand{\materia}[0]{Contexto Socioeconómico de México}
\newcommand{\grupo}[0]{BI-BCSM-2002-B1-012}
\newcommand{\unidad}[0]{Unidad 3}

%%%%%%%%%%%%%%%%%%%%%%%%%%%%%%%%%%
%%%%%%%%%%%%%%%%%%%%%%%%%%%%%%%%%%
\usepackage{amssymb}
\usepackage{enumerate}
\usepackage[margin=3.3cm]{geometry}
\usepackage{mathtools}
\usepackage{multicol}
\usepackage{soul}

\usepackage{graphicx}
	\graphicspath{ {assets/} }

\usepackage{hyperref}
	\hypersetup{
			pdftex,
		        pdfauthor={bench},
		        pdftitle={\titulo},
		        pdfsubject={\materia},
		        pdfkeywords={\grupo, \unidad, UnADM},
		        pdfproducer={Latex with hyperref, Ubuntu},
		        pdfcreator={pdflatex, or other tool},
			colorlinks=true,
				linkcolor=red,
				urlcolor=cyan,
				filecolor=yellow}

%%%%%%%%%%%%%%%%%%%%%%%%%%%%%%%%%%
%%%%%%%%%%%%%%%%%%%%%%%%%%%%%%%%%%

\title{
	\includegraphics{../../../assets/logo-unadm} \\
	\ \\ Benjam\'in Rivera \\
	\bf{\titulo}\\\ \\}

\author{
	Universidad Abierta y a Distancia de México \\
	TSU en Biotecnolog\'ia \\
	\textit{Materia:} \materia \\
	\textit{Grupo:} \grupo \\
	\textit{Unidad:} \unidad \\
	\\
	\textit{Matricula:} ES202105994 }

\date{\textit{Fecha de entrega:} \today}


%%%%%%%%%%%%%%%%%%%%%%%%%%%%%
%%        Documento         %%
%%%%%%%%%%%%%%%%%%%%%%%%%%%%%%%


%%%%%%%%%%%%%%%%%%%%%%%%%%%%%
%%        Documento         %%
%%%%%%%%%%%%%%%%%%%%%%%%%%%%%%%
\begin{document}
\maketitle\newpage

	\par Desde los inicios del tiempo, la humanidad ha vivido en sociedad. La fortaleza de nuestra espcie, no radica en nuestra fuerza, ni en nuestra capacidad de supervivencia, ni en nuestra resistencia, si no en nuestra capacidad para comuniarnos en sociedad y poder utilizar el ingenio para optimizar ciertos procesos. Dado que nuestra especie ha vivido en sociedad pora alg\'un tiempo, hemos tenido la capcaidad para desarrollar distintas formas de orgnizarnos. Y las distintas capacidades de cada individuo, y las necesidades de la comunidad, le han dado valor a cada una de las personas; no ha sido discriminaci\'on, ha sido selecci\'on natural, del m\'as apto para cada actividad que se ha requerido en la comunidad.
	
	\par En ciertas ocasiones, los m\'as aptos han decidido obtener provecho de sus capacidades privilegiadas para desarrollar actividades relevantes en la comunidad. Para solucionar esto, hemos tratado de implementar distintas formas de organizarnos; desde pueblos y comunidades, pasando por la monarqu\'ia y el feudalismo, hasta los m\'as modernos como la democracia, el socialismo y el comunismo. Dado que nuestro interes no son los sistemas de gobierno, \'unicamente nos centraremos en la democracia.
	
	\par Despu\'es de una \textit{guerra de independencia}, la \textit{revoluci\'on mexicana} y algunas que otras reformas y guerras internas, logramos establecer una democracia en M\'exico; que por cierto, es es las formas de gobierno m\'as usadas en el mundo, aunque no se si la mejor. La democracia busca que todos los grupos sociales que conforman una comunidad	tengan representaci\'on en las decisiones de del pa\'is y nuestro pa\'is no es la exepci\'on, desde que la constituci\'on de 1917, empezamos a reformar aspectos importantes para nuestra naci\'on, que hasta ahora hab\'ian estado en manos corruptas que no permit\'ian satisfacer las necesidades del pueblo. Poco despu\'es de estas reformas. 
	
	\par Y una vez que se crearon las bases de nuestro pa\'is (las instituciones), llego el periodo de mayor propseridad en M\'exico, el milagro mexicano, aunque junto con \'el, pol\'iticas proteccionistas y nacionalistas, que buscaban priorizar a los productos y materias nacionales sobre las extrabjeras; estas pol\'iticas rindieron sus frutos por el contexto golbal (la segunda Guerra Mundial), pero no eran sustentables con el tiempo. cuando esta concluyo, debimos haber cambiado esztas pol\'iticas. Posterior a este periodo, el neoliberalismo empezo a rodear los ideales del nuevo mundo, pero nuestras pol\'iticas proteccionistas nos retrasaron para alcanzarlo. Antes de continuar diremos que
	\begin{quote}\it
		el \textbf{neoliberalismo} es el ideal político y económico que busca limitar la intervención del Estado en asuntos del pueblo, promover el desarrllo interno y la econom\'ia mixta,
	\end{quote}
	además de introducir la teocracia en nuestra forma de gobierno; en cuanto los pol\'iticos perimiteron que estos ideales se establecieran en el pa\'is, nos permitio entrar en la nueva era del mundo, la era \textit{globalizada}. Podemos decir que
	\begin{quote}
		la \textbf{globalización} es el proceso político y económico mediante el cual se impulso la comunicación e interdependencia entre los paises del mundo.
	\end{quote}
	
	\par Me parece facil ver, desde mi perspectiva, que esto nos hubiera beneficiado ampliamente si desde el inicio hubieramos abandoado nuestros ideales proteccionistas, ya que nuestro alto poder industrial, nos habr\'ia permitido alargar al \textit{Milagro Mexicano}. Fuera de lo qeu pudiera o no haber sido, una vez nos sumergimos en la globalizaci\'on, fuimos capaces de comerciar y utilizar tanto la importaci\'on como la exportaci\'on a nuestro pa\'is; ejemplo de esto son los mltiples tratados de libre comercio que han existido entre M\'exico y los paises de Am\'erica. Adem\'as, con el impulso de la comunicaci\'on que porpici\'o la globalizaci\'on, el desarrollo de las ciencias y las tecnolog\'ias fue potenciado, porque de esta manera los investigadores y acad\'emicos de todo el mundo ten\'ian la posibilidad de compartir sus resultados con cualquier otro que estuviera comunicado.
	
	\par Considero que uno de los m\'as grandes pasos que el pa\'is dio en el desarrollo acad\'emico, fue la creaci\'on del \textbf{Consejo Nacional de Educación Superior y de la Investigación Científica}, ya que a pesar de que ya exist\'ian grandes universidades en nuestro pa\'is, este consejo fue el encargado de organizar los apoyos gubernamentales y gestionar los convenios con otras naciones, para de esta manera, poder darle un caracter internacional a la investigaci\'on y la academia en M\'exico. Adem\'as de lo anterior, el priorizar la capacitaci\'on de recursos humanos especificos para cada uno de los centros de investigaci\'on publicos que se estaban creando en el pa\'is, fue una de las  mejores pol\'iticas de impulso de la investigaci\'on en el pais. Es una lastima que actualmente los recursos del ahora \textbf{CONCAYT}, esten siendo recortados de manera tan dr\'astica y con tan poca justificaci\'on; principalmente porrque son los cientificos quienes podr\'ian tratar de solucionar la pandemia que tiene en jaque al mundo actual.
	
	\par Hablando de la biolog\'ia, y especificiamente del area de biotecnolog\'ia, podemos ver en \cite[p\'ag 35]{basica} que, desde los inicios de la ciencia en M\'exico, ha formado parte de los objetivos prioritarios en el desarrollo del pa\'is.
	
	\par En general, considero que el desarrollo de la ciencia y la tecnolog\'ia ha sido bastante bueno, a exepci\'on de el ultimo a\~no que ha recibido muchas reducciones de prespuesto; y a pesar de que las ciencias pueden tener mala fama de personas engreidas y asociales, son las personas que estan guiando el futuro del pa\'is, al menos hasta donde la pol\'itica se los permite.
	
%%%%%%%%%%%%%%%%%%%%%%%%%%%%%%%%
%%         Bibliografia        %%
%%%%%%%%%%%%%%%%%%%%%%%%%%%%%%%%%%

\noindent\makebox[\linewidth]{\rule{\paperwidth}{0.4pt}}
\begin{thebibliography}{X}
	\bibitem{basica} UnADM. (2020). \textit{U3 $|$ EL neoliberalismo y la política de Estado en ciencia y tecnología}. 2 de agosto de 2020, de División de Ciencias de la Salud, biológicas y Ambientales Sitio web: \url{https://campus.unadmexico.mx/contenidos/DCSBA/TC/CSM/unidad_03/descargables/CSM_U3_Contenido.pdf}
\end{thebibliography}

\end{document}