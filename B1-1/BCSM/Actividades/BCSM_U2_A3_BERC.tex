\documentclass[12pt]{article}
\usepackage[spanish]{babel}

\usepackage{amssymb}
\usepackage{enumerate}
\usepackage{geometry}
\usepackage{longtable}
\usepackage{mathtools}
\usepackage{multicol}
\usepackage{soul}

\usepackage{graphicx}
	\graphicspath{ {assets/} }


%%%%%%%%%%%%%%%%%%%%%%%%%%%%%%%%%%
%%%%%%%%%%%%%%%%%%%%%%%%%%%%%   %%
%%        Datos Trabajo     %%  %%
%%%%%%%%%%%%%%%%%%%%%%%%%%%%%%%%%%
\newcommand{\titulo}[0]{Actividad 3. Políticas de crecimiento y desarrollo económico de México en la segunda mitad del siglo XX.}
\newcommand{\materia}[0]{
Contexto socioeconómico de México
}
\newcommand{\grupo}[0]{BI-BCSM-2002-B1-012}
\newcommand{\unidad}[0]{Unidad 2}
%%%%%%%%%%%%%%%%%%%%%%%%%%%%%%%%%%
%%%%%%%%%%%%%%%%%%%%%%%%%%%%%%%%%%
\usepackage[pdftex,
            pdfauthor={bench},
            pdftitle={\titulo},
            pdfsubject={\materia},
            pdfkeywords={\grupo, \unidad, UnADM},
            pdfproducer={Latex with hyperref, or other system},
            pdfcreator={pdflatex, or other tool}]{hyperref}
%%%%%%%%%%%%%%%%%%%%%%%%%%%%%%%%%%
%%%%%%%%%%%%%%%%%%%%%%%%%%%%%%%%%%

\title{
	\includegraphics{../../../assets/logo-unadm} \\\ \\
	Benjam\'in Rivera \\
	\bf{\titulo}\\}

\author{
	Universidad Abierta y a Distancia de México \\
	TSU en Biotecnolog\'ia \\
	\textit{Materia:} \materia \\
	\textit{Grupo:} \grupo \\
	\textit{Unidad:} \unidad \\
	\\
	\textit{Matricula:} ES202105994 }

\date{\textit{Fecha de entrega:} \today}


%%%%%%%%%%%%%%%%%%%%%%%%%%%%%
%%        Documento         %%
%%%%%%%%%%%%%%%%%%%%%%%%%%%%%%%
\begin{document}
\maketitle\newpage

\subsection*{Fichs de trabajo}

	\begin{longtable}[]{@{}ll@{}}

		\begin{minipage}[b]{0.54\columnwidth}\raggedright
			Periodo histórico\strut
		\end{minipage} & 
			\begin{minipage}[b]{0.40\columnwidth}\raggedright
				Primer ciclo recesivo 1913-1916\strut
			\end{minipage}\tabularnewline
		
		\endhead
		
		\begin{minipage}[t]{0.54\columnwidth}\raggedright
			Personajes\strut
		\end{minipage} & 
			\begin{minipage}[t]{0.40\columnwidth}\raggedright
				Venustiano Carranza\strut
			\end{minipage}\tabularnewline
		
		\begin{minipage}[t]{0.54\columnwidth}\raggedright
			Hechos relevantes\strut
		\end{minipage} & 
			\begin{minipage}[t]{0.40\columnwidth}\raggedright
				La revolución mexicana y la promulgación de la constitución de1917\strut
			\end{minipage}\tabularnewline
	
	\end{longtable}.\dotfill

	\begin{longtable}[]{@{}ll@{}}
	
		\begin{minipage}[b]{0.54\columnwidth}\raggedright
			Periodo histórico\strut
		\end{minipage} & 
			\begin{minipage}[b]{0.40\columnwidth}\raggedright
				Segundo ciclo de recuperación 1933-1952\strut
		\end{minipage}\tabularnewline
	
		\endhead
	
		\begin{minipage}[t]{0.54\columnwidth}\raggedright
			Personajes\strut
		\end{minipage} & 
			\begin{minipage}[t]{0.40\columnwidth}\raggedright
				Manuel Avila Camacho, Miguel Aleman Valdés, Antonio Ortiz Mena\strut
			\end{minipage}\tabularnewline
	
		\begin{minipage}[t]{0.54\columnwidth}\raggedright
			Hechos relevantes\strut
		\end{minipage} & 
			\begin{minipage}[t]{0.40\columnwidth}\raggedright
				Crecimiento de la actividad industrial, incremento en producción de energía, se crea \textbf{luz y fuerza de México}\strut
		\end{minipage}\tabularnewline
	
	\end{longtable}.\dotfill

	\begin{longtable}[]{@{}ll@{}}
	
		\begin{minipage}[b]{0.54\columnwidth}\raggedright
			Periodo histórico\strut
		\end{minipage} & 
			\begin{minipage}[b]{0.40\columnwidth}\raggedright
				Fundación del Banco de México\strut
			\end{minipage}\tabularnewline
	
	\endhead
	
		\begin{minipage}[t]{0.54\columnwidth}\raggedright
		Personajes\strut
		\end{minipage} & 
			\begin{minipage}[t]{0.40\columnwidth}\raggedright
				Adolfo de la Hueta, Alberto J.. Pani, Manuel Gómez Morín\strut
			\end{minipage}\tabularnewline
		
		\begin{minipage}[t]{0.54\columnwidth}\raggedright
			Hechos relevantes\strut
		\end{minipage} & 
			\begin{minipage}[t]{0.40\columnwidth}\raggedright
				La Primera Convención Nacional Bancaria, creación de la Comisión Nacional Bancaria, la Reforma Monetario-Bancaria,\strut
		\end{minipage}\tabularnewline

	\end{longtable}.\dotfill

	\begin{longtable}[]{@{}ll@{}}

		\begin{minipage}[b]{0.54\columnwidth}\raggedright
			Periodo histórico\strut
		\end{minipage} & 
			\begin{minipage}[b]{0.40\columnwidth}\raggedright
				Proteccionismo\strut
			\end{minipage}\tabularnewline

	\endhead

		\begin{minipage}[t]{0.54\columnwidth}\raggedright
			Personajes\strut
		\end{minipage} & 
			\begin{minipage}[t]{0.40\columnwidth}\raggedright
				Adolfo López Mateos y Gustavo Díaz Ordaz\strut
		\end{minipage}\tabularnewline
	
		\begin{minipage}[t]{0.54\columnwidth}\raggedright
				Hechos relevantes\strut
		\end{minipage} & 
			\begin{minipage}[t]{0.40\columnwidth}\raggedright
				Altas restricciones a la importación de ciertos productos, favorecimiento de la industria nacional sobre la extranjera, altos impuestos a las importaciones,\strut
		\end{minipage}\tabularnewline

	\end{longtable}.\dotfill
	
	\begin{longtable}[]{@{}ll@{}}

		\begin{minipage}[b]{0.54\columnwidth}\raggedright
			Periodo histórico\strut
		\end{minipage} & 
			\begin{minipage}[b]{0.40\columnwidth}\raggedright
				Milagro Mexicano\strut
			\end{minipage}\tabularnewline
	
	\endhead
		
		\begin{minipage}[t]{0.54\columnwidth}\raggedright
			Personajes\strut
		\end{minipage} & 
			\begin{minipage}[t]{0.40\columnwidth}\raggedright
				Manuel Ácila Camacho, Miguel Alemán Valdez, Adolfo Ruiz Cortines, Adolfo López Mateos y Gustavo Díaz Ordaz\strut
			\end{minipage}\tabularnewline
	
		\begin{minipage}[t]{0.54\columnwidth}\raggedright
			Hechos relevantes\strut
		\end{minipage} & 
			\begin{minipage}[t]{0.40\columnwidth}\raggedright
				Industrialización e instittucionalización del pais, grandes obras de infraestructura en el pais, importantes movimientos sociales y la Fundación del Banco de México\strut
		\end{minipage}\tabularnewline

	\end{longtable}.\dotfill

\subsection*{Reflexi\'on}

\par Pasada la primera mitad del siglo XX, donde podemos destacar la Revolución mexicana, el caudillismo, la Constitución de 1917 y la institucionalización nacional, el país llego a un punto de equilibrio, en donde por fin teniamos paz y control sobre nuestras deccisiones mediante la democracia. En la segunda mitad, aprovechando lo ganado durante la priemra, considero que el proteccionismo fue de las mejor politicas que se pudo haber adpotado.

\par Gracias al proteccionismo, México era una nación industrialmente fuerte cuando la segunda Guerra Mundial y fuimos capaces de satisfacer las necesidades que los otros paises habían dejado de lado por la industria bélica. Lo anterior catapulto la época el Milagro mexicano, es una lastima que no se haya mantenido ese impulos hasta nuestros días.

\par Creo que la razón de la finalización del milagro mexiano fue, principalmente, que no nos adaptamos tan bien como deberíamos a la globalización, y nos aferramos demasiado a una politica parecida al proteccionismo en un munod que pedía más apertura con todo el mundo.






%%%%%%%%%%%%%%%%%%%%%%%%%%%%%%%%
%%         Bibliografia        %%
%%%%%%%%%%%%%%%%%%%%%%%%%%%%%%%%%%
\newpage
\begin{thebibliography}{X}
	\bibitem{biblio} UnADM. (S/D). \textit{U2 $|$ Historia econ\'omica y pol\'itica de M\'exico en el siglo XX}. 28 de Julio de 2020, de Universidad Abierta y a Distancia de M\'exico $|$ DCSBA. Sitio web: \url{https://campus.unadmexico.mx/contenidos/DCSBA/TC/CSM/unidad_02/descargables/CSM_U2_Contenido.pdf} 
	\bibitem{twiter} @Banobras\_MX. (2018). Industrialización y Desarrollo Estabilizador o Milagro Mexicano: México en Expansión 1940-1970. 28 de julio de 2020, de Twiter Sitio web: \url{https://twitter.com/Banobras_mx/status/966386143267840000/photo/1}
	
\end{thebibliography}

\end{document}