\documentclass[12pt]{article}
\usepackage[spanish]{babel}

\usepackage{amssymb}
\usepackage{enumerate}
\usepackage{geometry}
\usepackage{multicol}

\usepackage{graphicx}
	\graphicspath{ {assets/} }


%%%%%%%%%%%%%%%%%%%%%%%%%%%%%%%%%%
%%%%%%%%%%%%%%%%%%%%%%%%%%%%%   %%
%%        Datos Trabajo     %%  %%
%%%%%%%%%%%%%%%%%%%%%%%%%%%%%%%%%%
\newcommand{\titulo}[0]{Actividad colaborativa. Análisis del problema}
\newcommand{\materia}[0]{\'Algebra lineal}
\newcommand{\grupo}[0]{BI-BALI-2002-B1-012}
\newcommand{\unidad}[0]{Unidad 1}
%%%%%%%%%%%%%%%%%%%%%%%%%%%%%%%%%%
%%%%%%%%%%%%%%%%%%%%%%%%%%%%%%%%%%
\usepackage[pdftex,
            pdfauthor={bench},
            pdftitle={\titulo},
            pdfsubject={\materia},
            pdfproducer={Latex with hyperref, or other system},
            pdfcreator={pdflatex, or other tool}]{hyperref}
%%%%%%%%%%%%%%%%%%%%%%%%%%%%%%%%%%
%%%%%%%%%%%%%%%%%%%%%%%%%%%%%%%%%%

\title{
	\includegraphics{../../../assets/logo-unadm} \\
	\ \\ Benjam\'in Rivera \\
	\bf{\titulo}\\\ \\}

\author{
	Universidad Abierta y a Distancia de México \\
	TSU en Biotecnolog\'ia \\
	\textit{Materia:} \materia \\
	\textit{Grupo:} \grupo \\
	\textit{Unidad:} \unidad \\
	\\
	\textit{Matricula:} ES202105994 }

\date{\textit{Fecha de entrega:} \today}


%%%%%%%%%%%%%%%%%%%%%%%%%%%%%
%%        Documento         %%
%%%%%%%%%%%%%%%%%%%%%%%%%%%%%%%
\begin{document}
\maketitle\newpage

\section{Actividad}
	
	\par \textit{Responde la pregunta que formule el docente en línea (mediador del foro), de forma clara y asertiva.}
	\begin{enumerate}
		\item \textbf{Qu\'e son los vectores y cual es su clasificación?} \\ Los vectores tienen muchos usos y definiciones que varian poco a poco entre las distintas disciplinas que los utilizan, las que es más común que los utilicen son la física y las matemáticas. Para las matemáticas, un vector es un arreglo, o conjunto, de datos que representan información variada, que puede ir desde un punto en une espacio de dimensión n, hasta un conjunto de entradas para una función. En la física, los vectores son flechitas que indican la magnitud y dirección de algo. Los que se utilizan en física son los más comunes, y entre ellos destacan los que son fijos, ligados y opuestos.
		
		\item \textbf{Cuál es la información que proporcionan los vectores?} \\ Depende de la interpretación que se les este aplicando a ellos, pero siguiendo el contenido de la unidad, los vectores proporcionan
			\begin{enumerate}
				\item Punto de inicio,
				\item Punto final,
				\item Direcci\'on y
				\item Magnitud
			\end{enumerate}
		
		\item \textbf{Menciona dos ejemplos en la vida cotidiana donde utilices los vectores.} \\ En mi vida cotidiana suelo programar, en estos periodos utilizo los vectores matemáticos constantemente, ya que lo arreglos son una estructura bastante útil y común en la programación, y también utilizo los vectores fisicos cuando estoy trabajando con movimiento de automatas y planeación de trayectorias. Fuera de lo anterior, no hay lugares donde utilice los vectores directamente, sin embargo fenomenos en la vida diaria pueden ser representados con ellos como las fuerzas para equilibrar un columpio, las direcciones de los vehiculos en una calle o el movimiento del sol en la boveda celeste.
		
		\item \textbf{Qué son los productos vectoriales y triples productos?} \\ El producto vectorial es una operación binaria que definimos entre vectores para poder hacer manipulaciones más intuitivas, geometricamente el resultado es un vector perpendicular al plano que contiene a los dos vectores. Y los triples productos son las formas en que podemos relacionar las operaciones de producto vectorial y de vector por escalar

	\end{enumerate}
\section{Retroalimentaci\'on}
	
	\includegraphics[width=0.9\textwidth]{AL-U1-A1-1}


\section*{Fuentes de informaci\'on}
	\begin{itemize}
		\item UnADM. (S/D). \textit{Primer semestre Algebra Lineal}. 13 de Julio de 2020, de Universidad Abierta y a Distancia de M\'exico, DCSBA. Sitio web: \url{https://dmd.unadmexico.mx/contenidos/DCSBA/BLOQUE1/BI/01/BALI/unidad_01/descargables/BALI_U1_Contenido.pdf}
		\item Bernard Kolman, David R. Hill. (2006). \textit{Álgebra lineal} (8a. Edición). México: Pearson Educación
	\end{itemize}

\end{document}